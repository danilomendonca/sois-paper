\section{Motivating Scenario}

%Next, we present two applications examples for which one or more of these requirements apply.

Mobile crowdsensing consists of a paradigm in which the sensors of user-companioned devices are employed in the measurement of urban and social phenomena~\cite{}. +++

Many crowdsensing campaigns target a well defined geographic region and period of time. 
%Accordingly, the mobile devices running the application may either be within the same Wi-Fi zone or able to interact using a device-to-device (D2D) communication technology. 
Today, popular mobile crowdsensing applications range from private and public urban transportation monitoring (e.g., Waze\footnote{waze website} and Moovit\footnote{moovit website}) to atmospheric pressure and air polution (e.g., AirPatrol\footnote{airpatrol website} and PressureNet\footnote{pressurenet website}). In the first two cases, geolocation is the main type of data, which is fetched from GPS sensors. In AirPatrol and other air polution monitoring applications, an additional sensor that monitors the quality of the air and communicates with smartphones through bluetooth must be carried by users. 

In general, crowdsensing involves the activation of battery-draining sensors which would otherwise be idle or used sporadically. Also, the majority of nowadays applications are designed following a client-server architecture in which application nodes collect data from their sensors independently from one another. This approach has the following drawbacks:

\begin{itemize}
	
	\item A backend server coordinates the activation of the sensing tasks, which increases the server and the network loads; or
	
	\item Each application node perform the same tasks -- there is no coordination nor collaboration between nodes; 
	
\end{itemize}

Without coordination, application nodes cannot adapt to situations in which multiple devices can provide the same information, thus they tend to consume unnecessary resources as sensing tasks could be allocated to a subset of the available devices. With local coordination, nodes can perceive each other and agree on which nodes should perform which tasks, as well as to collectively adapt to changes in physical and social contexts (e.g., churn of devices, variations in the quality of sensors measurements, battery level, etc). 

Without collaboration, each node sends its data to the backend server responsible for filtering and aggregating samples, which also increases the server and the network loads. Additionally, all nodes are assumed to have Internet access to communicate with the server and no collaborative routing of data is used. With local collaboration, application nodes can aggregate data collected from different sensors in order to select those with best accuracy. Accordingly, less data has to be transmitted to and processed by the backend server.

\section{Background}

The kind of application targeted by the proposed framework are those with the following characteristics:

\begin{itemize}
	
	\item \textbf{Mobility:} application nodes are hosted by mobile devices that can enter/leave a given zone/network dynamically; 
	
	\item \textbf{Volatility:} the environment in which application nodes operate is subject to frequent changes that may affect their capabilities to perform application tasks;
	
	\item \textbf{Large scale:} a large number of application nodes is expected to co-exist in a given zone/network;
	
	\item \textbf{Data locality:} application nodes are 
	both producers and consumers of data that must be shared locally;
	
	%\item \textbf{Complexity:} application nodes are expected to perform data/computational-intensive tasks 
	
	%\item \textbf{Delay-sensitive:} data should be exchanged between different application nodes with minimum delay.
	
	%	\item Ad hoc: the interaction between application nodes are expected to happen opportunistically without the intermediation or support of dedicated servers.
	
	%	\item Transient: application data is created, shared, and modified locally; remote persistence, when needed, should happen asynchronously.
	
	%poses the challenge of dealing with scalability issues; thus, the algorithms and methods that support the ad hoc interaction among application nodes must scale to a large number of interacting devices.
	
	%\item Robustness: responsibilities assigned to local devices that leave the system or fail must be resumed by other available devices without disruption of normal behavior.
	%
	%\item Efficiency: responsibilities assigned to local devices must take into account their functional and non-functional capabilities according to system specification.
	%
	%\item Fairness: responsibilities assigned to local devices must take into account their participation history to avoid unfair use of devices resources.
	
	%the problems caused by the intermittent connectivity to remote servers should be mitigated either by opportunistically assigning responsibilities to local devices. 
	
\end{itemize}


%This last example brings more complexity to the ad hoc edge space formation and maintenance. To illustrate it, let's consider a city in which public transport agency provides no real time information of the whereabouts of buses. A collaborative crowdsensing application has bee designed to collect this information from the GPS in passengers devices. Users participation is conditioned to a low battery consumption and mobile data plan usage. Next, we describe how the proposed Ad Hoc Edge Spaces features can help to improve these metrics when compared to a self-contained, cloud-based solution.
%
%The physical-virtual space mapping in this example is clear: each bus is a physical space where an ad hoc edge space may exist. In a cloud-based solution, each mobile device would host a self-contained application that would collect the GPS data and send it to the server. Instead, the first device to enter the bus creates a \textit{bus monitoring group}, in which two roles can be performed: \textit{GPS fetcher} and \textit{GPS aggregator}. While the former requires just one instance, the GPS fetcher role is replicated to increase geolocation data accuracy, as it varies in time and from one device to another. Each fetcher sends the data to the aggregator to be validated and sent to the backend servers to be further processed and shared with users waiting for that bus.
%
%The advantages of this solution are: total GPS use is reduced, as not all devices must play the GPS fetcher role; quality of the data sent to backend servers is improved, while the quantity is reduce, thus reducing the use of the mobile data plan and the load of the backend servers. Considering the plethora of sensors in the smartphones, the efficiency of mobile crowdsensing campaigns could be significantly improved by reproducing this model with other types of data.

Each of these characteristics imposes challenges to the engineering of mobile applications for which interactions among nodes is a requirement. Next, we discuss the most important requirements related to the realization of such applications, the current solutions and the gaps to achieve these requirements.

\subsection{Discovery and Contextualization}

%FIGURE: MOBILE DEVICES NOT/RUNNING A GIVEN APPLICATION INTERCONNECTED THROUGH WI-FI

\begin{figure}[t!]
	\centering
	\includegraphics[width=0.45\textwidth]{figures/discovery}
	\caption{Application nodes in the same network must become aware of each other's presence}
	\label{fig:discovery}
\end{figure}

%TODO provide details about IoTivity

%The realization of an interaction space poses many technical challenges. First, 
The application nodes sharing the same network must become aware of each other, mainly for two reasons: 1) application users must be aware of who they can interact with (e.g., in a chatting or a multiplayer game application); or 2) the social context  

In a black-box model, 

%Content and control can be broadcasted;  

Popular platforms such as Android and iOS provide native mechanisms that allow nodes to advertise and discover their services and resources~\cite{ANDROID_NSD, IOS_Bonjour}. Additionally, third party frameworks also provide discovery and other features based on different paradigms and protocols~\cite{Alljoyn, IoTivity}. 

In mobile crowdsensing, the discovery of peer nodes is particularly important as each node is a potential contributor to the existing campaigns. 

%TALK ABOUT THE DCVRY METHODS AND THEIR SUITABILITY TO THE SCENARIO AND REQUIREMENTS PRESENTED BY THIS WORK.

Whereas previous coordinating models for mobile computing assumed the worse case for connectivity (nodes eventually meet and exchange data)~\cite{LIME, TUCSON, OTHERS}, nowadays wireless technologies makes it possible for multiple application nodes to be interconnected at the same time. Given the resource limitations of mobile devices, discovery mechanisms must avoid creating too much computational overhead or flood the network with discovery messages. As the peer scale grows, such requirements becomes critic for the feasibility of an edge space for interactive applications. 

\subsection{Interoperation}



\subsection{Asymmetry}

In addition to discovery and contextualization, the dynamic allocation of tasks to different application nodes is a central feature that further distinguishes this work from previous frameworks~\cite{}. Similarly to well studied allocation of tasks to agents/robots with dynamic and distinct capabilities~\cite{}, mobile devices are heterogeneous and subject to hight volatility, therefore their fitness in performing different tasks varies from one device to the other and throughout time. 

Asymmetry is more evident when mobile application instances are designed to be more than just clients of remote servers. For example, in opportunistic mobile crowdsensing, data must be fetched from devices sensors without user participation. Application nodes may autonomously collaborate by aggregating fetched data from multiple devices into a single device~\cite{Rajagopalan:2006}, which in turn sends a preprocessed version to the server~\cite{}. Thus, one-out-of-many nodes must assume the role of an aggregator (and perform aggregation). Moreover, if the number of application nodes at a given moment is high, not every node should perform the same sensing tasks. Instead, a subset of application nodes can be assigned to specific sensing task(s), reducing the overall battery and mobile data consumption. Figure~\ref{fig:asymmetry} presents some examples of symmetric and asymmetric behavior within groups.

%FIGURE: CIRCLE WITH HALF SYMMETRIC HALF ASYMMETRIC BEHAVIOR

\begin{figure}[t!]
	\centering
	\includegraphics[width=0.48\textwidth]{figures/asymmetry}
	\caption{Left: application nodes perform the same behavior (e.g., content sharing); middle: application nodes perform distinct behaviors without hierarchy (e.g., different sensing tasks); right: one node performs an hierarchical behavior (e.g., coordinator, data aggregator)}
	\label{fig:asymmetry}
\end{figure}

\subsection{Adaptation}

