The large scale adoption of pervasive computing constitutes a new scenario for which new types of applications can be designed. This scenario has specific characteristics, such as a high heterogeneity and volatility, intermittent connectivity, and resource limitations. To further explore its potential, applications must be able to adapt to context changes. In specific, they must be able to self-organize its components to seamlessly achieve application goals and make proper use of resources. 
%As more and more devices join the network or because of low latency requirements, a server-base architecture in which servers in the cloud handle most of the computation may not deliver the expected results. Conversely, 
%Many devices are now equipped with more powerful resources. This enables more computation to be performed at the edge of the network infrastructure. Some applications already explore machine-to-machine (M2M) communication to allow distributed components to interact and coordinate their behavior through local and mobile ad hoc networks (MANETs). Despite the potential benefits, an horizontal architectural must also deal with problems related to the churn of devices, consistency, robustness, availability, resource limitations, fairness, dynamic policies, security, and others.  

In the literature, data-centric models are the state-of-art for coordinating distributed components, as they enable both space and time decoupling. However, their high abstraction leaves many aspects open. In this work, we propose a model for the development of self-organizing and distributed applications that explore the potential of nowadays pervasive computing. Our model inherits well-known organizational abstractions, namely groups and roles, which allow a flexible modularization of the system and a dynamic assignment of activities among heterogeneous devices in scenarios of high volatility. In addition, we detail the self-organization mechanisms and their incorporation by a middleware.  Finally, we have showed the use of our model in different application scenarios.