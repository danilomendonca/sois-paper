%\subsection{Temperature Consensus}
%
%In many places, it is often the case that co-workers have disagreements on the temperature to be set in the physical environment they share. The ideal temperature for the environment varies from person to person, leading to conflicts among colleagues. To mitigate this problem, the air-conditioning temperature control should be set to a value that corresponds to the average of the current preferences in the office. Willing to solve this problem, a software engineer proposed the creation of an application for smartphones with a simple interface to allow users to set their preferred temperature value. However, its request for a backend application was denied and the temperature consensus problem would have to be solved locally.
%
%In such example, each company office represents the physical space delimitation where mobile devices can interact to achieve a common goal (temperature consensus). Also, devices are assumed to be interconnected through a corporative Wi-Fi. Finally, the air-conditioning controller exposes an interface through which commands can be sent with a temperature parameter. While all devices must host the basic user interface, only a single device in the room should aggregate co-workers' input, average them and communicate with the air-conditioner controller. Thus, devices must not only be able to communicate, but also elect the one responsible to play the aggregator role.
%
%A possible solution is described as follows: the first person to enter the office triggers the creation of a \textit{temperature control group}, which is a basic organization structure to which system roles are binded to. This device assumes the \textit{aggregator role}, which is responsible for keeping the group state (members and their preferred temperature). Next devices to enter the room shall join the existing group. After taking the average of the preferred temperatures each time a device enters or leaves the room, the aggregator value is sent to the air-conditioner controller, which in turn adjusts the temperature in the office. 
%
%This example illustrates how a realistic solution can be achieved through ad hoc and local interactions. The \textit{ad hoc} aspect becomes clear once there is no infrastructure for hosting or supporting the application besides the mobile devices themselves and the air-conditioner controller. If all workers leave the office with their devices, the edge space disappears. Also, the mapping between an office and the temperature control group provides a crispy correlation among physical and virtual spaces. Finally, the aggregator role represents a local instance of a server with which other users devices interact as clients. 
%
%%In the example, the data managing role keeps transient data about the preferred temperature level. To avoid fetching this data from all members and improve its availability, the data must be replicated by other devices. The replication degree may be fixed or proportional to the group size or volatility, i.e., the rate in which devices leave the group. For instance, a replication degree of 25\% would mean that an office with 8 devices shall have two replicas of the data managing role. When the active role quits the system, one of the two replicas should be activated. The role election and the realization of different role replication schemes at runtime is a central contribution of this work. As the criteria used for electing a role and its replica may require information from participants, it is important to consider the communication overhead in the trade off between availability and performance of the distributed election algorithm. The proposed election and replication mechanisms are fully described in Section~\ref{sec:model}.
%
%%We show how a simple application based on our framework could mitigate such problem.
%