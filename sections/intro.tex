%!TEX root = main.tex
% -*- root: main.tex -*-
\section{Introduction}
\label{sec:intro}
%context and motivation: in this paragraph I present the new scenario composed of heterogeneous devices

%TODO: Bring Internet of Things to the paragraph and make mobile devices and applications more important; (deprecated)

%1: The scenario of pervasive and mobile computing today
In the last decade, the world has witnessed the massive popularization of pervasive and mobile devices. Some of them are equipped with more powerful CPUs, memory, and storage (e.g. modern smartphones and tablets), while others are designed for specific functions and have limited resources (e.g., sensors and gadgets). In particular, applications hosted by these devices are increasing in number and complexity. 

\begin{figure}[t!]
	\centering
	\includegraphics[width=0.7\linewidth]{figures/pervasive_devices}
	\caption{Different classes of pervasive devices; a) a laptop; b) a Raspberry PI; c) a smartphone; d) a tablet; e) a gadget (watch); f) a Bluetooth beacon}
	\label{fig:pervasive_devices}
\end{figure}

%2: Today's mainstream architectural model
Today, the mainstream architectural model for pervasive applications consists of: 1) a client application running at users devices; 2) a backend application hosted by cloud servers. Accordingly, data produced by these applications at the edge of the network is sent to the cloud, processed there and sent back to the edge. As good as this model has proven to be, there are existing and emerging use cases that cannot afford the latency introduced by networking with distant servers, cannot assume a stable and reliable connection with remote servers, or users are not willing to accept the costs of remote communication, specially if part of the data can be processed and consumed locally.

%3: Examples of P2P based solutions
%TODO: rethink the purpose of this paragraph and fix it
In its simplest form, a device-to-device communication targets the exchange of application data. However, novel applications are targeting more complex and sophisticated goals (e.g., to autonomously sense the physical environment in specific areas of a city). In this paper, we argue that some of these goals can be achieved more efficiently through the direct communication and collaboration between the application nodes. %In specific, we target the serendipitous situations in which the devices hosting these applications become visible to each other and are able to communicate. 

%3: Functional plasticity + relations between nodes 
In this paper, we focus on the relations that can be formed (and dissolved) between the application nodes composing a pervasive ecosystem. In specific, given the functional plasticity of today's pervasive and mobile devices equipped with more powerful computational resources and several types of sensors, 
%we argue that 
%sense and actuate the environment, or even to perform computation-intensive tasks,
new types of relations -- others than just the one between clients and backend servers -- can be opportunistically created and dissolved among devices at the edge of the network, i.e., without intermediation of remote servers. These relations must cope with the heterogeneity and volatility characterizing the pervasive ecosystem, including the mobility of nodes and the fluctuations of their connectivity and resources. 


%4: Contributions
\textbf{Contribution of the work:} 
%4.1: SOIS concepts
as contribution, this paper introduce the concept of a \textit{self-organizing interaction spaces} (SOIS) as part of a framework for the engineering of pervasive applications. 
%that require or benefit from the interaction between its nodes. 
%4.1a: The SOIS rationale
Firstly, we discuss the rationale of how some existing and emerging applications goals, empowered by nowadays technologies for wireless communication and the functional plasticity of its hosting devices, can be met by letting nodes to assume distinct functional roles. 
%4.1b: The self-organization mechanisms
Secondly, we provide the organization abstractions for modeling and programming these applications. Finally, we propose the self-organization mechanisms to maintain and evolve the organization structure based on the social and individual contexts of hosting devices.
%application nodes capabilities and state in the formation and adaptation of the ecosystem organization. 
%Finally, we present the requirements for a middleware to support the programming and operation of pervasive applications based on SOIS.

The feasibility of the framework has been demonstrated with case examples of pervasive applications and the complexity of the self-organization mechanisms evaluated by means of asymptotic analysis and a simulation of a mobile crowd-sensing application.
%was evaluated with a simulation of mobile crowdsensing application. %The results showed the feasibility of the approach and, 
In comparison with a pure client-server model, the SOIS based solution reduced the overall battery consumption in X\% and the volume of data sent to backend servers in Y\%, while preserving the low-latency requirements.

%5: How the paper is structured
The paper is structured as follows: Section~\ref{sec:background} brings relevant background information; Section~\ref{sec:overview} and the application example used to motivate and illustrate the framework, relating it to the characteristics of targeted applications. Section~\ref{sec:edge_spaces} presents the organization abstractions and the self-organization mechanisms composing the framework. Details about these mechanisms are provided in Section~\ref{sec:self_organization}, whereas Section~\ref{sec:evaluation} reports on the evaluation of our framework with a mobile crowd-sensing application. Finally, Section~\ref{sec:related_work} compares SOIS with related works and Section~\ref{sec:conclusion} concludes this paper with final considerations and future works.
