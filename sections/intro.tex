%!TEX root = main.tex
% -*- root: main.tex -*-
\section{Introduction}
\label{sec:intro}

% % % % % % %

%Pervasive Computing + Opportunistic Interactions through Wireless Networks

%Asymmetry + role assignment

%Decentralized self-adaptation in pervasive computing

% % % % % % %

%1: The scenario of pervasive and mobile computing today
In the last decade, the world has witnessed the massive diffusion of pervasive and mobile devices. Some of them are equipped with more powerful CPUs, memory, and storage (e.g. modern smartphones and tablets), others are designed for specific functions and have limited resources (e.g., sensors and gadgets). In both cases, applications hosted by these devices are increasing in number and complexity. 
%
%\begin{figure}[t!]
%	\centering
%	\includegraphics[width=0.7\linewidth]{figures/pervasive_devices}
%	\caption{Different classes of pervasive devices; a) a laptop; b) a Raspberry PI; c) a smartphone; d) a tablet; e) a gadget (watch); f) a Bluetooth beacon}
%	\label{fig:pervasive_devices}
%\end{figure}

%2: Today's mainstream architectural model
Today, pervasive applications are often organized around: 1) a client application running on users' devices; 2) a backend application hosted on cloud servers. Accordingly, data produced by these applications at the edge of the network are sent to the cloud, processed there and sent back to the edge. Even if this model is appropriate in many situations, there are existing and emerging use cases that cannot afford the latency introduced by communicating with distant servers, cannot assume a stable and reliable connection with them, or users are not willing to accept the costs of remote communication, specially if part of the data can be processed and consumed locally.

%3: Why not the existing proposals
The literature covering device-to-device interaction in pervasive computing is vast and range from multi-agent platforms to peer-to-peer protocols. This body of knowledge includes specific proposals for distributed coordination and collaboration. However, existing proposals targeting local and opportunistic interactions either tackle a narrow set of goals (e.g., content lookup and sharing), or they require a centralized platform to work (JADE platform for multi-agents). 
% in scenarios of mobility and constrained resources.

%4: Examples of P2P/D2D based solutions
In its simplest form, a device-to-device interaction targets the exchange of application data (e.g., content generated by users). However, some features are characterized by their autonomy 
%are targeting more complex and sophisticated goals 
(e.g., to autonomously decide when to sense the physical environment in specific areas of a city), their intensity (e.g., battery-consuming and data-intensive tasks), or by their sensitivity to delay (e.g., players in a mobile multiplayer game). In these and other cases, devices may become responsible for distinguished roles depending on their social and individual contexts. In this work we demonstrate how simple organizational abstractions, namely, roles and groups, can be employed to engineer pervasive applications that explore the serendipitous interactions among devices. We additionally show how classic self-organization principles (bottom-up) can be combined with top-down architecture-based self-adaptation to cope with the specific constraints of pervasive computing.

%In this paper, we tackle both conceptual and concrete aspects that can enable the adoption of opportunistic interactions among pervasive devices for a broader range of applications.

%These features may require the coordination among instances of an application hosted by different devices (hereafter referred to as \textit{application nodes}). 
%4: The paper's proposal in terms of 
%We argue that existing and novel features from pervasive applications can be achieved more efficiently by exploring serendipitous interactions among devices. % generated by users' mobility and opportunistic wireless communications.

%letting the application hosted by different devices (hereafter referred to as application \textit{nodes}) to 

%%In specific, we target the serendipitous situations in which the devices hosting these applications become visible to each other and are able to communicate. 
%%3: Functional plasticity + relations between nodes 
%In specific, this work focuses on the relations that can be formed (and dissolved) among the devices composing a pervasive ecosystem. 
%%Given the functional plasticity of today's 
%Since pervasive and mobile devices are equipped with more powerful computational resources and several types of sensors, 
%%we argue that 
%%sense and actuate the environment, or even to perform computation-intensive tasks,
%new types of relations -- others than just the one between clients and backend servers -- can be opportunistically created and dissolved among devices at the edge of the network, i.e., without intermediation of remote servers. These relations must cope with the heterogeneity and volatility that characterize the pervasive ecosystem, including the mobility of nodes and the fluctuations of their connectivity and resources. 


%%4: Contributions
%\textbf{Contribution of the work:} 
%%4.1: SOIS concepts
%as contribution, the paper introduces the concept of a \textit{self-organizing interaction spaces} (SOIS) 
%%as part of a framework for the 
%for engineering pervasive applications. 
%%that require or benefit from the interaction between its nodes. 
%%4.1a: The SOIS rationale
%Firstly, we discuss the rationale of how some existing and emerging application features, empowered by today's technologies for wireless communication, 
%%and the functional plasticity of its hosting devices, 
%can be met by letting application nodes assume distinct functional roles. 
%%4.1b: The self-organization mechanisms
%Secondly, we provide the abstractions for modeling and programming these applications. Finally, we propose the self-organization mechanisms that maintain and evolve the structure of the application based on the individual and social contexts of devices.
%%application nodes capabilities and state in the formation and adaptation of the ecosystem organization. 
%%Finally, we present the requirements for a middleware to support the programming and operation of pervasive applications based on SOIS.


%The feasibility of the proposal is demonstrated on examples of pervasive applications and the complexity of the self-organization mechanisms evaluated by means of asymptotic analysis and a simulation of a mobile crowd-sensing application.
%%was evaluated with a simulation of mobile crowdsensing application. %The results showed the feasibility of the approach and, 
%In comparison with a pure client-server model, the SOIS based solution reduced the overall battery consumption in X\% and the volume of data sent to backend servers in Y\%, while preserving low-latency.

%5: How the paper is structured
The paper is structured as follows: Section~\ref{sec:background} introduces relevant background information; Section~\ref{sec:overview} presents the proposed solution and uses an example application to motivate it and illustrate its characteristics. Section~\ref{sec:edge_spaces} presents the modeling and programming abstractions, whereas the self-organization mechanisms are presented in Section~\ref{sec:self_organization}. Section~\ref{sec:evaluation} reports on the evaluation of our framework with a mobile crowd-sensing application. Section~\ref{sec:related_work} surveys related work and Section~\ref{sec:conclusion} concludes the paper.