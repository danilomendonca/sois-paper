

\section{Evaluation and Discussion}\label{sec:evaluation}



The simulation experiments aimed at showing the benefits of the approach and measuring the overhead imposed by the self-organization mechanisms.
%, namely self-grouping and distributed role allocation. 
As these methods add no significant overhead in terms of processing or memory, an asymptotic analysis focused on the communication overhead.
%, as the exchange of messages through wireless mediums consumes battery from devices and is subject to delays that may interfere with the application behavior. 
As for the benefits of the approach, we executed simulation experiments of a public transport monitoring application using a pure client-server and using our framework. The goal was to compare, in each case, the following two metrics:

\begin{enumerate}[label=\textbf{M}\arabic*:]
	
	\item the total number of sensing tasks performed and requisitions fired from clients to backend servers (evaluates battery consumption and Internet traffic); and
	
	\item the total number of failures in reaching backend servers due to intermittent connectivity (evaluates robustness).
	
\end{enumerate}

%In specific, this metric was evaluated 

%\begin{enumerate}[label=\Alph*]
%	
%	\item The asymptotic overhead, given by worse case number of exchanged messages required by each self-organization mechanism, as a function of the number of application nodes, groups and roles (Section~\ref{fig:asymmetry}). 
%	
%	\item The measured overhead, given by the number of exchanged messages counted during simulated executions of a MCS application for public bus monitoring, in which the number of group/roles is fixed and the number of nodes, as well as their capabilities, varies according to probabilistic distributions.
%	
%\end{enumerate}


\subsection{Asymptotic Analysis} 

\subsubsection{\textbf{Self-grouping}} the complexity analysis was divided in two parts: a) the overhead when a node joins/leaves a group (registry update); b) the additional overhead when a node joins a group (registry copy).

In the worse case scenario, registry update (a) takes $n-1$ unicast messages ($O(n)$), with $n$ the group size, and a single registry line as payload. However, if a broadcast communication is used, a single broadcast message (e.g., an UDP broadcast over Wi-Fi) can advertise the registry update ($O(1)$).
%This is the case, e.g., of an UDP broadcast over the 802.11 network protocol (Wi-Fi). 
The registry copy (b), in its turn, requires a single unicast message to be transmitted each time a node joins a group ($O(1)$). Important to mention, in contrast with the registry update message, the registry copy includes information about all $n-1$ nodes in the group. 

\subsubsection{\textbf{Discentralized Role Allocation}} 

%TODO: check the definition of FS and use it accross the paper

this mechanism involves the exchange of fitness scores ($FS$) among eligible nodes in the advent of the events depicted in the previous subsection. The actual number of $FS$ messages sent/received during a role position election depends on how many nodes in the $n$ size group satisfy the role restrictive criteria (RRC) (see Eq.~\ref{eq:rrc} in Section~\ref{sec:edge_spaces}). Once more, the type of network is a determining factor.

In the worse case, represented by an election of a vacant position (following a \textit{vacancy} or \textit{resignation} event) and without broadcast, each node $e$ in the set of eligible nodes $E$ must send each other an unicast message. If $|E| = n$, the message count is given by $O(n * (n-1)) = O(n^2)$. If the nodes in $E$ can communicate through broadcast, this number is reduced to $O(n)$. In its turn, the challenge event produces smaller overhead as the exchange of messages is restricted to a request-response between challenger and challenged and to the subsequent advertisement of the result ($O(n)$ without broadcast, otherwise $(O(1))$). 

%added to the node's classification list in polynomial time ($O(p*log(p)))$, with $1 \le p \le n$. 

Once the communication overhead of a single election round is known, the overall overhead can be estimated by the number of groups and roles in the application and the frequency of events triggering new elections. Whereas the vacancy and resignation events should be handled with a new election, the frequency in which a challenge event happens is greatly affected by the choice for $\delta$: the higher this factor is, the lower the probability of a challenge (and the need for the elected nodes to update their peers about changes in their $FS$). Therefore, the decision of which $\delta$ to be used depends on the criticality of the attributes that compose the fitness score of a node. 



\subsection{Public Transport Monitoring Simulation}

\subsubsection{Experiments Design}

the group-role specification (List.~\ref{lst:bm_criteria} in Section~\ref{sec:edge_spaces}) was used for the modeling of a dynamic bus monitoring application scenario. In it, the following variables were considered:

\begin{itemize}
	
	\item the number of nodes within a bus: \textit{from 2 to 10 nodes};
	
	\item the battery level of each node: \textit{from 10\% to 90\%};
	
	\item the Internet type of each node: \textit{cellular or Wi-Fi};
	
	\item the GPS signal level in each node: \textit{from 0\% to 100\%};
	
\end{itemize}

In the client-server approach, each node tries to collect data from its accelerometer and GPS before sending it to the server through whatever type of Internet connection available (Wi-Fi or cellular data plan). In our approach, at most two simultaneous nodes are assigned to each type of sensor and one node to aggregate data before sending a preprocessed batch to the server. The number of times each sensor is fetched, as well as the number of times a connection between a client/aggregator and the server, were counted in separate (\textbf{M1}). Finally, we also counted the monitoring windows in which no measurement reached the server (\textbf{M2}).

%We modeled the cost of sampling the GPS as 5 power units (pw) and 1pw for the accelerometer, as in most cases it works continuously, but still requires a background activity running. Plus, we modeled the cost of sending data to the server through cellular plan as 10, through Wi-Fi as 5, and through 

The experiments were performed with, \textit{PeerSim}, an open source peer-to-peer simulator~\cite{p2p09-peersim}. This tool supports the creations of different network P2P topologies and provides a handful JAVA programming interface.



%To simulate a public bus monitoring crowd-sensing application, experiments were performed with an increasing number of nodes. Each role can potentially play a \textit{gps-monitor}, a \textit{acceleration-monitor}, or an \textit{aggregator} role. Each role has been implemented as a protocol extending the \textit{CDProtocol} class. At each simulation cycle, 
%
%the application roles of a \textit{gps-monitor} and \textit{acceleration-monitor} were implemented as protocols, which are called at each simulation cycle. 
%
%The simulation was executed in a computer with .... 

\subsection{Results}

Add Figures here

\subsection{Discussion}

Discuss the obtained results