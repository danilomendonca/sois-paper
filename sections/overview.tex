%Why do we need this section in the paper?
%This section provides the overview over the main concepts of the paper, as well as a general comparison with existing work (to explain why a new framework is needed). 
%The detailed comparison with other proposals is found in the Related Work Section.
\section{Overview}\label{sec:overview}

%This subsection contextualizes the reader w.r.t. existing fields in which OI take place and explains why they fail to address the specific requirements for engineering PA.
%Do we need to discuss the past of LOI? Why? Here?
%R: we cannot ignore MAS research as it tackles similar issues regarding coordination and collaboration among distributed entities (DAI). The section is correct as it brings info about state-of-art research and discuss why they are not enough for what we need.
\subsection{Opportunistic D2D Interactions}

For opportunistic D2D interactions we refer to the situated engagement of pervasive devices as they are able to communicate through wireless networks, including both infrastructure-based (e.g., Wi-Fi) and ad hoc (Wi-Fi direct, Bluetooth, etc). 

%4: Examples of P2P/D2D based solutions
%TODO fix the intensity example
In its simplest form, a D2D interaction targets the exchange of application data (e.g., content generated by users in a mobile P2P application). However, some features are characterized by their autonomy 
%are targeting more complex and sophisticated goals 
(e.g., to autonomously decide when to sense the physical environment in specific areas of a city), their intensity (e.g., process and data-intensive tasks), or by their sensitivity to delay (e.g., players in a mobile multiplayer game). 

Much of today's pervasive applications are structured around the interactions remote servers (e.g., through RESTful endpoints) and users (e.g., through Activities in Android). % and follow a model-view-controller architectural pattern. 
%remote servers; 
Opportunistic D2D interactions are not common, despite its potential to enable new features and to improve non-functional attributes like availability, cost, and communication performance. There are many possible reasons for this, including the complexity of existing solutions and scalability problems.

Opportunistic interactions have been addressed by different research areas. 
Distributed artificial intelligence has made substantial contributions by means of the agent abstraction and the decentralized mechanisms that govern their collective behavior and adaptation~\cite{}. Other research communities also dedicated efforts in the same direction. Nonetheless, many of the proposals~\cite{} tackle specific types of interaction and do not provide a more general model that could be tailored for different applications. In contrast, in this work we follow a top-down approach in which modeling and programming abstractions are first introduced. Based on these building blocks we propose mechanisms and protocols tailored for particular scenarios of pervasive computing, namely different levels of volatility, scale and resource constraints.

%Nonetheless, there are significant differences with respect to the specification, design and programming of multi-agent systems (MAS) from how software engineers develop pervasive applications (PA)~\cite{}. Whereas multi-robots and other MAS based solutions are designed for interaction among agents, much of today's PA are structured around the interaction with users (e.g., through Activities in Android) and % and follow a model-view-controller architectural pattern. 
%remote servers; device-to-device interactions are not common. %, despite its potential to enable new features and to improve non-functional attributes like availability and communication performance. 
%%Finally, as already stated elsewhere, agents have the dual responsibility of performing application tasks and managing the interaction space. %TODO add Weyns critizism to the double role of agents.. 
%Despite the potential of adopting MAS solutions for enabling opportunistic interactions in pervasive applications, these significant differences, both at the conceptual and the operational levels, must be taken into account.



%%Finally, as already stated elsewhere, agents have the dual responsibility of performing application tasks and managing the interaction space. %TODO add Weyns critizism to the double role of agents.. 
%Despite the potential of adopting MAS solutions for enabling opportunistic interactions in pervasive applications, these significant differences, both at the conceptual and the operational levels, must be taken into account.



%Such differences must be taken into account when adapting solutions from MAS community.

%must not disrupt, but complement well established software engineering principles and practices.



%Conti et al. have proposed a middleware for opportunistic mobile social networks.  

%There ha also been extensive work to enable opportunistic interactions in pervasive computing research community~\cite{}. In many cases, the proposed solutions are narrow with respect to the type of interaction they aim to achieve; in others, the proposed solutions... %TODO: this is a critical point. I have yet to talk about Weyns (SA) and Zambonelli (SO). But before them, there's a vast literature from PC community that cannot be ignored. Why their proposals are not good enough? Because they don't deal with evolution/adaptation?



%must not disrupt, but complement well established software engineering principles and practices. In this sense, a pure agent-oriented approach fail to meet such requirement.  

%there is still a gap, both at the conceptual and concrete levels, of how software engineering community can make use of the relevant solutions from distributed artificial intelligence to the development of pervasive applications. % that exhibit similar requirements for decentralization, context-awareness, and adaptation. 

%Precisely, what does SoR means? 
\subsection{Separation of Responsibilities}



%3: functional plasticity of nowaday's devices
With the technology advancements, a class of pervasive devices has become able to perform different kinds of computations, as well as to communicate by different means and perceive the physical world through multiple sensors. As a consequence, the type of interaction among pervasive devices is less frequently defined by their model and specific functionality. Instead, it may depend on dynamic factors like their current resource levels and mobility.

%Moreover, a single application may contain software components with responsibilities and purposes that are activated in specific contexts. For instance, one device may become the coordinator for a period of time, while coordinated devices may become responsible for monitoring the environment through their sensors. 

%The consequence of this change is twofold. 
%
%First, 
%
%Second, 

To address this new scenario, the original focus of pervasive computing research targeting compositions and interactions among devices with specific functionality~\cite{Schumman} must now consider the ambiguity in the runtime and context-dependent decision of which devices should become responsible for what. %while they engage in opportunistic interactions for various purposes. 
Also, applying well known software engineering principles like separation of concerns and modularization to the distinct functionalities a device may become responsible should improve the quality of the pervasive application design and implementation, including the autonomous and contextual responsibility allocation.

%These enhancements entails a \textit{functional plasticity}, i.e., the ability of these devices to provide variations of their functionalities not only for different applications, but also in different contexts of the same application. 

%%2: from functional plasticity to a separation of functionalities
%In a client-server architecture, commonly used nowadays, the instances of a pervasive application depend on remote servers. The functionalities provided by each client are self-contained and do not address the needs of other clients. Hence, clients are symmetric with respect to each other; asymmetry is restricted to the relation between clients and servers. 
%In an different model, the functional plasticity of smartphones and other devices could be explored by letting application instances (hereafter referred to as an \textit{application node}) to assume distinct functionalities.

%, i.e., to assume different responsibilities over the application organization.

%In contrast with this rigid symmetry, 
%with the rigid symmetry that characterizes the behavior performed by client nodes in a client-server architecture -- commonly used in today's applications -- 
%or compared to the narrow set of functionalities provided by specialized and extremely resource-constrained devices, 

%In this work, we tackle this challenge with a context-aware assignment of responsibilities.




%This subsection contextualizes the reader about role-orientation in the literature and the novelty of using the role abstraction to build collaborative PA
\subsection{Role-orientation}

%For this, we propose to elevate the concept of a role to a first-class abstraction. 
%A tree with the types of role: symmetric and asymmetric branches with the corresponding types of roles in increasing levels of concreteness 

%What's the difference between tasks and roles?

The concept of roles has been applied in very different areas of information systems, including object-oriented programming, distributed multi-agents, role-based access control, and others. Despite its widespread adoption, there is no common definition for the concept of roles, but actually distinct meanings depending on the context they are employed. Nonetheless, a role is generally associated with rights, responsibilities, and capabilities~\cite{Roles:Survey}.

There is an extensive number of proposals in which the role abstraction is an important aspect or even a key part of the solution. However, not much attention has been paid to the use of role abstraction in the context of pervasive computing. %specially when considering the plethora of heterogeneous devices or situations of high volatility. 
This work aims to fulfill this gap. % by combining role and other organizational abstractions with the state of the art in discentralized self-adaptation and self-organization mechanisms.

%In specific, role-orientation could leverage the potential of context-dependent and opportunistic interactions among pervasive devices.

\subsection{Self-organization and Self-adaptation}

Whereas self-organization has been proposed and used for building decentralized, scalable, and adaptable multi-agent systems, self-adaptation is yet to show its feasibility when subject to volatility and large scale of adaptive entities (e.g., distributed components, agents, etc) and no centralized control. 

The gap between bottom-up self-organization and top-down self-adaptation has been a focus of research~\cite{Kramer:, Zambonelli:, Weyns:}. Such a combination may deem beneficial in the context of pervasive computing: in one hand, the volatility and resource limitations characterizing pervasive devices could cope with the scalability achieved by decentralized self-organization mechanisms. In the other hand, a self-adaptation control loop could mitigate undesirable behaviors that may emerge from self-organization and to further improve the capabilities of the system to adapt to varying situations~\cite{Zambonelli:}. In this paper, we propose the combination of both strategies in the formation and adaptation of opportunistic organizations of pervasive entities.

%Thus, the challenge is twofold: to preserve the well known benefits of existing software engineering methodologies, and inherit/adapt the mechanisms from DAI that can cope with the characteristic of pervasive applications.

%This subsection presents a set of situations in which opportunistic organizations are formed to enable the interaction among pervasive application nodes
%In specific, the interactions should include: 
%collaborative data fetching from remote servers
%collaborative data posting to remote servers
%coordenation of autonomous sensing activities
%a possibility is to describe some apps used by a PhD student, such as: monitoring of public traffic (MCS); air-conditioning temperature consensus; collaborative music streaming;
%\subsection{Motivating Scenario}\label{sec:motivating}
%
%\subsubsection{Mobile Crowd-sensing}
%
%to illustrate the need of opportunistic interactions in pervasive computing, we present a set of applications hosted by the mobile device of an hypothetical user. We start with an example of an opportunistic mobile crowd-sensing (MCS) application installed by the user in its smartphone for public transport monitoring.
%
%MCS consists of a paradigm in which the sensors of user-companioned devices are employed in the measurement of urban and social phenomena~\cite{Guo:2015}. Existing MCS applications range from private and public urban transportation monitoring (e.g., Waze\footnote{https://www.waze.com} and Moovit\footnote{https://www.moovitapp.com}) to atmospheric pressure, noise and air pollution measurement (e.g., PressureNet\footnote{http://www.pressurenet.io}, AirPatrol~\footnote{http://www.crowdfunder.co.uk/crowdsource-air-pollution-in-london}). In transportation monitoring applications, geolocation -- fetched from GPS sensors -- is the main data to be collected, while in other cases, geolocation provides geographic contextualization of the data from other sensors. Finally, some MCS applications require additional wearable sensors (e.g., to monitor the quality of air in AirPatrol) that communicates its results through Bluetooth to a main device (e.g., smartphone).
%
%Many MCS applications target a well defined geographic region and period of time. Accordingly, multiple mobile devices running the application may be sharing the same Wi-Fi network or within range of D2D communication (\textbf{C1}). Also, while moving within a campaign area, devices may suffer from fluctuations of the received Wi-Fi/Bluetooth/GPS signals, as well as exhibit distinct levels of battery and sensors of different quality (\textbf{C2}).
%%Also, user-companioned devices hosting the application are expected to join/leave the campaign area (\textbf{C2}). 
%Whereas some campaigns allow data to be analyzed later, real-time crowd-sensing require a minimum delay between the collection of data and their delivery to backend servers for further processing and publication (\textbf{C3}). Sensors like GPS impose a significant battery drain, and campaigns like transportation monitoring require a frequent activation of this sensor (\textbf{C4}). Finally, with the popularization of crowd-sensing applications, a large number of participants may eventually co-exist in the same campaign area (\textbf{C5}). 
%
%Like other mobile applications, the majority of nowadays MCS are designed following a client-server architecture in which application nodes collect data from their sensors independently from one another. 
%%At most, sensing activities in the client applications are coordinated by the backend server, incurring in additional processing by these servers and exchange of data through Internet. 
%This common approach has the following drawbacks:
%
%\begin{itemize}
%	
%	\item \textbf{Coordination}
%	
%	\begin{enumerate}[label=-]
%		
%		\item Without coordination, application nodes cannot adapt to situations in which multiple devices can provide the same information, thus they tend to consume unnecessary resources as sensing tasks could otherwise be allocated to a subset of the available devices.
%		
%		\item With coordination, nodes could agree on which nodes should perform which tasks at each moment, taking both individual and social contexts into account (e.g., churn of devices, variations in the quality of sensors measurements, intermittent connection with servers, low battery level, etc). 
%		
%	\end{enumerate}
%	
%	\item \textbf{Collaboration}
%	
%	\begin{enumerate}[label=-]
%		
%		\item Without collaboration, each node sends its data to the backend server responsible for filtering and aggregating samples, increasing the the server load and the Internet usage. Additionally, all nodes are assumed to have Internet access to communicate with the server and no collaborative of data is employed (e.g., \cite{Rajagopalan:2006});
%		
%		\item With collaboration, elected application nodes could aggregate data collected from different sensors (e.g., \cite{Wang:2015}) and average those with higher accuracy. Accordingly, less data would to be transmitted to and processed by backend servers. 
%		
%	\end{enumerate}
%\end{itemize}
%
%%\subsubsection{Collaborative Music Streaming}
%
%%Next, we consider the situation in which the user arrives at the office he/she shares with co-workers. To avoid common disagreements on the temperature to be set in the air-conditioning system, co-workers make use of an application that collect their opinion and provides a target temperature consensus. To avoid the costs of a backend server infrastructure, the application has been designed to explore device-to-device interactions. To this end, devices must discover each other and recognize the physical location they are in before joining a consensus group representing the office.
%%
%%In such ad hoc scenario, a single device may become responsible for collecting votes and communicating with the air-conditioning control interface. However, there are no guarantees with respect to the presence of users in the room, which raises the question of how to assign this role as devices enter and leave the room. 
%
%\subsubsection{Local Group Messaging}
%
%the adoption of messaging applications has skyrocket in the last decade, achieving the marks of a billion users for the two most popular applications. One of the most appealing features of these apps are the groups in which a set of users can share messages and, more and more frequently, contents like photos, audio, and videos. In many cases, people working together make use of a group for co-workers. Also common, people sharing the same network infrastructure or within range of ad hoc networking. Today, these applications do not explore device-to-device message exchange, but always rely on backend servers to relay the messages, even when generated and consumed locally. 
%
%One way of exploring opportunistic interactions could be to assign, among the subset of group members within communication range, the role of a \textit{local-relay}. Such role would be responsible for receiving messages from the server and relaying them locally to other group members. The main benefit would be to reduce the client-server communication, as the content should be downloaded once and distributed locally. Also, devices without Internet connection could still communicate through ad hoc network.
%
%\subsubsection{Collaborative Music Streaming}
%
%last but not least, we present a situation in which our user has finished work and join his colleagues in the park nearby for a picknick. Nowadays, popular services allow users to listen to music streamed from the Internet to their mobile devices. It has also become popular the use of portable speakers equipped with Bluetooth, so that users can enjoy a more powerful audio than provided by their smartphones and tablets. This combination of pervasive devices is specially appealing for external gatherings in parks and other recreational areas. Nowadays, many of these spaces provide public Wi-Fi and/or good cellular network service coverage (\textbf{C1}). However, both the streaming of music from the Internet and the communication through Bluetooth are expensive features in terms of battery and networking, as large volumes of data must be transferred (\textbf{C3}) with minimum delay (\textbf{C4}). Hence, to mitigate the battery drain and, if no Wi-Fi is available, to alleviate the amount of data each device has to download from its data plan, as many as possible devices should share this responsibility. In other words:
%
%\begin{itemize}
%	
%	\item the playlist interface must by synchronized among smartphones for collaborative modifications;
%	
%	\item a \textit{streaming-role} must be dynamically assigned to one device at a time without interrupting the music play;
%	
%	\item conditions such as the battery level, as well as the Internet and Bluetooth throughputs must be taken into account; and
%	
%	\item as many capable devices as possible must play the streaming-role at different times.
%	
%\end{itemize}
%
%
%Among the assumptions and characteristics of the scenario tackled by work, the following are considered as key features:
%
%\begin{enumerate}[label=C\arabic*]
%	
%	\item \textbf{Connectivity:} devices are expected to communicate with each other, while in the same area, through infrastructure Wi-Fi or D2D communication.
%	
%	\item \textbf{Volatility and Heterogeneity:} application nodes are expected to enter/leave a given zone/network without notice; also, the physical and computational environment in which application nodes operate is subject to changes that may affect capabilities required by the application.
%	
%	\item \textbf{Intensity:} application nodes may be required to perform resource-intensive tasks or to exchange large volumes of data, or a mix of both.
%	
%	\item \textbf{Delay-sensitive:} among the application features, some may be sensible to latency.
%	
%	\item \textbf{Scale:} up to a large number of application nodes are expected to co-exist in a given zone/network and potentially interact.
%
%	
%\end{enumerate}
%
%Next, we present examples of applications with use cases that motivate our approach.
