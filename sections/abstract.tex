The large scale adoption of pervasive and mobile computing constitutes a scenario in which data is been produced and consumed at an unprecedented rate by mobile applications at the network edge. In many cases, application nodes must interact for purposes like: 1) exchanging application data; 2) coordinating autonomous behaviors; 3) collaborating for efficiency. 
%offloading computation?
Such interactions are, in most of the cases, proxied by servers deployed in the cloud. While cloud computing has been a successful model for distributed computing in general, problems such as latency, cost and intermittent connectivity still need to be harassed. With the advent of 5G, disruptive new features are expected, including location-awareness and device-to-device (D2D) communication. The combination of this technology with the existing wireless technologies allows mobile application engineers to explore the potential of a more distributed architecture, one that allows the heterogeneous and volatile nodes of pervasive applications 
%to rely on each other to solve local problems locally. 
to form adaptive organizations in which the functionality they provide is defined by their individual and social contexts. Targeting this scenario, we introduce the concept of a self-organizing interaction spaces (SOIS), a framework for the engineering of pervasive applications. As a contribution of this paper, we specify two abstractions for modeling and programing this kind of applications based on an organization mindset, as well as the mechanisms for adapting the dynamic organization structure. 
The concepts and abstractions were demonstrated with case examples and the performance of its methods evaluated with a simulation of a mobile crowd-sensing application. Results corroborate the feasibility and the benefits of the approach. 

%It also provides insight for novel research contributions.


%Through edge spaces, application nodes are allowed to interact locally using different communication and coordination styles. 
%Additionally, given the volatility of mobile devices, asymmetric responsibilities are dynamically assigned to the best candidates according to their context and capability using a distributed allocation method. 


%Targeting this scenario, the 5th generation mobile networks and wireless systems (5G) shall bring disruptive developments to the mobile telecommunication services, including context-awareness and device-to-device (D2D) communication. This combination increases the potential for new types of applications, but also poses challenges to their engineering. %, as low latency, cost, volatility and heterogeneity of devices still need to be harassed.

%While cloud computing has been and a successful model for distributed computing in general, problems such as latency, cost and intermittent connectivity still need to be harassed. In this paper, we introduce the concept of \textit{Ad Hoc Edge Spaces} as part of a framework for the engineering of mobile applications that cannot depend solely on remote servers to satisfy their communication, processing and coordination requirements. Through edge spaces, application nodes can interact locally using different communication and coordination styles. Additionally, given the volatility and heterogeneity of mobile devices, asymmetric responsibilities are dynamically assigned to the best candidates within the edge space according to their context and capability. The feasibility of this approach was evaluated with the implementation of a crowdsensing application for Android platform.

%The purpose is to avoid bottlenecks, costs and the latency of cloud servers.

%This scenario has specific characteristics, such as a high heterogeneity and volatility, intermittent connectivity, and resource limitations. To further explore its potential, pervasive and mobile applications must be able to overcome challenges such as the churn of components and fluctuations of physical and computational resources. 

%Moreover, considering the nature of personal devices, their heterogeneity, and scale, applications must make proper use of resources with an efficient and coherent distribution of its activities among participant devices.
%As more and more devices join the network or because of low latency requirements, a server-base architecture in which servers in the cloud handle most of the computation may not deliver the expected results. Conversely, 
%Many devices are now equipped with more powerful resources. This enables more computation to be performed at the edge of the network infrastructure. Some applications already explore machine-to-machine (M2M) communication to allow distributed components to interact and coordinate their behavior through local and mobile ad hoc networks (MANETs). Despite the potential benefits, an horizontal architectural must also deal with problems related to the churn of devices, consistency, robustness, availability, resource limitations, fairness, dynamic policies, security, and others.  


%In the literature, data-centric models are the state-of-art for distributed components coordination, as they enable both space and time decoupling. However, their high abstraction leaves many aspects open. In this work, we propose a framework for the development of distributed and self-adaptive applications willing to explore the potential of nowadays pervasive computing. Our model inherits well-known organizational abstractions, namely groups and roles, which allow a flexible modularization of the system and a dynamic assignment of activities among heterogeneous hosting devices in scenarios of high volatility. The abstractions and features provided by the framework target the application itself (managed system) and also the layer responsible for its adaptation (managing system) -- which should preserve the normal behavior and avoid violations of the application requirement in the advent of context changes.

%We have showed the feasibility of this approach with a real-case application for public transport monitoring and compared its features with other models and mechanisms for self-adaptation.

%In addition, we detail the self-organization mechanisms and their incorporation by a middleware.  Finally, we have showed the use of our model in different application scenarios.