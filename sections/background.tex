\section{Background}\label{sec:background}

%TODO architectural fluidity
%TODO characterize the music stream example w.r.t. C[1-5] and anticipate the benefits of collaboration

%1: The network model
\subsection{Wireless Networking}

Not only computers have pervaded the space inhabit by humans, wireless networking is quickly fulfilling any remaining gaps in connectivity through which pervasive devices can remain connected, including while they are carried from one region to the other. Whereas the connectivity between devices was mostly restricted to the areas with a private or public Wi-Fi coverage, recent developments in device-to-device (D2D) technology expand the situations in which devices can communicate and interoperate. For example, in addition to the already consolidated Wi-Fi direct and Bluetooth Low Energy (BLE) technologies, the fifth generation mobile networks (5G) standards include the support for D2D communication~\cite{Tehrani:2014}. Thus, a new and exciting landscape for pervasive and mobile computing is taking form. Figure~\ref{fig:network_model} illustrates this scenario.

\begin{figure*}[t!]
	\centering
	\includegraphics[width=0.8\textwidth]{figures/network_model}
	\caption{Application nodes communicating through Wi-Fi or D2D technologies}
	\label{fig:network_model}
\end{figure*}

%2: Nowadays Pervasive and mobile computing
\subsection{Pervasive Applications}~\label{sec:characterization}

%TODO: fix this paragraph
The market of applications crafted for pervasive and specially mobile devices continues to increase as more people have access to this technology. In specific, applications can be hosted by pervasive devices with more less degree of mobility and computational power, such as smartphones, tables, gadges, and miniaturized computer platforms such as Raspberry PI, which now supports the Android platform for Internet of Things~\footnote{https://developer.android.com/things/hardware/raspberrypi.html}. Also, new types of specialized devices like Bluetooth beacons compose the rich ecosystem in which pervasive applications can exist.

While mobile devices represent the preeminent type of pervasive computing nowadays and the majority of applications target mobile platforms, we adopted \textit{pervasive} as a broader qualitative (instead of mobile) to avoid the exclusion of pervasive devices that exhibit limit mobility. 

