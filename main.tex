%% bare_conf.tex
%% V1.4b
%% 2015/08/26
%% by Michael Shell
%% See:
%% http://www.michaelshell.org/
%% for current contact information.
%%
%% This is a skeleton file demonstrating the use of IEEEtran.cls
%% (requires IEEEtran.cls version 1.8b or later) with an IEEE
%% conference paper.
%%
%% Support sites:
%% http://www.michaelshell.org/tex/ieeetran/
%% http://www.ctan.org/pkg/ieeetran
%% and
%% http://www.ieee.org/

%%*************************************************************************
%% Legal Notice:
%% This code is offered as-is without any warranty either expressed or
%% implied; without even the implied warranty of MERCHANTABILITY or
%% FITNESS FOR A PARTICULAR PURPOSE! 
%% User assumes all risk.
%% In no event shall the IEEE or any contributor to this code be liable for
%% any damages or losses, including, but not limited to, incidental,
%% consequential, or any other damages, resulting from the use or misuse
%% of any information contained here.
%%
%% All comments are the opinions of their respective authors and are not
%% necessarily endorsed by the IEEE.
%%
%% This work is distributed under the LaTeX Project Public License (LPPL)
%% ( http://www.latex-project.org/ ) version 1.3, and may be freely used,
%% distributed and modified. A copy of the LPPL, version 1.3, is included
%% in the base LaTeX documentation of all distributions of LaTeX released
%% 2003/12/01 or later.
%% Retain all contribution notices and credits.
%% ** Modified files should be clearly indicated as such, including  **
%% ** renaming them and changing author support contact information. **
%%*************************************************************************


% *** Authors should verify (and, if needed, correct) their LaTeX system  ***
% *** with the testflow diagnostic prior to trusting their LaTeX platform ***
% *** with production work. The IEEE's font choices and paper sizes can   ***
% *** trigger bugs that do not appear when using other class files.       ***                          ***
% The testflow support page is at:
% http://www.michaelshell.org/tex/testflow/



%\documentclass[10pt, conference, compsocconf]{IEEEtran}
\documentclass[twocolumn,twoside]{IEEEtran}
% Some Computer Society conferences also require the compsoc mode option,
% but others use the standard conference format.
%
% If IEEEtran.cls has not been installed into the LaTeX system files,
% manually specify the path to it like:
% \documentclass[conference]{../sty/IEEEtran}



% Some very useful LaTeX packages include:
% (uncomment the ones you want to load)


% *** MISC UTILITY PACKAGES ***
%
%\usepackage{ifpdf}
% Heiko Oberdiek's ifpdf.sty is very useful if you need conditional
% compilation based on whether the output is pdf or dvi.
% usage:
% \ifpdf
%   % pdf code
% \else
%   % dvi code
% \fi
% The latest version of ifpdf.sty can be obtained from:
% http://www.ctan.org/pkg/ifpdf
% Also, note that IEEEtran.cls V1.7 and later provides a builtin
% \ifCLASSINFOpdf conditional that works the same way.
% When switching from latex to pdflatex and vice-versa, the compiler may
% have to be run twice to clear warning/error messages.


\usepackage{subcaption}
\captionsetup[subfigure]{width=.9\textwidth}
\usepackage{enumitem}
\usepackage{color}
\definecolor{lightgrey}{gray}{0.98}
\definecolor{blue}{rgb}{0, 0, 0.25}

% *** CITATION PACKAGES ***
%
%\usepackage{cite}
% cite.sty was written by Donald Arseneau
% V1.6 and later of IEEEtran pre-defines the format of the cite.sty package
% \cite{} output to follow that of the IEEE. Loading the cite package will
% result in citation numbers being automatically sorted and properly
% "compressed/ranged". e.g., [1], [9], [2], [7], [5], [6] without using
% cite.sty will become [1], [2], [5]--[7], [9] using cite.sty. cite.sty's
% \cite will automatically add leading space, if needed. Use cite.sty's
% noadjust option (cite.sty V3.8 and later) if you want to turn this off
% such as if a citation ever needs to be enclosed in parenthesis.
% cite.sty is already installed on most LaTeX systems. Be sure and use
% version 5.0 (2009-03-20) and later if using hyperref.sty.
% The latest version can be obtained at:
% http://www.ctan.org/pkg/cite
% The documentation is contained in the cite.sty file itself.






% *** GRAPHICS RELATED PACKAGES ***
%
\ifCLASSINFOpdf
  \usepackage[pdftex]{graphicx}
  % declare the path(s) where your graphic files are
  % \graphicspath{{../pdf/}{../jpeg/}}
  % and their extensions so you won't have to specify these with
  % every instance of \includegraphics
  % \DeclareGraphicsExtensions{.pdf,.jpeg,.png}
  %%Draw graphics
\usepackage{pgfplots}
\usepackage{pgfplotstable}
\pgfplotsset{compat=newest}
\pgfplotscreateplotcyclelist{my chart colors}{%
blue, solid, every mark/.append style={solid, fill=blue}, mark=*\\%
red, solid, every mark/.append style={solid, fill=red}, mark=square*\\%
brown, solid, every mark/.append style={solid, fill=brown}, mark=triangle*\\%
cyan, solid, every mark/.append style={solid, fill=gray}, mark=diamond*\\%
black, dashdotted, every mark/.append style={solid, fill=red}, mark=halfdiamond*\\%
olive, dashdotted, every mark/.append style={solid, fill=green}, mark=halfcircle*\\%
orange, dashdotted, every mark/.append style={solid, fill=gray}, mark=*\\%
purple, dashdotted, every mark/.append style={solid, fill=blue}, mark=halfsquare left*\\%
teal, densely dashed, every mark/.append style={solid, fill=green}, mark=oplus*\\%
violet, densely dashed, every mark/.append style={solid, fill=gray}, mark=diamond*\\%
magenta, densely dashed, every mark/.append style={solid, fill=blue}, mark=triangle*\\%
lime, densely dashed, every mark/.append style={solid, fill=red}, mark=square*\\%
}

\newcommand{\errorband}[5][]{ % x column, y column, error column, optional argument for setting style of the area plot
\pgfplotstableread[col sep=comma, skip first n=2]{#2}\datatable
    % Lower bound (invisible plot)
    \addplot [draw=none, stack plots=y, forget plot] table [
        x={#3},
        y expr=\thisrow{#4}-\thisrow{#5}
    ] {\datatable};

    % Stack twice the error, draw as area plot
    \addplot [draw=none, fill=gray!40, stack plots=y, area legend, #1] table [
        x={#3},
        y expr=2*\thisrow{#5}
    ] {\datatable} \closedcycle;

    % Reset stack using invisible plot
    \addplot [forget plot, stack plots=y,draw=none] table [x={#3}, y expr=-(\thisrow{#4}+\thisrow{#5})] {\datatable};
}
  %\usepackage{flushend}
\else
  % or other class option (dvipsone, dvipdf, if not using dvips). graphicx
  % will default to the driver specified in the system graphics.cfg if no
  % driver is specified.
  % \usepackage[dvips]{graphicx}
  % declare the path(s) where your graphic files are
  % \graphicspath{{../eps/}}
  % and their extensions so you won't have to specify these with
  % every instance of \includegraphics
  % \DeclareGraphicsExtensions{.eps}
\fi
% graphicx was written by David Carlisle and Sebastian Rahtz. It is
% required if you want graphics, photos, etc. graphicx.sty is already
% installed on most LaTeX systems. The latest version and documentation
% can be obtained at: 
% http://www.ctan.org/pkg/graphicx
% Another good source of documentation is "Using Imported Graphics in
% LaTeX2e" by Keith Reckdahl which can be found at:
% http://www.ctan.org/pkg/epslatex
%
% latex, and pdflatex in dvi mode, support graphics in encapsulated
% postscript (.eps) format. pdflatex in pdf mode supports graphics
% in .pdf, .jpeg, .png and .mps (metapost) formats. Users should ensure
% that all non-photo figures use a vector format (.eps, .pdf, .mps) and
% not a bitmapped formats (.jpeg, .png). The IEEE frowns on bitmapped formats
% which can result in "jaggedy"/blurry rendering of lines and letters as
% well as large increases in file sizes.
%
% You can find documentation about the pdfTeX application at:
% http://www.tug.org/applications/pdftex





% *** MATH PACKAGES ***
%
\usepackage{amsmath}
% A popular package from the American Mathematical Society that provides
% many useful and powerful commands for dealing with mathematics.
%
% Note that the amsmath package sets \interdisplaylinepenalty to 10000
% thus preventing page breaks from occurring within multiline equations. Use:
%\interdisplaylinepenalty=2500
% after loading amsmath to restore such page breaks as IEEEtran.cls normally
% does. amsmath.sty is already installed on most LaTeX systems. The latest
% version and documentation can be obtained at:
% http://www.ctan.org/pkg/amsmath


\newtheorem{definition}{Definition} 


% *** SPECIALIZED LIST PACKAGES ***
%
%\usepackage{algorithmic}
% algorithmic.sty was written by Peter Williams and Rogerio Brito.
% This package provides an algorithmic environment fo describing algorithms.
% You can use the algorithmic environment in-text or within a figure
% environment to provide for a floating algorithm. Do NOT use the algorithm
% floating environment provided by algorithm.sty (by the same authors) or
% algorithm2e.sty (by Christophe Fiorio) as the IEEE does not use dedicated
% algorithm float types and packages that provide these will not provide
% correct IEEE style captions. The latest version and documentation of
% algorithmic.sty can be obtained at:
% http://www.ctan.org/pkg/algorithms
% Also of interest may be the (relatively newer and more customizable)
% algorithmicx.sty package by Szasz Janos:
% http://www.ctan.org/pkg/algorithmicx
%\usepackage{flushend}
\usepackage{listings}
%\usepackage{filecontents}


% *** ALIGNMENT PACKAGES ***
%
\usepackage{array}
\usepackage{tabularx} % Danilo added to adjust the width of the tables
% Frank Mittelbach's and David Carlisle's array.sty patches and improves
% the standard LaTeX2e array and tabular environments to provide better
% appearance and additional user controls. As the default LaTeX2e table
% generation code is lacking to the point of almost being broken with
% respect to the quality of the end results, all users are strongly
% advised to use an enhanced (at the very least that provided by array.sty)
% set of table tools. array.sty is already installed on most systems. The
% latest version and documentation can be obtained at:
% http://www.ctan.org/pkg/array


% IEEEtran contains the IEEEeqnarray family of commands that can be used to
% generate multiline equations as well as matrices, tables, etc., of high
% quality.




% *** SUBFIGURE PACKAGES ***
%\ifCLASSOPTIONcompsoc
%  \usepackage[caption=false,font=normalsize,labelfont=sf,textfont=sf]{subfig}
%\else
%  \usepackage[caption=false,font=footnotesize]{subfig}
%\fi
% subfig.sty, written by Steven Douglas Cochran, is the modern replacement
% for subfigure.sty, the latter of which is no longer maintained and is
% incompatible with some LaTeX packages including fixltx2e. However,
% subfig.sty requires and automatically loads Axel Sommerfeldt's caption.sty
% which will override IEEEtran.cls' handling of captions and this will result
% in non-IEEE style figure/table captions. To prevent this problem, be sure
% and invoke subfig.sty's "caption=false" package option (available since
% subfig.sty version 1.3, 2005/06/28) as this is will preserve IEEEtran.cls
% handling of captions.
% Note that the Computer Society format requires a larger sans serif font
% than the serif footnote size font used in traditional IEEE formatting
% and thus the need to invoke different subfig.sty package options depending
% on whether compsoc mode has been enabled.
%
% The latest version and documentation of subfig.sty can be obtained at:
% http://www.ctan.org/pkg/subfig




% *** FLOAT PACKAGES ***
%
%\usepackage{fixltx2e}
% fixltx2e, the successor to the earlier fix2col.sty, was written by
% Frank Mittelbach and David Carlisle. This package corrects a few problems
% in the LaTeX2e kernel, the most notable of which is that in current
% LaTeX2e releases, the ordering of single and double column floats is not
% guaranteed to be preserved. Thus, an unpatched LaTeX2e can allow a
% single column figure to be placed prior to an earlier double column
% figure.
% Be aware that LaTeX2e kernels dated 2015 and later have fixltx2e.sty's
% corrections already built into the system in which case a warning will
% be issued if an attempt is made to load fixltx2e.sty as it is no longer
% needed.
% The latest version and documentation can be found at:
% http://www.ctan.org/pkg/fixltx2e


%\usepackage{stfloats}
% stfloats.sty was written by Sigitas Tolusis. This package gives LaTeX2e
% the ability to do double column floats at the bottom of the page as well
% as the top. (e.g., "\begin{figure*}[!b]" is not normally possible in
% LaTeX2e). It also provides a command:
%\fnbelowfloat
% to enable the placement of footnotes below bottom floats (the standard
% LaTeX2e kernel puts them above bottom floats). This is an invasive package
% which rewrites many portions of the LaTeX2e float routines. It may not work
% with other packages that modify the LaTeX2e float routines. The latest
% version and documentation can be obtained at:
% http://www.ctan.org/pkg/stfloats
% Do not use the stfloats baselinefloat ability as the IEEE does not allow
% \baselineskip to stretch. Authors submitting work to the IEEE should note
% that the IEEE rarely uses double column equations and that authors should try
% to avoid such use. Do not be tempted to use the cuted.sty or midfloat.sty
% packages (also by Sigitas Tolusis) as the IEEE does not format its papers in
% such ways.
% Do not attempt to use stfloats with fixltx2e as they are incompatible.
% Instead, use Morten Hogholm'a dblfloatfix which combines the features
% of both fixltx2e and stfloats:
%
% \usepackage{dblfloatfix}
% The latest version can be found at:
% http://www.ctan.org/pkg/dblfloatfix




% *** PDF, URL AND HYPERLINK PACKAGES ***
%
\usepackage{url}
% url.sty was written by Donald Arseneau. It provides better support for
% handling and breaking URLs. url.sty is already installed on most LaTeX
% systems. The latest version and documentation can be obtained at:
% http://www.ctan.org/pkg/url
% Basically, \url{my_url_here}.




% *** Do not adjust lengths that control margins, column widths, etc. ***
% *** Do not use packages that alter fonts (such as pslatex).         ***
% There should be no need to do such things with IEEEtran.cls V1.6 and later.
% (Unless specifically asked to do so by the journal or conference you plan
% to submit to, of course. )


% correct bad hyphenation here
%\hyphenation{op-tical net-works semi-conduc-tor}


\begin{document}
%
% paper title
% Titles are generally capitalized except for words such as a, an, and, as,
% at, but, by, for, in, nor, of, on, or, the, to and up, which are usually
% not capitalized unless they are the first or last word of the title.
% Linebreaks \\ can be used within to get better formatting as desired.
% Do not put math or special symbols in the title.

\title{Engineering Pervasive Applications with Self-organizing Interaction Spaces}


% author names and affiliations
% use a multiple column layout for up to three different
% affiliations
\author{\IEEEauthorblockN{Luciano Baresi}
\IEEEauthorblockA{Politecnico di Milano\\
	Dipartimento di Elettronica, Informazione e Bioingegneria\\
	Piazza L. da Vinci, 32 -- 20133 Milano, Italy\\
	Email: luciano.baresi@polimi.it}
\and
\IEEEauthorblockN{Danilo Filgueira Mendonça}
\IEEEauthorblockA{Politecnico di Milano\\
	Dipartimento di Elettronica, Informazione e Bioingegneria\\
	Piazza L. da Vinci, 32 -- 20133 Milano, Italy\\
	Email: luciano.baresi@polimi.it}
\and
\IEEEauthorblockN{James Kirk\\ and Montgomery Scott}
\IEEEauthorblockA{Starfleet Academy\\
	San Francisco, California 96678--2391\\
	Telephone: (800) 555--1212\\
	Fax: (888) 555--1212}}






% conference papers do not typically use \thanks and this command
% is locked out in conference mode. If really needed, such as for
% the acknowledgment of grants, issue a \IEEEoverridecommandlockouts
% after \documentclass

% for over three affiliations, or if they all won't fit within the width
% of the page, use this alternative format:
% 
%\author{\IEEEauthorblockN{Michael Shell\IEEEauthorrefmark{1},
%Homer Simpson\IEEEauthorrefmark{2},
%James Kirk\IEEEauthorrefmark{3}, 
%Montgomery Scott\IEEEauthorrefmark{3} and
%Eldon Tyrell\IEEEauthorrefmark{4}}
%\IEEEauthorblockA{\IEEEauthorrefmark{1}School of Electrical and Computer Engineering\\
%Georgia Institute of Technology,
%Atlanta, Georgia 30332--0250\\ Email: see http://www.michaelshell.org/contact.html}
%\IEEEauthorblockA{\IEEEauthorrefmark{2}Twentieth Century Fox, Springfield, USA\\
%Email: homer@thesimpsons.com}
%\IEEEauthorblockA{\IEEEauthorrefmark{3}Starfleet Academy, San Francisco, California 96678-2391\\
%Telephone: (800) 555--1212, Fax: (888) 555--1212}
%\IEEEauthorblockA{\IEEEauthorrefmark{4}Tyrell Inc., 123 Replicant Street, Los Angeles, California 90210--4321}}




% use for special paper notices
%\IEEEspecialpapernotice{(Invited Paper)}

% make the title area
\maketitle

% As a general rule, do not put math, special symbols or citations
% in the abstract
\begin{abstract}
The large scale adoption of pervasive and mobile computing constitutes a scenario in which data is been produced and consumed at an unprecedented rate by mobile applications at the network edge. In many cases, application nodes must interact for purposes like: 1) exchanging application data; 2) coordinating autonomous behaviors; 3) collaborating for efficiency. 
%offloading computation?
Such interactions are, in most of the cases, proxied by servers deployed in the cloud. While cloud computing has been a successful model for distributed computing in general, problems such as latency, cost and intermittent connectivity still need to be harassed. With the advent of 5G, disruptive new features are expected, including location-awareness and device-to-device (D2D) communication. The combination of this technology with the existing wireless technologies allows mobile application engineers to explore the potential of a more distributed architecture, one that allows the heterogeneous and volatile nodes of pervasive applications 
%to rely on each other to solve local problems locally. 
to form adaptive organizations in which the functionality they provide is defined by their individual and social contexts. Targeting this scenario, we introduce the concept of a self-organizing interaction spaces (SOIS), a framework for the engineering of pervasive applications. As a contribution of this paper, we specify two abstractions for modeling and programing the application based on an organization mindset, as well as the mechanisms for adapting the dynamic organization structure. 
The concepts and abstractions were demonstrated with case examples and the performance of its methods evaluated with a simulation of a mobile crowd-sensing application. Results corroborate the feasibility and the benefits of the approach. 

%It also provides insight for novel research contributions.


%Through edge spaces, application nodes are allowed to interact locally using different communication and coordination styles. 
%Additionally, given the volatility of mobile devices, asymmetric responsibilities are dynamically assigned to the best candidates according to their context and capability using a distributed allocation method. 


%Targeting this scenario, the 5th generation mobile networks and wireless systems (5G) shall bring disruptive developments to the mobile telecommunication services, including context-awareness and device-to-device (D2D) communication. This combination increases the potential for new types of applications, but also poses challenges to their engineering. %, as low latency, cost, volatility and heterogeneity of devices still need to be harassed.

%While cloud computing has been and a successful model for distributed computing in general, problems such as latency, cost and intermittent connectivity still need to be harassed. In this paper, we introduce the concept of \textit{Ad Hoc Edge Spaces} as part of a framework for the engineering of mobile applications that cannot depend solely on remote servers to satisfy their communication, processing and coordination requirements. Through edge spaces, application nodes can interact locally using different communication and coordination styles. Additionally, given the volatility and heterogeneity of mobile devices, asymmetric responsibilities are dynamically assigned to the best candidates within the edge space according to their context and capability. The feasibility of this approach was evaluated with the implementation of a crowdsensing application for Android platform.

%The purpose is to avoid bottlenecks, costs and the latency of cloud servers.

%This scenario has specific characteristics, such as a high heterogeneity and volatility, intermittent connectivity, and resource limitations. To further explore its potential, pervasive and mobile applications must be able to overcome challenges such as the churn of components and fluctuations of physical and computational resources. 

%Moreover, considering the nature of personal devices, their heterogeneity, and scale, applications must make proper use of resources with an efficient and coherent distribution of its activities among participant devices.
%As more and more devices join the network or because of low latency requirements, a server-base architecture in which servers in the cloud handle most of the computation may not deliver the expected results. Conversely, 
%Many devices are now equipped with more powerful resources. This enables more computation to be performed at the edge of the network infrastructure. Some applications already explore machine-to-machine (M2M) communication to allow distributed components to interact and coordinate their behavior through local and mobile ad hoc networks (MANETs). Despite the potential benefits, an horizontal architectural must also deal with problems related to the churn of devices, consistency, robustness, availability, resource limitations, fairness, dynamic policies, security, and others.  


%In the literature, data-centric models are the state-of-art for distributed components coordination, as they enable both space and time decoupling. However, their high abstraction leaves many aspects open. In this work, we propose a framework for the development of distributed and self-adaptive applications willing to explore the potential of nowadays pervasive computing. Our model inherits well-known organizational abstractions, namely groups and roles, which allow a flexible modularization of the system and a dynamic assignment of activities among heterogeneous hosting devices in scenarios of high volatility. The abstractions and features provided by the framework target the application itself (managed system) and also the layer responsible for its adaptation (managing system) -- which should preserve the normal behavior and avoid violations of the application requirement in the advent of context changes.

%We have showed the feasibility of this approach with a real-case application for public transport monitoring and compared its features with other models and mechanisms for self-adaptation.

%In addition, we detail the self-organization mechanisms and their incorporation by a middleware.  Finally, we have showed the use of our model in different application scenarios.
\end{abstract}

% no keywords
%!TEX root = main.tex
% -*- root: main.tex -*-
\section{Introduction}
\label{sec:intro}

% % % % % % %

%Pervasive Computing + Opportunistic Interactions through Wireless Networks

%Asymmetry + role assignment

%Decentralized self-adaptation in pervasive computing

% % % % % % %

%1: The scenario of pervasive and mobile computing today
In the last decade, the world has witnessed the massive diffusion of pervasive and mobile devices. Some of them are equipped with more powerful CPUs, memory, and storage (e.g. modern smartphones and tablets), others are designed for specific functions and have limited resources (e.g., sensors and gadgets). In both cases, applications hosted by these devices are increasing in number and complexity. 
%
%\begin{figure}[t!]
%	\centering
%	\includegraphics[width=0.7\linewidth]{figures/pervasive_devices}
%	\caption{Different classes of pervasive devices; a) a laptop; b) a Raspberry PI; c) a smartphone; d) a tablet; e) a gadget (watch); f) a Bluetooth beacon}
%	\label{fig:pervasive_devices}
%\end{figure}

%2: Today's mainstream architectural model
Today, pervasive applications are often organized around: 1) a client application running on users' devices; 2) a backend application hosted on cloud servers. Accordingly, data produced by these applications at the edge of the network are sent to the cloud, processed there and sent back to the edge. Even if this model is appropriate in many situations, there are existing and emerging use cases that cannot afford the latency introduced by communicating with distant servers, cannot assume a stable and reliable connection with them, or users are not willing to accept the costs of remote communication, specially if part of the data can be processed and consumed locally.

%3: Why not the existing proposals
The literature covering device-to-device interaction in pervasive computing is vast and range from multi-agent platforms to peer-to-peer protocols. This body of knowledge includes specific proposals for distributed coordination and collaboration. However, existing proposals targeting local and opportunistic interactions either tackle a narrow set of goals (e.g., content lookup and sharing), or they require a centralized platform to work (JADE platform for multi-agents). 
% in scenarios of mobility and constrained resources.

%4: Examples of P2P/D2D based solutions
In its simplest form, a device-to-device interaction targets the exchange of application data (e.g., content generated by users). However, some features are characterized by their autonomy 
%are targeting more complex and sophisticated goals 
(e.g., to autonomously decide when to sense the physical environment in specific areas of a city), their intensity (e.g., battery-consuming and data-intensive tasks), or by their sensitivity to delay (e.g., players in a mobile multiplayer game). In these and other cases, devices may become responsible for distinguished roles depending on their social and individual contexts. In this work we demonstrate how simple organizational abstractions, namely, roles and groups, can be employed to engineer pervasive applications that explore the serendipitous interactions among devices. We additionally show how classic self-organization principles (bottom-up) can be combined with top-down architecture-based self-adaptation to cope with the specific constraints of pervasive computing.

%In this paper, we tackle both conceptual and concrete aspects that can enable the adoption of opportunistic interactions among pervasive devices for a broader range of applications.

%These features may require the coordination among instances of an application hosted by different devices (hereafter referred to as \textit{application nodes}). 
%4: The paper's proposal in terms of 
%We argue that existing and novel features from pervasive applications can be achieved more efficiently by exploring serendipitous interactions among devices. % generated by users' mobility and opportunistic wireless communications.

%letting the application hosted by different devices (hereafter referred to as application \textit{nodes}) to 

%%In specific, we target the serendipitous situations in which the devices hosting these applications become visible to each other and are able to communicate. 
%%3: Functional plasticity + relations between nodes 
%In specific, this work focuses on the relations that can be formed (and dissolved) among the devices composing a pervasive ecosystem. 
%%Given the functional plasticity of today's 
%Since pervasive and mobile devices are equipped with more powerful computational resources and several types of sensors, 
%%we argue that 
%%sense and actuate the environment, or even to perform computation-intensive tasks,
%new types of relations -- others than just the one between clients and backend servers -- can be opportunistically created and dissolved among devices at the edge of the network, i.e., without intermediation of remote servers. These relations must cope with the heterogeneity and volatility that characterize the pervasive ecosystem, including the mobility of nodes and the fluctuations of their connectivity and resources. 


%%4: Contributions
%\textbf{Contribution of the work:} 
%%4.1: SOIS concepts
%as contribution, the paper introduces the concept of a \textit{self-organizing interaction spaces} (SOIS) 
%%as part of a framework for the 
%for engineering pervasive applications. 
%%that require or benefit from the interaction between its nodes. 
%%4.1a: The SOIS rationale
%Firstly, we discuss the rationale of how some existing and emerging application features, empowered by today's technologies for wireless communication, 
%%and the functional plasticity of its hosting devices, 
%can be met by letting application nodes assume distinct functional roles. 
%%4.1b: The self-organization mechanisms
%Secondly, we provide the abstractions for modeling and programming these applications. Finally, we propose the self-organization mechanisms that maintain and evolve the structure of the application based on the individual and social contexts of devices.
%%application nodes capabilities and state in the formation and adaptation of the ecosystem organization. 
%%Finally, we present the requirements for a middleware to support the programming and operation of pervasive applications based on SOIS.


%The feasibility of the proposal is demonstrated on examples of pervasive applications and the complexity of the self-organization mechanisms evaluated by means of asymptotic analysis and a simulation of a mobile crowd-sensing application.
%%was evaluated with a simulation of mobile crowdsensing application. %The results showed the feasibility of the approach and, 
%In comparison with a pure client-server model, the SOIS based solution reduced the overall battery consumption in X\% and the volume of data sent to backend servers in Y\%, while preserving low-latency.

%5: How the paper is structured
The paper is structured as follows: Section~\ref{sec:background} introduces relevant background information; Section~\ref{sec:overview} presents the proposed solution and uses an example application to motivate it and illustrate its characteristics. Section~\ref{sec:edge_spaces} presents the modeling and programming abstractions, whereas the self-organization mechanisms are presented in Section~\ref{sec:self_organization}. Section~\ref{sec:evaluation} reports on the evaluation of our framework with a mobile crowd-sensing application. Section~\ref{sec:related_work} surveys related work and Section~\ref{sec:conclusion} concludes the paper.

\section{Background}\label{sec:background}

%TODO architectural fluidity
%TODO characterize the music stream example w.r.t. C[1-5] and anticipate the benefits of collaboration

%1: The network model
\subsection{Wireless Networking}

Not only computers have pervaded the space inhabit by humans, wireless networking is quickly fulfilling the gaps in connectivity through which pervasive devices can communicate, including while in transit (Figure~\ref{fig:network_model}). Whereas the connectivity between devices was mostly restricted to the areas with a private or public Wi-Fi coverage, recent developments in device-to-device (D2D) technology expand the situations in which devices can communicate and interact. For example, in addition to the already consolidated Wi-Fi direct and Bluetooth Low Energy (BLE) technologies, the fifth generation mobile networks (5G) standards include the support for D2D communication~\cite{Tehrani:2014}. Thus, a new and exciting landscape for pervasive and mobile computing is taking form.

\begin{figure*}[t!]
	\centering
	\includegraphics[width=0.8\textwidth]{figures/network_model}
	\caption{Application nodes communicating through Wi-Fi or D2D technologies}
	\label{fig:network_model}
\end{figure*}

%2: Nowadays Pervasive and mobile computing
\subsection{Pervasive Applications}~\label{sec:characterization}

The market of applications crafted for pervasive and specially mobile devices continues to increase as more people have access to this technology. In specific, applications can be hosted by pervasive devices with higher or less degree of mobility and computational power, such as smartphones, tables, gadges, and miniaturized computer platforms such as Raspberry PI (which now supports the Android platform for Internet of Things~\footnote{https://developer.android.com/things/hardware/raspberrypi.html}). Moreover, new types of specialized devices like Bluetooth beacons compose the rich ecosystem in which pervasive applications can run. Despite mobile computing and devices to be the preeminent type of pervasive computing, as well as the target platform for the majority of applications, we adopted \textit{pervasive} as a broader qualitative term (instead of mobile) to avoid the discrimination of pervasive devices that exhibit limit mobility, but are potential candidates for composing the diverse scenario target by this proposal. 


%Why do we need this section in the paper?
%This section provides the overview over the main concepts of the paper, as well as a general comparison with existing work (to explain why a new framework is needed). 
%The detailed comparison with other proposals is found in the Related Work Section.
\section{Overview}\label{sec:overview}

%This subsection contextualizes the reader w.r.t. existing fields in which OI take place and explains why they fail to address the specific requirements for engineering PA.
%Do we need to discuss the past of LOI? Why? Here?
%R: we cannot ignore MAS research as it tackles similar issues regarding coordination and collaboration among distributed entities (DAI). The section is correct as it brings info about state-of-art research and discuss why they are not enough for what we need.
\subsection{Opportunistic D2D Interactions}

For opportunistic D2D interactions we refer to the situated engagement of pervasive devices as they are able to communicate through wireless networks, including both infrastructure-based (e.g., Wi-Fi) and ad hoc (Wi-Fi direct, Bluetooth, etc). 

%4: Examples of P2P/D2D based solutions
%TODO fix the intensity example
In its simplest form, a D2D interaction targets the exchange of application data (e.g., content generated by users in a mobile P2P application). However, some features are characterized by their autonomy 
%are targeting more complex and sophisticated goals 
(e.g., to autonomously decide when to sense the physical environment in specific areas of a city), their intensity (e.g., process and data-intensive tasks), or by their sensitivity to delay (e.g., players in a mobile multiplayer game). 

Much of today's pervasive applications are structured around the interactions remote servers (e.g., through RESTful endpoints) and users (e.g., through Activities in Android). % and follow a model-view-controller architectural pattern. 
%remote servers; 
Opportunistic D2D interactions are not common, despite its potential to enable new features and to improve non-functional attributes like availability, cost, and communication performance. There are many possible reasons for this, including the complexity of existing solutions and scalability problems.

Opportunistic interactions have been addressed by different research areas. 
Distributed artificial intelligence has made substantial contributions by means of the agent abstraction and the decentralized mechanisms that govern their collective behavior and adaptation~\cite{}. Other research communities also dedicated efforts in the same direction. Nonetheless, many of the proposals~\cite{} tackle specific types of interaction and do not provide a more general model that could be tailored for different applications. In contrast, in this work we follow a top-down approach in which modeling and programming abstractions are first introduced. Based on these building blocks we propose mechanisms and protocols tailored for particular scenarios of pervasive computing, namely different levels of volatility, scale and resource constraints.

%Nonetheless, there are significant differences with respect to the specification, design and programming of multi-agent systems (MAS) from how software engineers develop pervasive applications (PA)~\cite{}. Whereas multi-robots and other MAS based solutions are designed for interaction among agents, much of today's PA are structured around the interaction with users (e.g., through Activities in Android) and % and follow a model-view-controller architectural pattern. 
%remote servers; device-to-device interactions are not common. %, despite its potential to enable new features and to improve non-functional attributes like availability and communication performance. 
%%Finally, as already stated elsewhere, agents have the dual responsibility of performing application tasks and managing the interaction space. %TODO add Weyns critizism to the double role of agents.. 
%Despite the potential of adopting MAS solutions for enabling opportunistic interactions in pervasive applications, these significant differences, both at the conceptual and the operational levels, must be taken into account.



%%Finally, as already stated elsewhere, agents have the dual responsibility of performing application tasks and managing the interaction space. %TODO add Weyns critizism to the double role of agents.. 
%Despite the potential of adopting MAS solutions for enabling opportunistic interactions in pervasive applications, these significant differences, both at the conceptual and the operational levels, must be taken into account.



%Such differences must be taken into account when adapting solutions from MAS community.

%must not disrupt, but complement well established software engineering principles and practices.



%Conti et al. have proposed a middleware for opportunistic mobile social networks.  

%There ha also been extensive work to enable opportunistic interactions in pervasive computing research community~\cite{}. In many cases, the proposed solutions are narrow with respect to the type of interaction they aim to achieve; in others, the proposed solutions... %TODO: this is a critical point. I have yet to talk about Weyns (SA) and Zambonelli (SO). But before them, there's a vast literature from PC community that cannot be ignored. Why their proposals are not good enough? Because they don't deal with evolution/adaptation?



%must not disrupt, but complement well established software engineering principles and practices. In this sense, a pure agent-oriented approach fail to meet such requirement.  

%there is still a gap, both at the conceptual and concrete levels, of how software engineering community can make use of the relevant solutions from distributed artificial intelligence to the development of pervasive applications. % that exhibit similar requirements for decentralization, context-awareness, and adaptation. 

%Precisely, what does SoR means? 
\subsection{Separation of Responsibilities}



%3: functional plasticity of nowaday's devices
With the technology advancements, a class of pervasive devices has become able to perform different kinds of computations, as well as to communicate by different means and perceive the physical world through multiple sensors. As a consequence, the type of interaction among pervasive devices is less frequently defined by their model and specific functionality. Instead, it may depend on dynamic factors like their current resource levels and mobility.

%Moreover, a single application may contain software components with responsibilities and purposes that are activated in specific contexts. For instance, one device may become the coordinator for a period of time, while coordinated devices may become responsible for monitoring the environment through their sensors. 

%The consequence of this change is twofold. 
%
%First, 
%
%Second, 

To address this new scenario, the original focus of pervasive computing research targeting compositions and interactions among devices with specific functionality~\cite{Schumman} must now consider the ambiguity in the runtime and context-dependent decision of which devices should become responsible for what. %while they engage in opportunistic interactions for various purposes. 
Also, applying well known software engineering principles like separation of concerns and modularization to the distinct functionalities a device may become responsible should improve the quality of the pervasive application design and implementation, including the autonomous and contextual responsibility allocation.

%These enhancements entails a \textit{functional plasticity}, i.e., the ability of these devices to provide variations of their functionalities not only for different applications, but also in different contexts of the same application. 

%%2: from functional plasticity to a separation of functionalities
%In a client-server architecture, commonly used nowadays, the instances of a pervasive application depend on remote servers. The functionalities provided by each client are self-contained and do not address the needs of other clients. Hence, clients are symmetric with respect to each other; asymmetry is restricted to the relation between clients and servers. 
%In an different model, the functional plasticity of smartphones and other devices could be explored by letting application instances (hereafter referred to as an \textit{application node}) to assume distinct functionalities.

%, i.e., to assume different responsibilities over the application organization.

%In contrast with this rigid symmetry, 
%with the rigid symmetry that characterizes the behavior performed by client nodes in a client-server architecture -- commonly used in today's applications -- 
%or compared to the narrow set of functionalities provided by specialized and extremely resource-constrained devices, 

%In this work, we tackle this challenge with a context-aware assignment of responsibilities.




%This subsection contextualizes the reader about role-orientation in the literature and the novelty of using the role abstraction to build collaborative PA
\subsection{Role-orientation}

%For this, we propose to elevate the concept of a role to a first-class abstraction. 
%A tree with the types of role: symmetric and asymmetric branches with the corresponding types of roles in increasing levels of concreteness 

%What's the difference between tasks and roles?

The concept of roles has been applied in very different areas of information systems, including object-oriented programming, distributed multi-agents, role-based access control, and others. Despite its widespread adoption, there is no common definition for the concept of roles, but actually distinct meanings depending on the context they are employed. Nonetheless, a role is generally associated with rights, responsibilities, and capabilities~\cite{Roles:Survey}.

There is an extensive number of proposals in which the role abstraction is an important aspect or even a key part of the solution. However, not much attention has been paid to the use of role abstraction in the context of pervasive computing. %specially when considering the plethora of heterogeneous devices or situations of high volatility. 
This work aims to fulfill this gap. % by combining role and other organizational abstractions with the state of the art in discentralized self-adaptation and self-organization mechanisms.

%In specific, role-orientation could leverage the potential of context-dependent and opportunistic interactions among pervasive devices.

\subsection{Self-organization and Self-adaptation}

Whereas self-organization has been proposed and used for building decentralized, scalable, and adaptable multi-agent systems, self-adaptation is yet to show its feasibility when subject to volatility and large scale of adaptive entities (e.g., distributed components, agents, etc) and no centralized control. 

The gap between bottom-up self-organization and top-down self-adaptation has been a focus of research~\cite{Kramer:, Zambonelli:, Weyns:}. Such a combination may deem beneficial in the context of pervasive computing: in one hand, the volatility and resource limitations characterizing pervasive devices could cope with the scalability achieved by decentralized self-organization mechanisms. In the other hand, a self-adaptation control loop could mitigate undesirable behaviors that may emerge from self-organization and to further improve the capabilities of the system to adapt to varying situations~\cite{Zambonelli:}. In this paper, we propose the combination of both strategies in the formation and adaptation of opportunistic organizations of pervasive entities.

%Thus, the challenge is twofold: to preserve the well known benefits of existing software engineering methodologies, and inherit/adapt the mechanisms from DAI that can cope with the characteristic of pervasive applications.

%This subsection presents a set of situations in which opportunistic organizations are formed to enable the interaction among pervasive application nodes
%In specific, the interactions should include: 
%collaborative data fetching from remote servers
%collaborative data posting to remote servers
%coordenation of autonomous sensing activities
%a possibility is to describe some apps used by a PhD student, such as: monitoring of public traffic (MCS); air-conditioning temperature consensus; collaborative music streaming;
%\subsection{Motivating Scenario}\label{sec:motivating}
%
%\subsubsection{Mobile Crowd-sensing}
%
%to illustrate the need of opportunistic interactions in pervasive computing, we present a set of applications hosted by the mobile device of an hypothetical user. We start with an example of an opportunistic mobile crowd-sensing (MCS) application installed by the user in its smartphone for public transport monitoring.
%
%MCS consists of a paradigm in which the sensors of user-companioned devices are employed in the measurement of urban and social phenomena~\cite{Guo:2015}. Existing MCS applications range from private and public urban transportation monitoring (e.g., Waze\footnote{https://www.waze.com} and Moovit\footnote{https://www.moovitapp.com}) to atmospheric pressure, noise and air pollution measurement (e.g., PressureNet\footnote{http://www.pressurenet.io}, AirPatrol~\footnote{http://www.crowdfunder.co.uk/crowdsource-air-pollution-in-london}). In transportation monitoring applications, geolocation -- fetched from GPS sensors -- is the main data to be collected, while in other cases, geolocation provides geographic contextualization of the data from other sensors. Finally, some MCS applications require additional wearable sensors (e.g., to monitor the quality of air in AirPatrol) that communicates its results through Bluetooth to a main device (e.g., smartphone).
%
%Many MCS applications target a well defined geographic region and period of time. Accordingly, multiple mobile devices running the application may be sharing the same Wi-Fi network or within range of D2D communication (\textbf{C1}). Also, while moving within a campaign area, devices may suffer from fluctuations of the received Wi-Fi/Bluetooth/GPS signals, as well as exhibit distinct levels of battery and sensors of different quality (\textbf{C2}).
%%Also, user-companioned devices hosting the application are expected to join/leave the campaign area (\textbf{C2}). 
%Whereas some campaigns allow data to be analyzed later, real-time crowd-sensing require a minimum delay between the collection of data and their delivery to backend servers for further processing and publication (\textbf{C3}). Sensors like GPS impose a significant battery drain, and campaigns like transportation monitoring require a frequent activation of this sensor (\textbf{C4}). Finally, with the popularization of crowd-sensing applications, a large number of participants may eventually co-exist in the same campaign area (\textbf{C5}). 
%
%Like other mobile applications, the majority of nowadays MCS are designed following a client-server architecture in which application nodes collect data from their sensors independently from one another. 
%%At most, sensing activities in the client applications are coordinated by the backend server, incurring in additional processing by these servers and exchange of data through Internet. 
%This common approach has the following drawbacks:
%
%\begin{itemize}
%	
%	\item \textbf{Coordination}
%	
%	\begin{enumerate}[label=-]
%		
%		\item Without coordination, application nodes cannot adapt to situations in which multiple devices can provide the same information, thus they tend to consume unnecessary resources as sensing tasks could otherwise be allocated to a subset of the available devices.
%		
%		\item With coordination, nodes could agree on which nodes should perform which tasks at each moment, taking both individual and social contexts into account (e.g., churn of devices, variations in the quality of sensors measurements, intermittent connection with servers, low battery level, etc). 
%		
%	\end{enumerate}
%	
%	\item \textbf{Collaboration}
%	
%	\begin{enumerate}[label=-]
%		
%		\item Without collaboration, each node sends its data to the backend server responsible for filtering and aggregating samples, increasing the the server load and the Internet usage. Additionally, all nodes are assumed to have Internet access to communicate with the server and no collaborative of data is employed (e.g., \cite{Rajagopalan:2006});
%		
%		\item With collaboration, elected application nodes could aggregate data collected from different sensors (e.g., \cite{Wang:2015}) and average those with higher accuracy. Accordingly, less data would to be transmitted to and processed by backend servers. 
%		
%	\end{enumerate}
%\end{itemize}
%
%%\subsubsection{Collaborative Music Streaming}
%
%%Next, we consider the situation in which the user arrives at the office he/she shares with co-workers. To avoid common disagreements on the temperature to be set in the air-conditioning system, co-workers make use of an application that collect their opinion and provides a target temperature consensus. To avoid the costs of a backend server infrastructure, the application has been designed to explore device-to-device interactions. To this end, devices must discover each other and recognize the physical location they are in before joining a consensus group representing the office.
%%
%%In such ad hoc scenario, a single device may become responsible for collecting votes and communicating with the air-conditioning control interface. However, there are no guarantees with respect to the presence of users in the room, which raises the question of how to assign this role as devices enter and leave the room. 
%
%\subsubsection{Local Group Messaging}
%
%the adoption of messaging applications has skyrocket in the last decade, achieving the marks of a billion users for the two most popular applications. One of the most appealing features of these apps are the groups in which a set of users can share messages and, more and more frequently, contents like photos, audio, and videos. In many cases, people working together make use of a group for co-workers. Also common, people sharing the same network infrastructure or within range of ad hoc networking. Today, these applications do not explore device-to-device message exchange, but always rely on backend servers to relay the messages, even when generated and consumed locally. 
%
%One way of exploring opportunistic interactions could be to assign, among the subset of group members within communication range, the role of a \textit{local-relay}. Such role would be responsible for receiving messages from the server and relaying them locally to other group members. The main benefit would be to reduce the client-server communication, as the content should be downloaded once and distributed locally. Also, devices without Internet connection could still communicate through ad hoc network.
%
%\subsubsection{Collaborative Music Streaming}
%
%last but not least, we present a situation in which our user has finished work and join his colleagues in the park nearby for a picknick. Nowadays, popular services allow users to listen to music streamed from the Internet to their mobile devices. It has also become popular the use of portable speakers equipped with Bluetooth, so that users can enjoy a more powerful audio than provided by their smartphones and tablets. This combination of pervasive devices is specially appealing for external gatherings in parks and other recreational areas. Nowadays, many of these spaces provide public Wi-Fi and/or good cellular network service coverage (\textbf{C1}). However, both the streaming of music from the Internet and the communication through Bluetooth are expensive features in terms of battery and networking, as large volumes of data must be transferred (\textbf{C3}) with minimum delay (\textbf{C4}). Hence, to mitigate the battery drain and, if no Wi-Fi is available, to alleviate the amount of data each device has to download from its data plan, as many as possible devices should share this responsibility. In other words:
%
%\begin{itemize}
%	
%	\item the playlist interface must by synchronized among smartphones for collaborative modifications;
%	
%	\item a \textit{streaming-role} must be dynamically assigned to one device at a time without interrupting the music play;
%	
%	\item conditions such as the battery level, as well as the Internet and Bluetooth throughputs must be taken into account; and
%	
%	\item as many capable devices as possible must play the streaming-role at different times.
%	
%\end{itemize}
%
%
%Among the assumptions and characteristics of the scenario tackled by work, the following are considered as key features:
%
%\begin{enumerate}[label=C\arabic*]
%	
%	\item \textbf{Connectivity:} devices are expected to communicate with each other, while in the same area, through infrastructure Wi-Fi or D2D communication.
%	
%	\item \textbf{Volatility and Heterogeneity:} application nodes are expected to enter/leave a given zone/network without notice; also, the physical and computational environment in which application nodes operate is subject to changes that may affect capabilities required by the application.
%	
%	\item \textbf{Intensity:} application nodes may be required to perform resource-intensive tasks or to exchange large volumes of data, or a mix of both.
%	
%	\item \textbf{Delay-sensitive:} among the application features, some may be sensible to latency.
%	
%	\item \textbf{Scale:} up to a large number of application nodes are expected to co-exist in a given zone/network and potentially interact.
%
%	
%\end{enumerate}
%
%Next, we present examples of applications with use cases that motivate our approach.




\section{Self-organizing Interaction Spaces}\label{sec:edge_spaces}

\subsection{Role-orientation}

%1: The today's client-server model and the need for role-orientation
In today's client-server model, the different functionalities are modeled and programmed as part of a monolithic application. For the applications whose nodes are expected to interact, collaborate and play others than just the role of a client from a backend server, the distinct functionalities of each role form a concern of their own, i.e., they must be designed and programmed accordingly. 

%2: What do we propose in terms of role-orientation
In this work, we inherit the concept of a role, commonly used in human organizations, as part of a framework for the engineering of pervasive applications. By making the concept of a role a first-class abstraction, we provide the means to, since the early stages of its development, reason on the organizational aspects of the application, namely:

\begin{itemize}
	
	\item what are the roles in the application-to-be;
	
	\item what functionalities must be provided by each role;
	
	\item what capabilities are required by these functionalities; and
	
	%(e.g., computational resources, specific hardware components, etc); 
	
	\item how existing roles relate to each other, including: 
	
	\begin{enumerate}[label=-]
		
		\item the data flow, e.g.:
			\subitem coordinator $\leftarrow$ coordinated 
			\subitem aggregator  $\rightarrow$ sensors 
	
		\item their control hierarchy, e.g.:%. E.g.: 
			\subitem orchestration (hierarchical)
			\subitem choreography (non-hierarchical)
	
		\item their symmetry, i.e.:
			\subitem between same roles (symmetric) 
			\subitem between different roles (asymmetric)
		%. E.g.: symmetric relation between file-sharing peers, or asymmetric relations between server/client, supervisor/follower, coordinator/coordinated, provider/consumer, etc
		
	\end{enumerate}	
	
\end{itemize}

Figure~\ref{fig:asymmetry} illustrates cases of both symmetrical and asymmetrical relations among functional roles. 

\begin{figure}[t!]
	\centering
	\includegraphics[width=0.48\textwidth]{figures/asymmetry}
	\caption{Left: application nodes perform the same behavior (e.g., content sharing); middle: application nodes perform asymmetric roles without hierarchy among them (e.g., different sensing tasks); right: one node performs an hierarchical control (e.g., coordinator) or communication (e.g., data aggregator).}
	\label{fig:asymmetry}
\end{figure}

%TODO: add here the fitness function definition and examples

\subsection{Groups}\label{sec:groups}

\begin{figure}[t!]
	\centering
	\includegraphics[width=0.9\linewidth]{figures/group_view}
	\caption{Example of a group with a  membership criteria; out of 7 members, 2 nodes are idle (light blue) and 5 nodes are playing one of two types of roles: 1 node (red) plays a role related to the role played by the 4 others nodes (orange);}
	\label{fig:group_view}
\end{figure}

%1: Briedge with the role abstraction
The application organization -- so far represented by the roles that can be played by its nodes -- may be further characterized by its divisions, here named as \textit{groups}. 

%2: What a group is for an organization
At its highest level, a group boundary is limited by the network partition in which application nodes can interact. Nonetheless, these nodes, or a subset of them, may exhibit common properties or states of interest to the application. For instance, nodes may be grouped according to a \textit{functional} criteria (e.g., a group of nodes able to fetch data from a specific type of sensor), a \textit{non-functional} criteria (e.g., all nodes within a specific geographical area), or a mix of both (e.g., all nodes able to fetch data from a specific type of sensor within a specific geographical area). 

%Thus, multiple types of criteria may define the membership of a group.

%2: What a group is for a node
To its members, a group defines a \textit{social context} in which they (a) may or (b) must play certain roles. As the first case models a more general case of later case, we adopt a relaxed membership causality (a), i.e., group membership only defines the context in which its members \textit{may} play one or more roles, unless the group specification makes such restriction. Each of these cases are specified as follows:

\begin{itemize}
	
	\item \textbf{Strict:} a m-m specification (m instances of a role in a group of m nodes) defines a strict membership, i.e., a role that must be played by all members.
	
	\item \textbf{Relaxed:} a k-m specification defines a relaxed membership, i.e., a role to be played by a subset $K$ of the set $M$ of nodes in the group, with $|K| = k$, $|M| = m$, and $k < m$.
	
\end{itemize}

%3: The k-out-of-m allocation problem
Whenever the group specification is relaxed, the actual distribution of roles to group members must be decided dynamically (k-out-of-m allocation problem). This kind of specification is particularly useful for applications that explore the resources from a crowd of devices. For example, in mobile crowd-sensing, due to the potentially high density of devices within a given area, there may be a surplus of nodes able to perform one or more sensing tasks. Accordingly, a crowd-sensing group could be formed by all capable nodes, and only a few would actually be playing the sensing roles (after solving the k-out-of-m allocation). Figure~\ref{fig:group_view} shows the representation of a group with relaxed membership specification and two types of roles.

%4: The refined k-out-of-m allocation problem
The same reasoning applies to a group specified with two or more types of roles. If there is no predefined criteria for their assignment to specific nodes (e.g., based on the class of hosting devices), nodes must also agree on the specific roles they will play. Figure~\ref{fig:rationale} summarizes the rationale behind the assignment of functional roles to the nodes of an application based on their social and individual contexts.

%5: Role fitness and group specification
%Once the cases in which roles must be dynamically assigned to the members of a group has been depicted, and before moving to the more concrete group specification, we introduce the last key concept of the framework: the \textit{role fitness}. 

\subsection{Group-Role Specification}\label{sec:group_specification}


%1: What a role fitness is
Depending on the functionality to be provided by an application role, there may be some objective criteria to guide the decision of which nodes, at a given context, are suitable or represent the best candidates to play that role. To this end, we propose the specification of a \textit{role fitness} as a function composed of:

\begin{itemize}
	
	\item \textbf{Restrictive criteria:} consists of boolean variables whose satisfaction is a required condition for a role to be played by an application node.
	
	\item \textbf{Comparative criteria:} consists of the positive real scale indicating the fitness of a node in playing a role.
	
\end{itemize}

%The modeling of an application group starts with the specification of its membership criteria. 

%2: What each part composing a role fitness is useful for
Restrictive criteria is useful for filtering out nodes whose static (e.g., a hardware component) or dynamic (e.g., the battery level) capabilities are not compatible with the functionalities to be provided by a role. In contrast, comparative criteria distinguishes capable nodes in terms of their fitness to play a role. Thus, this type of criteria must be taken into account by the solution of the k-out-of-m role allocation problem. Next, a list of static and dynamic aspects of pervasive devices are presented as potential restrictive or comparative criteria:

\begin{itemize}
	
	\item Static criteria
	
	\begin{itemize}
		\item \textbf{Hardware capabilities:} refers to the presence of a given hardware component/module. E.g.: camera, GPS, thermometer, accelerometer, gyroscope, etc.
	\end{itemize}
	
	\item Dynamic criteria
	
	\begin{itemize}
		\item \textbf{Physical world:} refers to the physical world states a node must operate in. E.g.: its current battery level, available memory, geolocation coordinates, acceleration, speed, temperature, etc.
		
		\item \textbf{Application domain:} refers to the application states a node must be to belong to a group. E.g.: currently a member of another group (or non-member), joining a chat or game session, etc.
	\end{itemize}
\end{itemize}


%As SOIS refers to the collective of application nodes and their dynamic relations, 

To enable a flexible, intuitive, and unified placeholder for both group and role specifications, we propose the use of an hierarchical syntax. For this, we have used the eXtensible Markup Language (XML), as its standardized and well-known syntax enables the hierarchical representation of groups and roles, as well as the definition of their attributes. Next, we illustrate this specification for each of the two examples in Section~\ref{sec:motivating}.

\lstset{
	language=XML,
	breaklines=true,
	backgroundcolor=\color{lightgrey},
	basicstyle=\small\color{black},
	keywords={group,cardinality,role,name,criteria,type,term,value,minimum,maximum, pattern, after},
	keywordstyle={\bfseries\color{blue}},
	numbers=left,
	numbersep=5pt,
	numberstyle=\tiny\color{black}
}

\subsubsection{Collaborative Music Streaming} for this application, we envisioned one group with a single streamer role position (lines 1 and 3 in Listing~\ref{lst:ms_criteria}). As the functionalities of this role requires both Internet (to stream music from remote servers) and Bluetooth (to stream music to the stereo), two restrictive criteria have been added for both Internet and Bluetooth capabilities (lines 4 and 5). Finally, as the activity of streaming data in both directions are battery consuming, an additional criteria (line 6) has two purposes: a restrictive one (only devices with more than 20\% battery) and a comparative one (the best candidates are those with more battery level).

\begin{lstlisting}[caption=Specification of the music streaming group, label=lst:ms_criteria, captionpos=t]
<group name="music-streaming">

  <role name="streamer" cardinality="1">
    <criteria type="boolean" term="INTERNET" value="TRUE" />
    <criteria type="boolean" term="BLUETOOTH" value="TRUE" />
    <criteria type="float" term="BATTERY_LEVEL" minimum="20" />
  </role>
</group>
\end{lstlisting}

\subsubsection{Public Transport Monitoring}  

%TODO Remember to mention the importance of saving battery and other resources as a condition for the participation of users in MCS campaings
a MCS application aims to monitor the real-time geolocation of public buses -- to notify waiting passengers about their whereabouts -- and to register and later report unusual acceleration and deceleration events that may affect the user experience in this service. To address this scenario, a \textit{bus-monitoring} group (lines 1 in Listing~\ref{lst:bm_criteria}) is defined with three roles: a geolocator (line 5), an accelerometer (line 10), and an aggregator (14). In contrast with the previous example, a group level criteria (line 3) has been added, which means all roles inherit a minimum battery level of 15\% as a restrictive criteria. This criteria is overwritten by the geolocator role (line 6), as the GPS sensor consumes significantly more battery. Also, each sensing role has a corresponding restrictive criteria for the sensor it requires (lines 7 and 11). Finally, as the aggregator is responsible for receiving data from the other nodes and sending a preprocessed version to the backend server, Internet has been added as a restrictive criteria.

Another novelty in Listing~\ref{lst:bm_criteria} is the parametrized cardinality ($k1$ and $k2$). We intentionally left these fine-tunning parameters unspecified as the actual number of instances of each sensing role may depend on the context they operate: the higher the overall accuracy, less samples need to be aggregated. Hence, at runtime, the aggregator could provide feedback with respect to $k1$ and $k2$ based on the data it receives from the instances of each sensing role. 



\begin{lstlisting}[caption=Specification of a bus monitoring group, label=lst:bm_criteria, captionpos=t]
<group name="bus-monitoring">
  
  <criteria type="float" term="BATTERY_LEVEL" minimum="15"/>
  
  <role name="geolocator" cardinality="k1">
    <criteria type="boolean" term="GPS" value="TRUE" />
    <criteria type="float" term="BATTERY_LEVEL" minimum="30" />
  </role>
  
  <role name="accelerometer" cardinality="k2">
    <criteria type="boolean" term="ACCELEROMETER" value="TRUE" />
  </role>
  
  <role name="aggregator" cardinality="1">
    <criteria type="boolean" term="INTERNET" value="TRUE" />
  </role>
</group>
\end{lstlisting}

To avoid disturbing the users with the need of starting the application whenever they are within a bus, a fully opportunistic crowd-sensing solution~\cite{Guo:2015} requires the automatic detection of such context. By means of our framework, the facts defining this context can be modeled and later verified by the application. For example, if we assume that city buses provide wi-fi service -- as in an increasing number of real case scenarios -- the context of a passenger inside a bus ride can be defined with the additional restrictive criteria in Listing~\ref{lst:br_criteria}.

\begin{lstlisting}[caption=Additional criteria to specify a bus ride context, label=lst:br_criteria, captionpos=t]
<group name="bus-monitoring">

  <criteria type="string" term="BSSID" pattern="COMPANY_NAME" />
  <criteria type="float" term="WIFI_SIGNAL" minimum="50" />
  <criteria type="boolean" term="MOOVING" value="TRUE" after="300" />
  
  (...)

</group>
\end{lstlisting}

The additional criteria in Listing~\ref{lst:br_criteria} should be parsed as follows: the BSSID criteria (line 1) requires that a Wi-Fi with a basic service set identifier (BSSID) matching the pattern used by the bus company (e.g., the company's name); plus, the WI-FI signal strength (line 3) must not be less than 50\%, meaning the user is likely a passenger within the bus and not just in a nearby location. Finally, an important criteria identifying a bus ride is given by mobility: if the device's location, as measured by low-power detection methods like triangulation, remains unchanged for large periods, the use is either not in a bus ride or the bus is jammed and must not have its location monitored until it resumes its trip. In particular, this criteria has been modeled (line 5) with a boolean condition that fails unless it has been satisfied for more than 5 minutes (300 seconds).

Each criteria in a group-role specification can be mapped to a concrete function returning either a boolean value (for restrictive) or a float value (for comparative). The conjunction of these terms produces what we define as a \textit{restrictive function} and a \textit{fitness function}:

\begin{equation}\label{eq:rrc}
rrc(C_r) = rc_1(c_1) \wedge rc_2(c_2) \wedge ... \wedge rc_i(c_i)
\end{equation}

\begin{equation}\label{eq:fitness}
f(C_c) = cc_1(c_1) \wedge cc_2(c_2) \wedge ... \wedge cc_j(c_j)
\end{equation}

\noindent
with $C_r$ and $C_c$ the set of measurable context facts (restrictive and comparative, respectively) for a role. Last but not least, the entrance of a node to a group (group membership) is conditioned to its satisfaction of at least one $rrc$, i.e.:

\begin{equation}\label{eq:membership}
gm([C_{r1},...,C_{rp}]) = rrc_1(C_{r1}) \vee ... \vee rrc_p(C_{rp})
\end{equation}


\section{Self-organization Mechanisms}\label{sec:self_organization}
%TODO Merge this paragraph into the existing one(s)
%5: When relations among nodes can be established
%Still regarding collaboration, we argue that, 
Whenever two nodes of an application have the opportunity to communicate, there is a potential for a relation between them to be established. 
%In this relation, each part plays a similar (symmetric) or different (asymmetric) roles. 
%The same premise holds for multiple interconnected devices. 
%6: The nature of the relation between nodes in different situations
While the nature of the relation between devices of different classes is mostly predefined (e.g., between a smartphone that relays notifications to a watch through Bluetooth; or a tablet that streams content to an smart-tv), the relation between devices of the same class may depend on the dynamic context each device operates (e.g., a smartphone that, in a given situation, acts as the gateway for other smatphones without Internet access). Thus, not only the nodes of an application may assume distinct roles, but the actual allocation of these roles may depend on the individual and social contexts of each interacting node.

%1: On the need for a maleable organization structure
The structure of the application organization
%, formed by the two abstractions presented -- namely, roles and groups -- 
must be malleable (plastic) to accommodate changes in the units composing the pervasive ecosystem and in its physical environment. 

%2: What must evolve
In contrast with the much less dynamic cases of organizations in human societies (e.g. industry and military organizations), the volatility that affects and characterizes pervasive and mobile devices, caused by their mobility or fluctuations of their resources, may imply the formation or dissolution of relations among nodes and require the reassessment of the roles they play in the organization.

%3: What self-organization means in the literature
%TODO: move this to the related works
Self-organization have been extensively studied in the context of multi-agents and other fields as a phenomena and method to achieve system properties and goals by means of actions and interactions between individuals based on their local knowledge and no external control~\cite{DiMarzoSerugendo:2005, Banzhaf:2009}. In contrast, self-organization in this work refers to the constitution and adaptation of the application organization as understood by the serendipitous relations between application nodes formed and dissolved while the are able to communicate and collaborate. In specific, we propose mechanisms for each of the following organization aspects:

\begin{itemize}
	
	\item \textbf{Group membership:} as nodes join or leave a group, a group registry must be kept consistent among members;
	%a node must join or leave a group according to its satisfaction to the criteria in that group; 
	to address this, a \textit{self-grouping} mechanism is proposed;
	
	\item \textbf{Role allocation:} the nodes within a group must agree on which roles they shall play; to address this, a  \textit{discentralized role election} mechanism is proposed;
		
\end{itemize}

\subsection{Self-grouping} 

\subsubsection{\textbf{Definition}} As the idea of a group in this work is not related to security, the \textit{self-grouping} mechanism is not controlled by special-purpose components external to the system; instead, each node is responsible for checking its own satisfaction to the existing membership criteria in the application organization. 

Whenever an application group comes into contact with others and join an existing group, this event must be  advertised to current members, who in turn update their group membership registry. Analogously, when a member node leaves, this event must also be perceived by existing members, who must update their registry.

\subsubsection{\textbf{Grouping Protocol}} 

%1: Check restrictive criteria
A node should join a group if it satisfies all restrictive criteria of at least one of the roles (hereafter referred to as RRC) in that group. In this paper, we assume the group specification to be performed off-line and be available for each application node as part of its resources (e.g., as an XML specification, as depicted in Section~\ref{sec:group_specification}). Thus, each node have access the restrictive criteria of roles in a group specification and can locally check for their satisfaction. 



The diagrams in Figures~\ref{fig:join_or_leave} and~\ref{fig:registry_replica} present the activities of the self-grouping protocol. At each iteration, the node checks for its satisfaction of the RRC (Figure~\ref{fig:join_or_leave}, activity 1). If no RRC is satisfied and the node is currently a member, it must leave the group and advertise this event to the other members (Figure~\ref{fig:join_or_leave}, activity 2). Conversely, if a node satisfies an RRC, it proceed by joining the group and advertise this event (Figure~\ref{fig:join_or_leave}, activity 3). Last but not least, as nodes entering a group have no information about its current members, this knowledge must be acquired. In specific, after been notified about the new member, the oldest group member is responsible for sending the registry replica to the newcomer (Figure~\ref{fig:registry_replica}, activity 1).

%TODO despict how the problem of time coupling by using gossip or other algorithms for advertisement/discovery

\begin{figure}[t!]
	\centering
	\begin{subfigure}[b]{0.45\textwidth}
		\centering
		\includegraphics[width=0.9\textwidth]{figures/join_or_leave}
		\caption{Protocol for joining or leaving a group}
		\label{fig:join_or_leave}
	\end{subfigure}%
	
	\begin{subfigure}[b]{0.45\textwidth}
		\centering
		\includegraphics[width=0.5\textwidth]{figures/registry_replica}
		\caption{Oldest member protocol following a new member event}
		\label{fig:registry_replica}
	\end{subfigure}
	\caption{The self-grouping protocol}
	\label{fig:self_grouping}
\end{figure}


%TODO: describe which nodes is responsible for sending this information


%TODO: decide if the group registry must be sent to newcomers only for the first time they join the group (which implies to continue updating them while they are outside the group). R: always send the the group registry to newcomers
 
\subsection{Distributed Role Allocation} 


\begin{figure}[t!]
	\centering
	\begin{subfigure}[b]{0.45\textwidth}
		\centering
		\includegraphics[width=0.65\textwidth]{figures/elected_node}
		\caption{Elected node's protocol for advertising its $FS$}
		\label{fig:elected_node}
	\end{subfigure}%
	
	\begin{subfigure}[b]{0.55\textwidth}
		\centering
		\includegraphics[width=0.5\textwidth]{figures/eligible_node}
		\caption{Eligible node's protocol for triggering a challenge event}
		\label{fig:eligible_node}
	\end{subfigure}
	\caption{The distributed role election protocol}
	\label{fig:role_allocation}
\end{figure}
 
\subsubsection{\textbf{Definition}} The decision of which nodes should be assigned to which roles in a group may depend on many aspects. Any node capable of performing a role is a potential candidate. Notwithstanding this, attribute such as the wireless connections throughput, sensors accuracy, as well as the availability of resources like battery, memory and processing capacity tend to variate from one node to another and throughout time. Thus, a balanced role allocation must respect the trade off between what is best for the application goals and for the individual devices. 

\subsubsection{\textbf{Allocation Classification}} Gerkey and Matarić~\cite{Gerkey:2004} proposed a taxonomy for the classification of task allocation problems along three axes. We adopted this taxonomy as a means to improve the characterization of the role allocation problem.

In the first axis, robots~\footnote{in their proposed taxonomy, the authors refers to \textit{robots} and \textit{tasks}, while in this paper we refer to the (application) \textit{nodes} and the \textit{roles} they can perform.} 
are categorized into single-task versus multi-task robots. In the second axis, tasks are categorized into single-robot versus multi-robot tasks. Finally, in the third axis, the allocation is also categorized into two types: instantaneous assignment or time-extended assignment.

Regarding the first axis, as application nodes are generally capable of performing more than one role at a time (e.g., to fetch from multiple types of sensors) nodes are here considered as multi-role (analogous to multi-task robots in ~\cite{Gerkey:2004}). Regarding the second axis, many types of roles are to be performed by a single application node (e.g., the sensor data aggregator role), while others must be simultaneously performed by multiple nodes (e.g., simultaneously fetch from the same type of sensor). Thus, roles can be either single-node (analogous to single-robot tasks in ~\cite{Gerkey:2004}) or multi-node (analogous to multi-robot tasks in ~\cite{Gerkey:2004}). Finally, due to the volatility of mobile devices, including churn and fluctuations of their capabilities, the scheduling of future allocations tends to fail. Accordingly, we consider an instantaneous and adaptable assignment of roles based on the context of the involved devices. 

\subsubsection{\textbf{Allocation Method}} Auction-based allocation methods have been extensively studied in the multi-agents/robots domain~\cite{Korsah:2013}. In comparison with the task allocation problem tackled by auction-based methods, an assignment of roles to pervasive and mobile devices subject to high volatility needs to evolve as nodes leave or join the system and their fitness change. If the assignment should continuously reflect any context change, a frequent message exchange between bidding and auctioneers would lead to excessive communication overhead. Accordingly, the replacement of existing role positions (herein referred as reelection) should take into account the trade off between the gain of having a more fit node elected and the replacement cost. 

%To mitigate this problem, nodes should evaluate their fitness (self-evaluation) and only advertise it to other group members if a delta in the value occurs.

To address this problem, we propose an event-based protocol inspired in electoral systems. Accordingly, instead of reevaluating and potentially changing the assignment of role positions at fixed control periods, we define the set of events in which new elections for these positions should be performed (event-triggered control). In specific, these events are:	

\begin{itemize}
	
	\item \textbf{Vacancy:} a new role position is opened or the node playing a role exists the system (churn) due to a network disconnection from the remaining nodes or the abnormal termination of the application instance after a failure;
	
	\item \textbf{Resignation:} the node playing a role calls for a new election before quitting; e.g., the application node is exiting the system following a termination command issued by the users or the operational system;
	
	\item \textbf{Challenge:} one of eligible nodes calls for a new election after detecting it has a significantly higher fitness score for that role position than the actual node by the time it was elected;
	
	%\item \textbf{Consensus:} the set of nodes dependent from a functionality provided by an elected node calls for a new election; e.g., the latency experienced between a set of nodes and an elected node is high despite its higher fitness to play that role (latency is not part of the fitness criteria);
	
\end{itemize}
\medskip
 
The subtitle difference between vacancy and resignation consists of the way it is handled: the former implies in a \textit{vacant} period in which the role functionality is interrupted until the vacancy is detected and occupied (hard transition), whereas the later allows the replacement of the position before discontinuing the functionality provided by the resigning node (soft transition). 

In both cases, the eligible nodes must proceed with the election of a candidate by checking their current fitness score ($FS$) and advertising it (bidding). After receiving the fitness scores from all other candidates, each node becomes aware of the election result. Thus, each node proceeds either by: assuming the role position and registering itself as the winner (if it has the highest $FS$); or registering the identify of the winner (otherwise). 

The later case presents an inverse situation: the caller of the election is a node that perceives its actual fitness score ($FS_a$) as higher than the value scored by the elected node by time it was elected ($FS_e$). Thus, the challenger assumes this value has not increased, which must be confirmed or denied by the node currently in the position. To mitigate the communication overhead, the following strategy is proposed:

\begin{enumerate}

\item The challenge must only be called if the $FS_a$ of the challenger is greater than the $FS_e$ by a degree of $\delta$, i.e., $FS_a \ge \delta * FS_e$, with $\delta > 1$;

\item Assuming all eligible nodes adopt the same $\delta$, they are exempt of participating and the election happens between the challenger and the actual nodes;

\item To reduce the chances of having a failed challenge (when the actual $FS$ of the elected node has increased since it was elected), elected nodes should update their peers whenever their $FS$ has changed by a degree of $\delta$, i.e., $FS_a \ge \delta * FS_e$ or $FS_a \le (1 - \delta) * FS_e$.

\end{enumerate}

If the challenger node has indeed a higher $FS$ than its challenger, it assumes the role position and this result is advertised; otherwise, the actual fitness score of the elected node ($FS_a$) is advertised so that new challenges are based on the most updated score. The diagrams in Figures~\ref{fig:elected_node} and~\ref{fig:eligible_node} describe, respective, the protocol of an elected node and of an eligible node.




\section{Evaluation and Discussion}\label{sec:evaluation}



The simulation experiments aimed at showing the benefits of the approach and measuring the overhead imposed by the self-organization mechanisms.
%, namely self-grouping and distributed role allocation. 
As these methods add no significant overhead in terms of processing or memory, an asymptotic analysis focused on the communication overhead.
%, as the exchange of messages through wireless mediums consumes battery from devices and is subject to delays that may interfere with the application behavior. 
As for the benefits of the approach, we executed simulation experiments of a public transport monitoring application using a pure client-server and using our framework. The goal was to compare, in each case, the following two metrics:

\begin{enumerate}[label=\textbf{M}\arabic*:]
	
	\item the total number of sensing tasks performed and requisitions fired from clients to backend servers (evaluates battery consumption and Internet traffic); and
	
	\item the total number of failures in reaching backend servers due to intermittent connectivity (evaluates robustness).
	
\end{enumerate}

%In specific, this metric was evaluated 

%\begin{enumerate}[label=\Alph*]
%	
%	\item The asymptotic overhead, given by worse case number of exchanged messages required by each self-organization mechanism, as a function of the number of application nodes, groups and roles (Section~\ref{fig:asymmetry}). 
%	
%	\item The measured overhead, given by the number of exchanged messages counted during simulated executions of a MCS application for public bus monitoring, in which the number of group/roles is fixed and the number of nodes, as well as their capabilities, varies according to probabilistic distributions.
%	
%\end{enumerate}


\subsection{Asymptotic Analysis} 

\subsubsection{\textbf{Self-grouping}} the complexity analysis was divided in two parts: a) the overhead when a node joins/leaves a group (registry update); b) the additional overhead when a node joins a group (registry copy).

In the worse case scenario, registry update (a) takes $n-1$ unicast messages ($O(n)$), with $n$ the group size, and a single registry line as payload. However, if a broadcast communication is used, a single broadcast message (e.g., an UDP broadcast over Wi-Fi) can advertise the registry update ($O(1)$).
%This is the case, e.g., of an UDP broadcast over the 802.11 network protocol (Wi-Fi). 
The registry copy (b), in its turn, requires a single unicast message to be transmitted each time a node joins a group ($O(1)$). Important to mention, in contrast with the registry update message, the registry copy includes information about all $n-1$ nodes in the group. 

\subsubsection{\textbf{Discentralized Role Allocation}} 

%TODO: check the definition of FS and use it accross the paper

this mechanism involves the exchange of fitness scores ($FS$) among eligible nodes in the advent of the events depicted in the previous subsection. The actual number of $FS$ messages sent/received during a role position election depends on how many nodes in the $n$ size group satisfy the role restrictive criteria (RRC) (see Eq.~\ref{eq:rrc} in Section~\ref{sec:edge_spaces}). Once more, the type of network is a determining factor.

In the worse case, represented by an election of a vacant position (following a \textit{vacancy} or \textit{resignation} event) and without broadcast, each node $e$ in the set of eligible nodes $E$ must send each other an unicast message. If $|E| = n$, the message count is given by $O(n * (n-1)) = O(n^2)$. If the nodes in $E$ can communicate through broadcast, this number is reduced to $O(n)$. In its turn, the challenge event produces smaller overhead as the exchange of messages is restricted to a request-response between challenger and challenged and to the subsequent advertisement of the result ($O(n)$ without broadcast, otherwise $(O(1))$). 

%added to the node's classification list in polynomial time ($O(p*log(p)))$, with $1 \le p \le n$. 

Once the communication overhead of a single election round is known, the overall overhead can be estimated by the number of groups and roles in the application and the frequency of events triggering new elections. Whereas the vacancy and resignation events should be handled with a new election, the frequency in which a challenge event happens is greatly affected by the choice for $\delta$: the higher this factor is, the lower the probability of a challenge (and the need for the elected nodes to update their peers about changes in their $FS$). Therefore, the decision of which $\delta$ to be used depends on the criticality of the attributes that compose the fitness score of a node. 



\subsection{Public Transport Monitoring Simulation}

\subsubsection{Experiments Design}

the group-role specification (List.~\ref{lst:bm_criteria} in Section~\ref{sec:edge_spaces}) was used for the modeling of a dynamic bus monitoring application scenario. In it, the following variables were considered:

\begin{itemize}
	
	\item the number of nodes within a bus: \textit{from 2 to 10 nodes};
	
	\item the battery level of each node: \textit{from 10\% to 90\%};
	
	\item the Internet type of each node: \textit{cellular or Wi-Fi};
	
	\item the GPS signal in each node: \textit{from 0\% to 100\%};
	
\end{itemize}

In the client-server approach, each node tries to collect data from its accelerometer and GPS before sending it to the server through whatever type of Internet connection available (Wi-Fi or cellular data plan). In our approach, at most two simultaneous nodes are assigned to each type of sensor and one node to aggregate data before sending a preprocessed batch to the server. The number of times each sensor is fetched, as well as the number of times a connection between a client/aggregator and the server, were counted in separate (\textbf{M1}). Finally, we also counted the monitoring windows in which no measurement reached the server (\textbf{M2}).

%We modeled the cost of sampling the GPS as 5 power units (pw) and 1pw for the accelerometer, as in most cases it works continuously, but still requires a background activity running. Plus, we modeled the cost of sending data to the server through cellular plan as 10, through Wi-Fi as 5, and through 

The experiments were performed with, \textit{PeerSim}, an open source peer-to-peer simulator~\cite{p2p09-peersim}. This tool supports the creations of different network P2P topologies and provides a handful JAVA programming interface.



%To simulate a public bus monitoring crowd-sensing application, experiments were performed with an increasing number of nodes. Each role can potentially play a \textit{gps-monitor}, a \textit{acceleration-monitor}, or an \textit{aggregator} role. Each role has been implemented as a protocol extending the \textit{CDProtocol} class. At each simulation cycle, 
%
%the application roles of a \textit{gps-monitor} and \textit{acceleration-monitor} were implemented as protocols, which are called at each simulation cycle. 
%
%The simulation was executed in a computer with .... 

\subsection{Results}

Add Figures here

\subsection{Discussion}

Discuss the obtained results

\section{Related Work}\label{sec:related_work}

\subsection{Group and Role Abstractions}

The A-3 model~\cite{Baresi:2011:2} defined an architectural style consisting of groups that can be populated by a supervisor and its followers and composed with other groups. While this work share many of the motivations and have similarities with A-3, our model does not rely on the rigid \textit{supervisor-follower} structured, nor group compositions depend on shared members.
% features for topology control operations and 
In contrast with the more abstract A-3 support for self-adaptation~\cite{Baresi:2011:2}, we also investigate adaptation mechanisms that should guarantee the basic properties of groups, such as robustness, a high availability, efficient use of resources, and other attributes defined by the application through extension points. Finally, while A-3 rely on classical group communication methods, we investigate the integration of groups with tuple spaces for both inter-group and intra-group coordination.

Group and role abstractions has also been used in other domains. Ferber et al.~\cite{Ferber:2004} proposed an organization centered model for multi-agents systems that contrasts with agent centered models in which agents can communicate and interact freely. Among the problems of agent centered models, the authors cited security, modularity and the lack of support to other frameworks besides the multi-agent platform itself. In our work, we agree with those arguments as part of the justification for an organizational approach for distributed systems. Despite the model similarities, the works target different domains: instead of agents, we consider pervasive and mobile devices as the hosts of components that play roles in groups of distributed applications. %In addition, our work targets the particular problems of a high volatile and heterogeneous scenario that requires autonomous adaptation of the group-role organization.

 %Furthermore, we focus on the self-management of the group-based organization in the advent of context changes, while Ferber et al. have addressed only the conceptual elements of the organizational model.
 
\subsection{Self-organization and Self-adaptation} 
 
Kota et al.~\cite{Kota:2012} have proposed a method for adapting the relationship between agents in a multi-agent system. In their work, agents reason about adaptation using only historical knowledge about past interactions and the cost of adaptation (meta-reasoning). Despite the similarities with our work, namely the use of self-organization principles and the focus on the dynamics of relations, in that aspect our work addresses a different domain (pervasive applications) and adopt an organization-oriented perspective in which application nodes can play distinct roles. Thus, our focus is rather on the nature of the relation and its dynamics than solely on the decision of when or not nodes should interact. Notwithstanding this, the method proposed by Kota et al. for adapting the organization structure of problem solving agents has provided valuable insight for our work. 

A3-TAG, a programming model that facilitates the design of self-adaptive distributed systems based on group abstractions~[CT]. A3-TAG is an extension of the A-3 model, which is used as the organization model. A-3 key elements are groups and two types of roles, namely supervisor and follower. Each group has a supervisor and a variable number of followers. The main differences between this work and A3-TAG are dual: first, our organization model imposes no restriction about the number of roles a group can have, nor it specifies any hierarchy between roles. Second, our adaptation model is not based on coordination groups, but rather on the direct interaction between groups based on self-organization principles. We argue that the complexity of forming and managing coordination groups may result in excessive overhead. Instead, we propose a \textit{structureless} method to self-adapt the organization structure.

\subsection{Distributed Allocation Problem}

%TODO: Move this to the related works section;
%TODO: Add heterogeneity and volatility to the paragraph
In the literature, many works have tackled the problem of distributed task allocation~\cite{DTA}. In contrast with a task, a functional role defines a set of functionalities (possibly tasks) that a member of an organization is responsible to provide (perform). Hence, within an organization, a \textit{role} precedes a \textit{task}. Then, depending on the type of role, if multiple instances of a role have been assigned, a task allocation among these instances may still take place. Last but not least, while tasks usually have a concrete criteria for their beginning and completion and their assignment happens before task execution, roles lifespan tend to include multiple repetitions of a given functionality (or task). Therefore, in a dynamic scenario, a role assignment may have to evolve meanwhile roles are been performed.

Notwithstanding their differences, the two types of allocation problems share commonalities. For instance, in both cases an utility function may be used as a criteria for choosing an optimal or sub-optimal assignment of roles/tasks. Whereas the optimization of quality attributes may deem unfeasible due to its complexity, a sub-optimal allocation of can still be guided by the \textit{fitness} (or utility) of nodes in performing these tasks/roles. To this end, fitness/utility is modeled as real-value function of relevant features affecting one or more attributes of the application. Also, some of the existing taxonomy for classifying a task allocation problem can also be applied to the role allocation problem.




%!TEX root = main.tex
% -*- root: main.tex -*-
\section{Conclusion and Future Work}\label{sec:conclusion}

%Promissing...

Mobile crowd-sensing may be an exponent example of an application that exhibit the characteristics and requirements for autonomous interactions among its nodes. With the popularization of pervasive and mobile devices, existing and novel applications may benefit from the framework and the paradigm presented in this paper.

As future work, ....

% For peer review papers, you can put extra information on the cover
% page as needed:
% \ifCLASSOPTIONpeerreview
% \begin{center} \bfseries EDICS Category: 3-BBND \end{center}
% \fi
%
% For peerreview papers, this IEEEtran command inserts a page break and
% creates the second title. It will be ignored for other modes.
\IEEEpeerreviewmaketitle



%TODO uncomment after at least one item is present
%\begin{thebibliography}{1}
%
%\end{thebibliography}

\bibliographystyle{IEEEtran}
\bibliography{biblio}



% that's all folks
\end{document}


