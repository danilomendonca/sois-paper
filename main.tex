
%% bare_conf.tex
%% V1.4b
%% 2015/08/26
%% by Michael Shell
%% See:
%% http://www.michaelshell.org/
%% for current contact information.
%%
%% This is a skeleton file demonstrating the use of IEEEtran.cls
%% (requires IEEEtran.cls version 1.8b or later) with an IEEE
%% conference paper.
%%
%% Support sites:
%% http://www.michaelshell.org/tex/ieeetran/
%% http://www.ctan.org/pkg/ieeetran
%% and
%% http://www.ieee.org/

%%*************************************************************************
%% Legal Notice:
%% This code is offered as-is without any warranty either expressed or
%% implied; without even the implied warranty of MERCHANTABILITY or
%% FITNESS FOR A PARTICULAR PURPOSE! 
%% User assumes all risk.
%% In no event shall the IEEE or any contributor to this code be liable for
%% any damages or losses, including, but not limited to, incidental,
%% consequential, or any other damages, resulting from the use or misuse
%% of any information contained here.
%%
%% All comments are the opinions of their respective authors and are not
%% necessarily endorsed by the IEEE.
%%
%% This work is distributed under the LaTeX Project Public License (LPPL)
%% ( http://www.latex-project.org/ ) version 1.3, and may be freely used,
%% distributed and modified. A copy of the LPPL, version 1.3, is included
%% in the base LaTeX documentation of all distributions of LaTeX released
%% 2003/12/01 or later.
%% Retain all contribution notices and credits.
%% ** Modified files should be clearly indicated as such, including  **
%% ** renaming them and changing author support contact information. **
%%*************************************************************************


% *** Authors should verify (and, if needed, correct) their LaTeX system  ***
% *** with the testflow diagnostic prior to trusting their LaTeX platform ***
% *** with production work. The IEEE's font choices and paper sizes can   ***
% *** trigger bugs that do not appear when using other class files.       ***                          ***
% The testflow support page is at:
% http://www.michaelshell.org/tex/testflow/



\documentclass[conference]{IEEEtran}
% Some Computer Society conferences also require the compsoc mode option,
% but others use the standard conference format.
%
% If IEEEtran.cls has not been installed into the LaTeX system files,
% manually specify the path to it like:
% \documentclass[conference]{../sty/IEEEtran}



% Some very useful LaTeX packages include:
% (uncomment the ones you want to load)


% *** MISC UTILITY PACKAGES ***
%
%\usepackage{ifpdf}
% Heiko Oberdiek's ifpdf.sty is very useful if you need conditional
% compilation based on whether the output is pdf or dvi.
% usage:
% \ifpdf
%   % pdf code
% \else
%   % dvi code
% \fi
% The latest version of ifpdf.sty can be obtained from:
% http://www.ctan.org/pkg/ifpdf
% Also, note that IEEEtran.cls V1.7 and later provides a builtin
% \ifCLASSINFOpdf conditional that works the same way.
% When switching from latex to pdflatex and vice-versa, the compiler may
% have to be run twice to clear warning/error messages.






% *** CITATION PACKAGES ***
%
%\usepackage{cite}
% cite.sty was written by Donald Arseneau
% V1.6 and later of IEEEtran pre-defines the format of the cite.sty package
% \cite{} output to follow that of the IEEE. Loading the cite package will
% result in citation numbers being automatically sorted and properly
% "compressed/ranged". e.g., [1], [9], [2], [7], [5], [6] without using
% cite.sty will become [1], [2], [5]--[7], [9] using cite.sty. cite.sty's
% \cite will automatically add leading space, if needed. Use cite.sty's
% noadjust option (cite.sty V3.8 and later) if you want to turn this off
% such as if a citation ever needs to be enclosed in parenthesis.
% cite.sty is already installed on most LaTeX systems. Be sure and use
% version 5.0 (2009-03-20) and later if using hyperref.sty.
% The latest version can be obtained at:
% http://www.ctan.org/pkg/cite
% The documentation is contained in the cite.sty file itself.






% *** GRAPHICS RELATED PACKAGES ***
%
\ifCLASSINFOpdf
  \usepackage[pdftex]{graphicx}
  % declare the path(s) where your graphic files are
  % \graphicspath{{../pdf/}{../jpeg/}}
  % and their extensions so you won't have to specify these with
  % every instance of \includegraphics
  % \DeclareGraphicsExtensions{.pdf,.jpeg,.png}
  %Draw graphics
\usepackage{pgfplots}
\usepackage{pgfplotstable}
\pgfplotsset{compat=newest}
\pgfplotscreateplotcyclelist{my chart colors}{%
blue, solid, every mark/.append style={solid, fill=blue}, mark=*\\%
red, solid, every mark/.append style={solid, fill=red}, mark=square*\\%
brown, solid, every mark/.append style={solid, fill=brown}, mark=triangle*\\%
cyan, solid, every mark/.append style={solid, fill=gray}, mark=diamond*\\%
black, dashdotted, every mark/.append style={solid, fill=red}, mark=halfdiamond*\\%
olive, dashdotted, every mark/.append style={solid, fill=green}, mark=halfcircle*\\%
orange, dashdotted, every mark/.append style={solid, fill=gray}, mark=*\\%
purple, dashdotted, every mark/.append style={solid, fill=blue}, mark=halfsquare left*\\%
teal, densely dashed, every mark/.append style={solid, fill=green}, mark=oplus*\\%
violet, densely dashed, every mark/.append style={solid, fill=gray}, mark=diamond*\\%
magenta, densely dashed, every mark/.append style={solid, fill=blue}, mark=triangle*\\%
lime, densely dashed, every mark/.append style={solid, fill=red}, mark=square*\\%
}

\newcommand{\errorband}[5][]{ % x column, y column, error column, optional argument for setting style of the area plot
\pgfplotstableread[col sep=comma, skip first n=2]{#2}\datatable
    % Lower bound (invisible plot)
    \addplot [draw=none, stack plots=y, forget plot] table [
        x={#3},
        y expr=\thisrow{#4}-\thisrow{#5}
    ] {\datatable};

    % Stack twice the error, draw as area plot
    \addplot [draw=none, fill=gray!40, stack plots=y, area legend, #1] table [
        x={#3},
        y expr=2*\thisrow{#5}
    ] {\datatable} \closedcycle;

    % Reset stack using invisible plot
    \addplot [forget plot, stack plots=y,draw=none] table [x={#3}, y expr=-(\thisrow{#4}+\thisrow{#5})] {\datatable};
}
\else
  % or other class option (dvipsone, dvipdf, if not using dvips). graphicx
  % will default to the driver specified in the system graphics.cfg if no
  % driver is specified.
  % \usepackage[dvips]{graphicx}
  % declare the path(s) where your graphic files are
  % \graphicspath{{../eps/}}
  % and their extensions so you won't have to specify these with
  % every instance of \includegraphics
  % \DeclareGraphicsExtensions{.eps}
\fi
% graphicx was written by David Carlisle and Sebastian Rahtz. It is
% required if you want graphics, photos, etc. graphicx.sty is already
% installed on most LaTeX systems. The latest version and documentation
% can be obtained at: 
% http://www.ctan.org/pkg/graphicx
% Another good source of documentation is "Using Imported Graphics in
% LaTeX2e" by Keith Reckdahl which can be found at:
% http://www.ctan.org/pkg/epslatex
%
% latex, and pdflatex in dvi mode, support graphics in encapsulated
% postscript (.eps) format. pdflatex in pdf mode supports graphics
% in .pdf, .jpeg, .png and .mps (metapost) formats. Users should ensure
% that all non-photo figures use a vector format (.eps, .pdf, .mps) and
% not a bitmapped formats (.jpeg, .png). The IEEE frowns on bitmapped formats
% which can result in "jaggedy"/blurry rendering of lines and letters as
% well as large increases in file sizes.
%
% You can find documentation about the pdfTeX application at:
% http://www.tug.org/applications/pdftex





% *** MATH PACKAGES ***
%
%\usepackage{amsmath}
% A popular package from the American Mathematical Society that provides
% many useful and powerful commands for dealing with mathematics.
%
% Note that the amsmath package sets \interdisplaylinepenalty to 10000
% thus preventing page breaks from occurring within multiline equations. Use:
%\interdisplaylinepenalty=2500
% after loading amsmath to restore such page breaks as IEEEtran.cls normally
% does. amsmath.sty is already installed on most LaTeX systems. The latest
% version and documentation can be obtained at:
% http://www.ctan.org/pkg/amsmath





% *** SPECIALIZED LIST PACKAGES ***
%
%\usepackage{algorithmic}
% algorithmic.sty was written by Peter Williams and Rogerio Brito.
% This package provides an algorithmic environment fo describing algorithms.
% You can use the algorithmic environment in-text or within a figure
% environment to provide for a floating algorithm. Do NOT use the algorithm
% floating environment provided by algorithm.sty (by the same authors) or
% algorithm2e.sty (by Christophe Fiorio) as the IEEE does not use dedicated
% algorithm float types and packages that provide these will not provide
% correct IEEE style captions. The latest version and documentation of
% algorithmic.sty can be obtained at:
% http://www.ctan.org/pkg/algorithms
% Also of interest may be the (relatively newer and more customizable)
% algorithmicx.sty package by Szasz Janos:
% http://www.ctan.org/pkg/algorithmicx




% *** ALIGNMENT PACKAGES ***
%
%\usepackage{array}
% Frank Mittelbach's and David Carlisle's array.sty patches and improves
% the standard LaTeX2e array and tabular environments to provide better
% appearance and additional user controls. As the default LaTeX2e table
% generation code is lacking to the point of almost being broken with
% respect to the quality of the end results, all users are strongly
% advised to use an enhanced (at the very least that provided by array.sty)
% set of table tools. array.sty is already installed on most systems. The
% latest version and documentation can be obtained at:
% http://www.ctan.org/pkg/array


% IEEEtran contains the IEEEeqnarray family of commands that can be used to
% generate multiline equations as well as matrices, tables, etc., of high
% quality.




% *** SUBFIGURE PACKAGES ***
%\ifCLASSOPTIONcompsoc
%  \usepackage[caption=false,font=normalsize,labelfont=sf,textfont=sf]{subfig}
%\else
%  \usepackage[caption=false,font=footnotesize]{subfig}
%\fi
% subfig.sty, written by Steven Douglas Cochran, is the modern replacement
% for subfigure.sty, the latter of which is no longer maintained and is
% incompatible with some LaTeX packages including fixltx2e. However,
% subfig.sty requires and automatically loads Axel Sommerfeldt's caption.sty
% which will override IEEEtran.cls' handling of captions and this will result
% in non-IEEE style figure/table captions. To prevent this problem, be sure
% and invoke subfig.sty's "caption=false" package option (available since
% subfig.sty version 1.3, 2005/06/28) as this is will preserve IEEEtran.cls
% handling of captions.
% Note that the Computer Society format requires a larger sans serif font
% than the serif footnote size font used in traditional IEEE formatting
% and thus the need to invoke different subfig.sty package options depending
% on whether compsoc mode has been enabled.
%
% The latest version and documentation of subfig.sty can be obtained at:
% http://www.ctan.org/pkg/subfig




% *** FLOAT PACKAGES ***
%
%\usepackage{fixltx2e}
% fixltx2e, the successor to the earlier fix2col.sty, was written by
% Frank Mittelbach and David Carlisle. This package corrects a few problems
% in the LaTeX2e kernel, the most notable of which is that in current
% LaTeX2e releases, the ordering of single and double column floats is not
% guaranteed to be preserved. Thus, an unpatched LaTeX2e can allow a
% single column figure to be placed prior to an earlier double column
% figure.
% Be aware that LaTeX2e kernels dated 2015 and later have fixltx2e.sty's
% corrections already built into the system in which case a warning will
% be issued if an attempt is made to load fixltx2e.sty as it is no longer
% needed.
% The latest version and documentation can be found at:
% http://www.ctan.org/pkg/fixltx2e


%\usepackage{stfloats}
% stfloats.sty was written by Sigitas Tolusis. This package gives LaTeX2e
% the ability to do double column floats at the bottom of the page as well
% as the top. (e.g., "\begin{figure*}[!b]" is not normally possible in
% LaTeX2e). It also provides a command:
%\fnbelowfloat
% to enable the placement of footnotes below bottom floats (the standard
% LaTeX2e kernel puts them above bottom floats). This is an invasive package
% which rewrites many portions of the LaTeX2e float routines. It may not work
% with other packages that modify the LaTeX2e float routines. The latest
% version and documentation can be obtained at:
% http://www.ctan.org/pkg/stfloats
% Do not use the stfloats baselinefloat ability as the IEEE does not allow
% \baselineskip to stretch. Authors submitting work to the IEEE should note
% that the IEEE rarely uses double column equations and that authors should try
% to avoid such use. Do not be tempted to use the cuted.sty or midfloat.sty
% packages (also by Sigitas Tolusis) as the IEEE does not format its papers in
% such ways.
% Do not attempt to use stfloats with fixltx2e as they are incompatible.
% Instead, use Morten Hogholm'a dblfloatfix which combines the features
% of both fixltx2e and stfloats:
%
% \usepackage{dblfloatfix}
% The latest version can be found at:
% http://www.ctan.org/pkg/dblfloatfix




% *** PDF, URL AND HYPERLINK PACKAGES ***
%
%\usepackage{url}
% url.sty was written by Donald Arseneau. It provides better support for
% handling and breaking URLs. url.sty is already installed on most LaTeX
% systems. The latest version and documentation can be obtained at:
% http://www.ctan.org/pkg/url
% Basically, \url{my_url_here}.




% *** Do not adjust lengths that control margins, column widths, etc. ***
% *** Do not use packages that alter fonts (such as pslatex).         ***
% There should be no need to do such things with IEEEtran.cls V1.6 and later.
% (Unless specifically asked to do so by the journal or conference you plan
% to submit to, of course. )


% correct bad hyphenation here
\hyphenation{op-tical net-works semi-conduc-tor}


\begin{document}
%
% paper title
% Titles are generally capitalized except for words such as a, an, and, as,
% at, but, by, for, in, nor, of, on, or, the, to and up, which are usually
% not capitalized unless they are the first or last word of the title.
% Linebreaks \\ can be used within to get better formatting as desired.
% Do not put math or special symbols in the title.
\title{Supporting Crowd Sensing with A3Droid}


% author names and affiliations
% use a multiple column layout for up to three different
% affiliations
\author{\IEEEauthorblockN{L. Baresi, S. Guinea, and D. Mendon\c{c}a}
\IEEEauthorblockA{Politecnico di Milano\\
Dipartimento di Elettronica, Informazione e Bioingegneria\\
Piazza L. da Vinci, 32 -- 20133 Milano, Italy\\
\{luciano.baresi | sam.guinea | danilo.mendonca\}@polimi.it}
}

% conference papers do not typically use \thanks and this command
% is locked out in conference mode. If really needed, such as for
% the acknowledgment of grants, issue a \IEEEoverridecommandlockouts
% after \documentclass

% for over three affiliations, or if they all won't fit within the width
% of the page, use this alternative format:
% 
%\author{\IEEEauthorblockN{Michael Shell\IEEEauthorrefmark{1},
%Homer Simpson\IEEEauthorrefmark{2},
%James Kirk\IEEEauthorrefmark{3}, 
%Montgomery Scott\IEEEauthorrefmark{3} and
%Eldon Tyrell\IEEEauthorrefmark{4}}
%\IEEEauthorblockA{\IEEEauthorrefmark{1}School of Electrical and Computer Engineering\\
%Georgia Institute of Technology,
%Atlanta, Georgia 30332--0250\\ Email: see http://www.michaelshell.org/contact.html}
%\IEEEauthorblockA{\IEEEauthorrefmark{2}Twentieth Century Fox, Springfield, USA\\
%Email: homer@thesimpsons.com}
%\IEEEauthorblockA{\IEEEauthorrefmark{3}Starfleet Academy, San Francisco, California 96678-2391\\
%Telephone: (800) 555--1212, Fax: (888) 555--1212}
%\IEEEauthorblockA{\IEEEauthorrefmark{4}Tyrell Inc., 123 Replicant Street, Los Angeles, California 90210--4321}}




% use for special paper notices
%\IEEEspecialpapernotice{(Invited Paper)}




% make the title area
\maketitle

% As a general rule, do not put math, special symbols or citations
% in the abstract
\begin{abstract}
The large scale adoption of pervasive and mobile computing, ranging from smartphones and other gadgets to autonomous vehicles, constitutes a scenario in which data is been produced and consumed at an unprecedented rate by cyber-physical devices and users at the edge of the network. Targeting this scenario, the 5th generation mobile networks and wireless systems (5G) shall bring disruptive developments to telecommunication services, including context-awareness and device-to-device (D2D) communication. This combination increases the potential for new types of applications, but also poses challenges to their engineering. %, as low latency, cost, volatility and heterogeneity of devices still need to be harassed.

While cloud has been and will remain as an essential model for distributed computing, problems such as latency, cost and intermittent connectivity still need to be harassed. In this paper, we introduce the concept of \textit{Ad Hoc Edge Spaces} as part of a framework for the development of mobile applications that cannot depend solely on remote servers to satisfy their communication, processing and coordination requirements. Through edge spaces, devices can form ad hoc groups to interact locally using different communication and coordination styles and to dynamically assume application roles -- like fetching and processing data from their sensors -- and autonomously managing the ad hoc spaces itself -- like reorganizing the group-role structure and keeping group knowledge. We illustrate this idea with use cases that exploit its features and evaluate our framework with the implementation of a crowdsensing application for Android platform.

%The purpose is to avoid bottlenecks, costs and the latency of cloud servers.

%This scenario has specific characteristics, such as a high heterogeneity and volatility, intermittent connectivity, and resource limitations. To further explore its potential, pervasive and mobile applications must be able to overcome challenges such as the churn of components and fluctuations of physical and computational resources. 

%Moreover, considering the nature of personal devices, their heterogeneity, and scale, applications must make proper use of resources with an efficient and coherent distribution of its activities among participant devices.
%As more and more devices join the network or because of low latency requirements, a server-base architecture in which servers in the cloud handle most of the computation may not deliver the expected results. Conversely, 
%Many devices are now equipped with more powerful resources. This enables more computation to be performed at the edge of the network infrastructure. Some applications already explore machine-to-machine (M2M) communication to allow distributed components to interact and coordinate their behavior through local and mobile ad hoc networks (MANETs). Despite the potential benefits, an horizontal architectural must also deal with problems related to the churn of devices, consistency, robustness, availability, resource limitations, fairness, dynamic policies, security, and others.  


%In the literature, data-centric models are the state-of-art for distributed components coordination, as they enable both space and time decoupling. However, their high abstraction leaves many aspects open. In this work, we propose a framework for the development of distributed and self-adaptive applications willing to explore the potential of nowadays pervasive computing. Our model inherits well-known organizational abstractions, namely groups and roles, which allow a flexible modularization of the system and a dynamic assignment of activities among heterogeneous hosting devices in scenarios of high volatility. The abstractions and features provided by the framework target the application itself (managed system) and also the layer responsible for its adaptation (managing system) -- which should preserve the normal behavior and avoid violations of the application requirement in the advent of context changes.

%We have showed the feasibility of this approach with a real-case application for public transport monitoring and compared its features with other models and mechanisms for self-adaptation.

%In addition, we detail the self-organization mechanisms and their incorporation by a middleware.  Finally, we have showed the use of our model in different application scenarios.
\end{abstract}

% no keywords
%!TEX root = main.tex
% -*- root: main.tex -*-
\section{Introduction}
\label{sec:intro}
% target -> 1 page (with abstract)

The widespread diffusion of powerful mobile devices, such as smartphones and always-on wireless connectivity solutions as 3G and LTE, have radically changed the way we live our lives. Media sharing and consumption through social Internet have never been more engaging.

In this context, innovative scenarios are emerging. For example, pervasive and context-aware applications are aware of the end-users surroundings and behave accordingly to what is happening therein. To achieve this they rely on data that are obtained from the runtime environment itself, generally collected through appropriately deployed sensor networks. 

Since their inception, modern mobile devices have included a plethora of onboard sensors to enable applications such as navigation, gaming and health monitoring. This means that mobile devices can enable the construction of new kinds of sensor networks that can be deployed in unconventional ways to locations and contexts that were previously inconceivable. Moreover, the data that we collect through these networks are not limited by the onboard sensors; instead they could be data that are explicitly inputed into an app by the end-users. 

This novel data collection paradigm goes under the name of crowd sensing. Through crowd sensing we can gather invaluable information in disaster-recovery scenarios, we can perform geo-localized and time-limited sentiment analyses, and so on.

Crowd sensing introduces novel challenges. First of all, since the sensors are embed in mobile devices, they are always on the move and subject to high churn rates. Indeed, it is common for people to enter and leave the sensing application freely and unexpectedly. This can lead to robustness issues in which the required data cannot be collected at all times due to the absence of a certain type of sensor. Second, the precision and quality of the collected data is vital. A single source of information can sometimes be misleading; indeed, the data it provides could be imprecise, or just simply incorrect. Multiple sources, on the other hand, allow us to implement various kinds of redundancy:

\begin{itemize}

\item \emph{Temporal redundancy}, in which multiple sensors, in a limited amount of time, are providing similar information  (e.g., that the temperature is rising). 

\item \emph{Spatial redundancy}, in which mulitple sensors, in a limited amount of physical space, are providing similar information (e.g., the the temperature is rising in a certain building). 

\item \emph{Dimensional redundancy}, in which multiple sensors, of different kinds, are providing valuable and relevant information (e.g., the temperature is rising as is the level of $CO_{2}$). 

\end{itemize}

Finally and also important, applications based in crowdsensing will need to scale to support extensive sensor networks due to the high amounts of physical devices that may be available for the application at any given time.

In this paper, we advocate that there is a concrete need for a middleware for developing crowd sensing applications, one that can help developers cope with the more complex issues of crowd sensing applications without having to re-invent the wheel every time. To this end we provide A3Droid, a re-interpretation of the A-3 architectural style that specifically focuses on the needs of crowd sensing applications. 

A3Droid is based on the notion of \emph{group}. With A3Droid developers can group devices in the sensor network to reduce the dynamicity degree of the application components. Indeed, while single devices can enter and leave the application unexpectedly, entire groups of devices are less likely to do so. Groups also allow us to more easily implement temporal, spatial and dimensional sensor redundancy, as well as provide a concrete means to collect up-to-date snapshots of what devices are in the application at any given time. Finally, the groups in A3Droid can be composed and de-composed on the fly to ensure that the application scales according to the churn rate. 

The rest of the paper is structured as follows. Section~\ref{sec:a3} provides a rapid crash course on the A-3 architectural style. Therein we present the main abstractions that the architectural style provides to developers. Section~\ref{sec:a3droid} discusses in-depth details on how the A-3 architectural style was implemented for Android-based mobile devices in the form of the A3Droid middleware. Section~\ref{sec:evaluation} illustrates the experiments that were performed to assess A3Droid as an effective and efficient middleware for developing crowd sensing applications. Section~\ref{sec:related} discusses related work in literature, while Section~\ref{sec:conclusion} concludes the paper and presents insights into what we intend to do in our future work in the area.












%!TEX root = main.tex
% -*- root: main.tex -*-
\section{A-3 in a Nutshell}
\label{sec:a3}

A-3 is an innovative architectural style, and accompanying middleware, that facilitates the design and implementation of self-organizing distributed systems that are able to guarantee certain functional and non-functional qualities, even in the wake of high levels of component churn.  

In the A-3 style components are dynamically gathered into \emph{groups} based on application-specific criteria (e.g., a shared objective or the fact that the components therein have similar characteristics). The assumption is that, by creating groups, the overall application becomes less subject to high component churn rates. Indeed, while components make come and go, groups are more likely to remain stable. 

Each group in A-3 is composed of a single \emph{supervisor} component, and multiple \emph{follower} components. Supervisors are responsible for guiding the behaviour of its followers, as they attempt to reach their goals. 

Communication inside a group is achieved using asynchronous messagging; however, only three kinds of message exchange patterns are allowed. \emph{Supervisor-to-follower unicast} allows a supervisor to send a message to a single follower; \emph{supervisor-to-follower} multicast allows a supervisor to send a message to a subset (or all) of its followers; \emph{follower-to-supervisor unicast} allows a follower to send a message to its group supervisor. Follower-to-follower communication is not allowed, and if needed must be perfromed using a different channel. 

Groups are not confined to isolated existence; on the contrary they can collaborate through what is called \emph{group composition}. Group composition is enabled by allowing components to belong to more than one group at a time. Figure~\ref{fig:configurations} illustrates the three basic ways in which two groups can collaborate through composition. Each group is identified by a horizontal bar. On the top side of the bar we have the group's supervisor (in light gray); while on the bottom half of the bar we have the group's followers (in white).

\begin{figure}[ht]
\centering
\includegraphics[width=0.5\textwidth]{figures/configurations}
\caption{Possible variations of group composition.}
\label{fig:configurations}
\end{figure}

\noindent In case (a) we have a component that is a follower in two different groups. This means it is contributing its local knowledge to both groups. In case (b) we have a component that is a supervisor in one group, and a follower in another. In this case the component is contributing the knowledge it gathers as the supervisor of the first group within the second group, in which it is a follower. Case (c) is an extension of case (b); in this case a bidirectional communication is established between the two groups. 

\subsection{Self-Organizing Capabilities}
\label{sub:self-organizing}

In A-3 a group cannot exit in the absence of a supervisor. Fortunately, A-3 offers three built-in features that allow a group to autonomously re-organize itself should the supervisor leave for any reason.

First of all, A-3 offers \emph{Group State Management} (GSM). This feature allows supervisors to store data within the group itself. The data is replicated and synchronized across the group's follower components, so that it can easily be recovered when needed. 

Second, A-3 offers \emph{Runtime Membership Updates} (RMU). This feature uses group messaging to provide supervisors with up-to-date snapshots of ``who'' is in the group, since components are free to enter and/or leave the group at will. It also plays a key role in signalling to the followers that their supervisor has left the group. 

Third, A-3 offers \emph{Supervisor Failure Recovery} (SFR). This feature allows developers to define the distributed leader-election algorithm that they want the A-3 middleware to launch when a new group supervisor needs to be found. 

\subsection{Group Topology Management}
\label{sub:topologyManagement}

A-3 also provides supervisor components with a series of \emph{Topology Control Operations} (TCOs) that they can use to change the application's topology, i.e., how the groups are composed. These can be used to cope with a sudden increase (or decrease) in the number of components within a single group. A-3 provides four TCOs: \emph{split} and \emph{merge}, and \emph{stack} and \emph{unstack}. 

Split takes a subset of the superisor's follower components and places them in a newly created ``secondary'' group. This requires the availability of a component that can act as the new group's supervisor. One can specify the number of components that need to move, and in this case the components are chosen randomly; or one can specify a \emph{membership function}. In this case the membership function is called on all the components, and those that return the boolean value \emph{true} are moved to the new group. Merge is the inverse of split; it takes the components of a \emph{source} group and moves them to a \emph{destination} group. As a result, the source group ceases to exist.

Stack is used to stack one group on top of another, by making the second group's supervisor become a follower in the first group. Unstack is the inverse of this operation. 





%target -> 0.75 pages


%!TEX root = main.tex
% -*- root: main.tex -*-
\section{A3Droid}
\label{sec:a3droid}

The main contribution of this paper is an entirely new implementation of the A-3 middleware for Android smatphones that targets the development of crowd assisted sensing. 

Our new implementation of A-3 is called \emph{A3Droid} and it is built on top of AllJoyn~\cite{AllJoyn}, an open-source project developed by the Allseen Alliance for interconnecting heteregoneous devices, typically in a smart-living scenario.  

\subsection{A3Droid Nodes}
\label{subg:a3droidNodes}

\begin{figure*}[t!]
\centering
\includegraphics[width=\textwidth]{figures/implementation}
\caption{(a) A3Droid Implementation. (b) A3DroidNode roles.}
\label{fig:implementation}
\end{figure*}

In A3Droid each participant in an application must implement what is called an \texttt{A3DroidNode}. Within the A3DroidNode the developer will specify what groups' existence the node is aware of, and to what extent that node can participate in each group, i.e., if it can participate as a supervisor, as a follower, or as both. 

Since a node can participate in more than one group at a time, in A3Droid we say that it can play multiple \emph{roles}. Technically, a role is a Java class the extends either \texttt{A3DroidAbstractSupervisorRole} or \texttt{A3DroidAbstractFollowerRole}. By extending these classes the developer defines the behaviours that the nodes will need to execute within a specific group. Indeed, the developer must provide one extension of \texttt{A3DroidAbstractSupervisorRole} and/or one extension of \texttt{A3DroidAbstractFollowerRole} for each of groups he wants the node to participate in (see Figure~\ref{fig:implementation}(a)). 

When extending class \texttt{A3DroidAbstractFollowerRole} for a given group, the developer must provide a Java Runnable that implements the following two abstract methods. Method \texttt{run} should contain the node's application-specific follower behaviour for that specific group. \texttt{messageFromSupervisor} should contain the logic that needs to be run when a message is received from the node's supervisor. When extending class \texttt{A3DroidAbstractSupervisorRole} the developer must provide a Java Runnable that implements the following three abstract methods. Method \texttt{run} should contain the node's application-specific supervisor behaviour for that specific group. \texttt{messageFromFollower} should contain the logic that is run when the node receives a message from a follower. \texttt{updateFromGroup} is the logic that should be run every time there is a change in the group's membership. This method is stimulated autonomously by the A3Droid middleware. \texttt{fitnessFunc} represents the application-specific fitness function that is used by A3Droid when it is looking for a new supervisor for the group when performing Supervisor Failure Recovery. Keep in mind that within methods \texttt{run}, \texttt{messageFromFollower}, and \texttt{updateFromGroup} the supervisor can exploit the Topology Control Operations split/merge and stack/unstack discussed in Section~\ref{sub:topologyManagement}. 

Figure~\ref{fig:implementation}(b) illustrates the internal architecture of the A3DroidNode. Each node contains a unique ID, as well as four supporting data structures. \texttt{GroupInfos} keeps track of all the groups that the node is aware of; \texttt{SupervisorRoles} and \texttt{FollowerRoles} keep track of the behaviours that the node is aware of and capable of performing; \texttt{ActiveRoles} keeps track of the actual roles that are being performed at any given time. Each node also contains an \texttt{inNodeSharedMemory} data structure which is used for A-3's \emph{Group State Management}.

\subsection{AllJoyn-based Implementation}
\label{subg:a3droidNodes}

At the center of AllJoyn lies a \emph{virtual bus}, to which devices can connect. The bus distinguished between two kinds of nodes. \emph{Routing nodes} connect directly to the bus, and are responsible for directing messages through the network; \emph{leaf nodes} are nodes that can only connect to routing nodes. As a result, the virtual bus has the structure of a mesh of stars.

To participate in an AllJoyn application one must create a \emph{BusAttachment}, through this attachment publish a \emph{BusObject}, and proceed to ``discover'' the other participants that are connected to the bus. This is achieved through well-known names that are statically provided (e.g., app identifiers). Once the discovery has been completed one can create a \emph{session} and start interacting. A session is either a 1--to--1 or a 1--to--n grouping of devices. An interaction can consist in the invocation of a \emph{BusMethod}, i.e., a 1--to--1 synchronous exchange between two participants; or in the production of a \emph{BusSignal}, i.e., an asynchronous broadcast within the context of the session itself. BusSignals require appropriate \emph{BusListeners} to be registered on the bus.

In A3Droid when a new node wants to connect to a group it uses AllJoyn's discovery mechanism with the group's unique and well-known name. Three things can happen. Frist, if no results are returned, and the node can play the role of the supervisor in that group, the node proceeds to create a special purpose AllJoyn BusObject called the \texttt{Group Manager}. In A3Droid there is always one group manager per group. It is responsible for keeping the group alive by establishing its well-known name on the bus, and for keeping track of group membership. A second BusObject is then created, containing the actual Supervisor Role logic. Second, if no results are returned, and the node cannot play the role of the supervisor, the node gets added to a supporting \emph{wait} group. If this wait group does not exist it is dynamically created, and the node is added as its supervisor. (All A3Droid nodes are capable of being the wait group's supervisor.) Third, if a valid result is returned, and the node can play the role of the group's follower, it is added to the group's ongoing AllJoyn session. To do this it publishes its actual Follower logic onto the bus. 

When the Group Manager and the Supervisor Bus Objects are created they originate on the same physical device. However, this can change over time. The failure of a supervisor does not necessarily imply the failure of the group's Group Manager. In fact, if a supervisor fails but the same is not true for the group manager, when the new supervisor is selected through the leader-election algorithm, the group manager and the new supervisor can effectively come to be on different physical devices. 

When a valid group has been established communication becomes possible. For supervisor--to--follower multicast we use a combination of AllJoyn BusSignals and BusListeners. For supervisor--to--follower unicast communications we use AllJoyn BusMethods, and temporary 1--to--1 sessions are established. In this case the follower needs to publish a \texttt{UnicastReceiver} BusObject onto the bus. Finally, athough follower--to--supervisor unicast messaging could have been implemented in the same way, we decided to use AllJoyn's broadcasting mechanism; the reason for this is that it is inherently more performant than establishing temporary 1--to--1 sessions. To do this each follower is provided with an event emitter, and the supervisor is provided with an appropriate listener.

\begin{figure*}[t!]
\centering
\includegraphics[width=0.8\textwidth]{figures/a3droid}
\caption{A3Droid uses AllJoyn for node communication.}
\label{fig:communication}
\end{figure*}

Figure~\ref{fig:communication} shows an example in which we have a group of three devices called \emph{Group A}. \emph{Device 1} is the first to attempt to connect to group A, and as a result it instantiates both a \emph{Group Manager} BusObject and a \emph{Supervisor} BusObject. \emph{Device 1} and \emph{Device 2} join the group at a later moment. When they do they both instantiate a \emph{Follower} BusObject and connect it to \emph{Session Group A}, which already contains the group's \emph{Group Manager} and the \emph{Supervisor} BusObjects, so that they can start exchanging messages correctly. The two devices also instantiate one \emph{Unicast Receiver} BusObject each, to allow supervisor--to--follower unicast messaging to take place. \emph{Session Device 2} is created for exchanging messages with Device 2.













%target -> 1.75 pages


\section{Evaluation and Discussion}\label{sec:evaluation}



The simulation experiments aimed at showing the benefits of the approach and measuring the overhead imposed by the self-organization mechanisms.
%, namely self-grouping and distributed role allocation. 
As these methods add no significant overhead in terms of processing or memory, an asymptotic analysis focused on the communication overhead.
%, as the exchange of messages through wireless mediums consumes battery from devices and is subject to delays that may interfere with the application behavior. 
As for the benefits of the approach, we executed simulation experiments of a public transport monitoring application using a pure client-server and using our framework. The goal was to compare, in each case, the following two metrics:

\begin{enumerate}[label=\textbf{M}\arabic*:]
	
	\item the total number of sensing tasks performed and requisitions fired from clients to backend servers (evaluates battery consumption and Internet traffic); and
	
	\item the total number of failures in reaching backend servers due to intermittent connectivity (evaluates robustness).
	
\end{enumerate}

%In specific, this metric was evaluated 

%\begin{enumerate}[label=\Alph*]
%	
%	\item The asymptotic overhead, given by worse case number of exchanged messages required by each self-organization mechanism, as a function of the number of application nodes, groups and roles (Section~\ref{fig:asymmetry}). 
%	
%	\item The measured overhead, given by the number of exchanged messages counted during simulated executions of a MCS application for public bus monitoring, in which the number of group/roles is fixed and the number of nodes, as well as their capabilities, varies according to probabilistic distributions.
%	
%\end{enumerate}


\subsection{Asymptotic Analysis} 

\subsubsection{\textbf{Self-grouping}} the complexity analysis was divided in two parts: a) the overhead when a node joins/leaves a group (registry update); b) the additional overhead when a node joins a group (registry copy).

In the worse case scenario, registry update (a) takes $n-1$ unicast messages ($O(n)$), with $n$ the group size, and a single registry line as payload. However, if a broadcast communication is used, a single broadcast message (e.g., an UDP broadcast over Wi-Fi) can advertise the registry update ($O(1)$).
%This is the case, e.g., of an UDP broadcast over the 802.11 network protocol (Wi-Fi). 
The registry copy (b), in its turn, requires a single unicast message to be transmitted each time a node joins a group ($O(1)$). Important to mention, in contrast with the registry update message, the registry copy includes information about all $n-1$ nodes in the group. 

\subsubsection{\textbf{Discentralized Role Allocation}} 

%TODO: check the definition of FS and use it accross the paper

this mechanism involves the exchange of fitness scores ($FS$) among eligible nodes in the advent of the events depicted in the previous subsection. The actual number of $FS$ messages sent/received during a role position election depends on how many nodes in the $n$ size group satisfy the role restrictive criteria (RRC) (see Eq.~\ref{eq:rrc} in Section~\ref{sec:edge_spaces}). Once more, the type of network is a determining factor.

In the worse case, represented by an election of a vacant position (following a \textit{vacancy} or \textit{resignation} event) and without broadcast, each node $e$ in the set of eligible nodes $E$ must send each other an unicast message. If $|E| = n$, the message count is given by $O(n * (n-1)) = O(n^2)$. If the nodes in $E$ can communicate through broadcast, this number is reduced to $O(n)$. In its turn, the challenge event produces smaller overhead as the exchange of messages is restricted to a request-response between challenger and challenged and to the subsequent advertisement of the result ($O(n)$ without broadcast, otherwise $(O(1))$). 

%added to the node's classification list in polynomial time ($O(p*log(p)))$, with $1 \le p \le n$. 

Once the communication overhead of a single election round is known, the overall overhead can be estimated by the number of groups and roles in the application and the frequency of events triggering new elections. Whereas the vacancy and resignation events should be handled with a new election, the frequency in which a challenge event happens is greatly affected by the choice for $\delta$: the higher this factor is, the lower the probability of a challenge (and the need for the elected nodes to update their peers about changes in their $FS$). Therefore, the decision of which $\delta$ to be used depends on the criticality of the attributes that compose the fitness score of a node. 



\subsection{Public Transport Monitoring Simulation}

\subsubsection{Experiments Design}

the group-role specification (List.~\ref{lst:bm_criteria} in Section~\ref{sec:edge_spaces}) was used for the modeling of a dynamic bus monitoring application scenario. In it, the following variables were considered:

\begin{itemize}
	
	\item the number of nodes within a bus: \textit{from 2 to 10 nodes};
	
	\item the battery level of each node: \textit{from 10\% to 90\%};
	
	\item the Internet type of each node: \textit{cellular or Wi-Fi};
	
	\item the GPS signal in each node: \textit{from 0\% to 100\%};
	
\end{itemize}

In the client-server approach, each node tries to collect data from its accelerometer and GPS before sending it to the server through whatever type of Internet connection available (Wi-Fi or cellular data plan). In our approach, at most two simultaneous nodes are assigned to each type of sensor and one node to aggregate data before sending a preprocessed batch to the server. The number of times each sensor is fetched, as well as the number of times a connection between a client/aggregator and the server, were counted in separate (\textbf{M1}). Finally, we also counted the monitoring windows in which no measurement reached the server (\textbf{M2}).

%We modeled the cost of sampling the GPS as 5 power units (pw) and 1pw for the accelerometer, as in most cases it works continuously, but still requires a background activity running. Plus, we modeled the cost of sending data to the server through cellular plan as 10, through Wi-Fi as 5, and through 

The experiments were performed with, \textit{PeerSim}, an open source peer-to-peer simulator~\cite{p2p09-peersim}. This tool supports the creations of different network P2P topologies and provides a handful JAVA programming interface.



%To simulate a public bus monitoring crowd-sensing application, experiments were performed with an increasing number of nodes. Each role can potentially play a \textit{gps-monitor}, a \textit{acceleration-monitor}, or an \textit{aggregator} role. Each role has been implemented as a protocol extending the \textit{CDProtocol} class. At each simulation cycle, 
%
%the application roles of a \textit{gps-monitor} and \textit{acceleration-monitor} were implemented as protocols, which are called at each simulation cycle. 
%
%The simulation was executed in a computer with .... 

\subsection{Results}

Add Figures here

\subsection{Discussion}

Discuss the obtained results
%!TEX root = main.tex
% -*- root: main.tex -*-
\section{Related Work}
\label{sec:related}

%target -> 0.5 pages

A few related works also provide support for the development of crowdsensing applications. Mendes et al.~\cite{Mendes2015} proposed Maestro, a cooperative sensing framework based on a middleware that provides transparent access to data from heterogeneous devices though virtual sensors. The framework includes features related to data storage, cloud-based data analysis, and networking. Maestro focuses on sensors over large areas using a central server or a routing protocol, with data saved to local databases, to be sent when connectivity is available. A3Droid, on the other hand, focuses on providing a distributed fog-like architecture that exploits local network communication within and between groups. External communication is still allowed and can be achieved through application-specific means; it simply is not supported directly by the framework. Hu et al.~\cite{Hu:2014} proposed a social networking architecture for an ecosystem that enables social context awareness in mobile crowdsensing. Their human-centric approach aggregates context-related sensing data to solve specific problems and to achieve context awareness in mobile applications. In contrast to A3Droid, sensor data is only part of a broader information range that includes users input from social networks. It also depends on central servers and does not support local communication. 

Katsomallos et al. proposed EasyHarvest~\cite{Katsomallos:2014}, a framework for large-scale mobile crowdsensing applications aiming to simplify its deployment and to enable controlled use of device sensors. Its architecture is composed of two components: a server side component, where crowdsensing applications are deployed by providers, and a client side component, to which sensing tasks are assigned. Also designed as a generic platform for mobile crowdsensing, MOSDEN~\cite{Jayaraman:2013} aims to support sensor data collection, processing, storage and sharing, which are performed separately from application logic. Similarly, other works provide a generic platform in which sensing tasks are assigned to participating users, and their devices serve other users and their applications~\cite{Hu:2014, Bajaj:2015}. On the one hand, these are valuable proposals that separate data collection from its use to increase the odds of finding available sensors matching specific requirements. On the other, they rely on a more complex architecture and infrastructure that often requires a cloud component. Instead, A3Droid provides a programming framework that supports a wide variety of architectures; it can be used to develop both purely distributed and fog-like systems, where distributed sensing and data manipulation is paired with centralized data analysis and distribution.

%follows a simple, yet powerful architecture for the development of mobile crowdsensing applications.

%1 

%2 More human-centric approach that aggregates context-related sensing data to solve specific problems and to achieve context awareness of mobile applications with respect to social aspects. Sensors are only part of a broader information range that includes users input from social networks. 

%3 

%In contrast with those works, A3Droid focuses on both spacial and temporal (read the paper after Sam modifications, then write the differences)

% is based in A-3 architectural style that enables the oportunistic gathering of data from a group of sensors in a local area - instead of larger or disconnected areas. Accordingly, it is designed for online crowdsensing within a local area network without internet dependency. This characteristic enables temporal and spatial redundancies as local area .... It's main novelty consists of groups abstraction, in which groups and roles, not specific devices, are addressed in order to mitigate the effects of the dynamicity and churn.

%Moreover, it offers a framework to ease the burden of developers when applying its architectural style, which includes lower level networking involved in supervisor-follower communication.

%main difference: how A3 abstracts sensors. instead of 'virtual nodes', agnostic group members.
%fog is a big difference: instead of sensors connected in a wide area by usual internet, groups of sensors connected by a local area network
%also, how sensing is performed in time
%!TEX root = main.tex
% -*- root: main.tex -*-
\section{Conclusion and Future Work}\label{sec:conclusion}


% For peer review papers, you can put extra information on the cover
% page as needed:
% \ifCLASSOPTIONpeerreview
% \begin{center} \bfseries EDICS Category: 3-BBND \end{center}
% \fi
%
% For peerreview papers, this IEEEtran command inserts a page break and
% creates the second title. It will be ignored for other modes.
\IEEEpeerreviewmaketitle



%TODO uncomment after at least one item is present
%\begin{thebibliography}{1}

%\end{thebibliography}




% that's all folks
\end{document}


