%% bare_conf.tex
%% V1.4b
%% 2015/08/26
%% by Michael Shell
%% See:
%% http://www.michaelshell.org/
%% for current contact information.
%%
%% This is a skeleton file demonstrating the use of IEEEtran.cls
%% (requires IEEEtran.cls version 1.8b or later) with an IEEE
%% conference paper.
%%
%% Support sites:
%% http://www.michaelshell.org/tex/ieeetran/
%% http://www.ctan.org/pkg/ieeetran
%% and
%% http://www.ieee.org/

%%*************************************************************************
%% Legal Notice:
%% This code is offered as-is without any warranty either expressed or
%% implied; without even the implied warranty of MERCHANTABILITY or
%% FITNESS FOR A PARTICULAR PURPOSE! 
%% User assumes all risk.
%% In no event shall the IEEE or any contributor to this code be liable for
%% any damages or losses, including, but not limited to, incidental,
%% consequential, or any other damages, resulting from the use or misuse
%% of any information contained here.
%%
%% All comments are the opinions of their respective authors and are not
%% necessarily endorsed by the IEEE.
%%
%% This work is distributed under the LaTeX Project Public License (LPPL)
%% ( http://www.latex-project.org/ ) version 1.3, and may be freely used,
%% distributed and modified. A copy of the LPPL, version 1.3, is included
%% in the base LaTeX documentation of all distributions of LaTeX released
%% 2003/12/01 or later.
%% Retain all contribution notices and credits.
%% ** Modified files should be clearly indicated as such, including  **
%% ** renaming them and changing author support contact information. **
%%*************************************************************************


% *** Authors should verify (and, if needed, correct) their LaTeX system  ***
% *** with the testflow diagnostic prior to trusting their LaTeX platform ***
% *** with production work. The IEEE's font choices and paper sizes can   ***
% *** trigger bugs that do not appear when using other class files.       ***                          ***
% The testflow support page is at:
% http://www.michaelshell.org/tex/testflow/



%\documentclass[10pt, conference, compsocconf]{IEEEtran}
\documentclass[twocolumn,twoside]{IEEEtran}
% Some Computer Society conferences also require the compsoc mode option,
% but others use the standard conference format.
%
% If IEEEtran.cls has not been installed into the LaTeX system files,
% manually specify the path to it like:
% \documentclass[conference]{../sty/IEEEtran}



% Some very useful LaTeX packages include:
% (uncomment the ones you want to load)


% *** MISC UTILITY PACKAGES ***
%
%\usepackage{ifpdf}
% Heiko Oberdiek's ifpdf.sty is very useful if you need conditional
% compilation based on whether the output is pdf or dvi.
% usage:
% \ifpdf
%   % pdf code
% \else
%   % dvi code
% \fi
% The latest version of ifpdf.sty can be obtained from:
% http://www.ctan.org/pkg/ifpdf
% Also, note that IEEEtran.cls V1.7 and later provides a builtin
% \ifCLASSINFOpdf conditional that works the same way.
% When switching from latex to pdflatex and vice-versa, the compiler may
% have to be run twice to clear warning/error messages.


\usepackage{subcaption}
\captionsetup[subfigure]{width=.9\textwidth}
\usepackage{enumitem}
\usepackage{color}
\definecolor{lightgrey}{gray}{0.98}
\definecolor{blue}{rgb}{0, 0, 0.25}

% *** CITATION PACKAGES ***
%
%\usepackage{cite}
% cite.sty was written by Donald Arseneau
% V1.6 and later of IEEEtran pre-defines the format of the cite.sty package
% \cite{} output to follow that of the IEEE. Loading the cite package will
% result in citation numbers being automatically sorted and properly
% "compressed/ranged". e.g., [1], [9], [2], [7], [5], [6] without using
% cite.sty will become [1], [2], [5]--[7], [9] using cite.sty. cite.sty's
% \cite will automatically add leading space, if needed. Use cite.sty's
% noadjust option (cite.sty V3.8 and later) if you want to turn this off
% such as if a citation ever needs to be enclosed in parenthesis.
% cite.sty is already installed on most LaTeX systems. Be sure and use
% version 5.0 (2009-03-20) and later if using hyperref.sty.
% The latest version can be obtained at:
% http://www.ctan.org/pkg/cite
% The documentation is contained in the cite.sty file itself.






% *** GRAPHICS RELATED PACKAGES ***
%
\ifCLASSINFOpdf
  \usepackage[pdftex]{graphicx}
  % declare the path(s) where your graphic files are
  % \graphicspath{{../pdf/}{../jpeg/}}
  % and their extensions so you won't have to specify these with
  % every instance of \includegraphics
  % \DeclareGraphicsExtensions{.pdf,.jpeg,.png}
  %%Draw graphics
\usepackage{pgfplots}
\usepackage{pgfplotstable}
\pgfplotsset{compat=newest}
\pgfplotscreateplotcyclelist{my chart colors}{%
blue, solid, every mark/.append style={solid, fill=blue}, mark=*\\%
red, solid, every mark/.append style={solid, fill=red}, mark=square*\\%
brown, solid, every mark/.append style={solid, fill=brown}, mark=triangle*\\%
cyan, solid, every mark/.append style={solid, fill=gray}, mark=diamond*\\%
black, dashdotted, every mark/.append style={solid, fill=red}, mark=halfdiamond*\\%
olive, dashdotted, every mark/.append style={solid, fill=green}, mark=halfcircle*\\%
orange, dashdotted, every mark/.append style={solid, fill=gray}, mark=*\\%
purple, dashdotted, every mark/.append style={solid, fill=blue}, mark=halfsquare left*\\%
teal, densely dashed, every mark/.append style={solid, fill=green}, mark=oplus*\\%
violet, densely dashed, every mark/.append style={solid, fill=gray}, mark=diamond*\\%
magenta, densely dashed, every mark/.append style={solid, fill=blue}, mark=triangle*\\%
lime, densely dashed, every mark/.append style={solid, fill=red}, mark=square*\\%
}

\newcommand{\errorband}[5][]{ % x column, y column, error column, optional argument for setting style of the area plot
\pgfplotstableread[col sep=comma, skip first n=2]{#2}\datatable
    % Lower bound (invisible plot)
    \addplot [draw=none, stack plots=y, forget plot] table [
        x={#3},
        y expr=\thisrow{#4}-\thisrow{#5}
    ] {\datatable};

    % Stack twice the error, draw as area plot
    \addplot [draw=none, fill=gray!40, stack plots=y, area legend, #1] table [
        x={#3},
        y expr=2*\thisrow{#5}
    ] {\datatable} \closedcycle;

    % Reset stack using invisible plot
    \addplot [forget plot, stack plots=y,draw=none] table [x={#3}, y expr=-(\thisrow{#4}+\thisrow{#5})] {\datatable};
}
  %\usepackage{flushend}
\else
  % or other class option (dvipsone, dvipdf, if not using dvips). graphicx
  % will default to the driver specified in the system graphics.cfg if no
  % driver is specified.
  % \usepackage[dvips]{graphicx}
  % declare the path(s) where your graphic files are
  % \graphicspath{{../eps/}}
  % and their extensions so you won't have to specify these with
  % every instance of \includegraphics
  % \DeclareGraphicsExtensions{.eps}
\fi
% graphicx was written by David Carlisle and Sebastian Rahtz. It is
% required if you want graphics, photos, etc. graphicx.sty is already
% installed on most LaTeX systems. The latest version and documentation
% can be obtained at: 
% http://www.ctan.org/pkg/graphicx
% Another good source of documentation is "Using Imported Graphics in
% LaTeX2e" by Keith Reckdahl which can be found at:
% http://www.ctan.org/pkg/epslatex
%
% latex, and pdflatex in dvi mode, support graphics in encapsulated
% postscript (.eps) format. pdflatex in pdf mode supports graphics
% in .pdf, .jpeg, .png and .mps (metapost) formats. Users should ensure
% that all non-photo figures use a vector format (.eps, .pdf, .mps) and
% not a bitmapped formats (.jpeg, .png). The IEEE frowns on bitmapped formats
% which can result in "jaggedy"/blurry rendering of lines and letters as
% well as large increases in file sizes.
%
% You can find documentation about the pdfTeX application at:
% http://www.tug.org/applications/pdftex





% *** MATH PACKAGES ***
%
\usepackage{amsmath}
% A popular package from the American Mathematical Society that provides
% many useful and powerful commands for dealing with mathematics.
%
% Note that the amsmath package sets \interdisplaylinepenalty to 10000
% thus preventing page breaks from occurring within multiline equations. Use:
%\interdisplaylinepenalty=2500
% after loading amsmath to restore such page breaks as IEEEtran.cls normally
% does. amsmath.sty is already installed on most LaTeX systems. The latest
% version and documentation can be obtained at:
% http://www.ctan.org/pkg/amsmath


\newtheorem{definition}{Definition} 


% *** SPECIALIZED LIST PACKAGES ***
%
%\usepackage{algorithmic}
% algorithmic.sty was written by Peter Williams and Rogerio Brito.
% This package provides an algorithmic environment fo describing algorithms.
% You can use the algorithmic environment in-text or within a figure
% environment to provide for a floating algorithm. Do NOT use the algorithm
% floating environment provided by algorithm.sty (by the same authors) or
% algorithm2e.sty (by Christophe Fiorio) as the IEEE does not use dedicated
% algorithm float types and packages that provide these will not provide
% correct IEEE style captions. The latest version and documentation of
% algorithmic.sty can be obtained at:
% http://www.ctan.org/pkg/algorithms
% Also of interest may be the (relatively newer and more customizable)
% algorithmicx.sty package by Szasz Janos:
% http://www.ctan.org/pkg/algorithmicx
%\usepackage{flushend}
\usepackage{listings}
%\usepackage{filecontents}


% *** ALIGNMENT PACKAGES ***
%
\usepackage{array}
\usepackage{tabularx} % Danilo added to adjust the width of the tables
% Frank Mittelbach's and David Carlisle's array.sty patches and improves
% the standard LaTeX2e array and tabular environments to provide better
% appearance and additional user controls. As the default LaTeX2e table
% generation code is lacking to the point of almost being broken with
% respect to the quality of the end results, all users are strongly
% advised to use an enhanced (at the very least that provided by array.sty)
% set of table tools. array.sty is already installed on most systems. The
% latest version and documentation can be obtained at:
% http://www.ctan.org/pkg/array


% IEEEtran contains the IEEEeqnarray family of commands that can be used to
% generate multiline equations as well as matrices, tables, etc., of high
% quality.




% *** SUBFIGURE PACKAGES ***
%\ifCLASSOPTIONcompsoc
%  \usepackage[caption=false,font=normalsize,labelfont=sf,textfont=sf]{subfig}
%\else
%  \usepackage[caption=false,font=footnotesize]{subfig}
%\fi
% subfig.sty, written by Steven Douglas Cochran, is the modern replacement
% for subfigure.sty, the latter of which is no longer maintained and is
% incompatible with some LaTeX packages including fixltx2e. However,
% subfig.sty requires and automatically loads Axel Sommerfeldt's caption.sty
% which will override IEEEtran.cls' handling of captions and this will result
% in non-IEEE style figure/table captions. To prevent this problem, be sure
% and invoke subfig.sty's "caption=false" package option (available since
% subfig.sty version 1.3, 2005/06/28) as this is will preserve IEEEtran.cls
% handling of captions.
% Note that the Computer Society format requires a larger sans serif font
% than the serif footnote size font used in traditional IEEE formatting
% and thus the need to invoke different subfig.sty package options depending
% on whether compsoc mode has been enabled.
%
% The latest version and documentation of subfig.sty can be obtained at:
% http://www.ctan.org/pkg/subfig




% *** FLOAT PACKAGES ***
%
%\usepackage{fixltx2e}
% fixltx2e, the successor to the earlier fix2col.sty, was written by
% Frank Mittelbach and David Carlisle. This package corrects a few problems
% in the LaTeX2e kernel, the most notable of which is that in current
% LaTeX2e releases, the ordering of single and double column floats is not
% guaranteed to be preserved. Thus, an unpatched LaTeX2e can allow a
% single column figure to be placed prior to an earlier double column
% figure.
% Be aware that LaTeX2e kernels dated 2015 and later have fixltx2e.sty's
% corrections already built into the system in which case a warning will
% be issued if an attempt is made to load fixltx2e.sty as it is no longer
% needed.
% The latest version and documentation can be found at:
% http://www.ctan.org/pkg/fixltx2e


%\usepackage{stfloats}
% stfloats.sty was written by Sigitas Tolusis. This package gives LaTeX2e
% the ability to do double column floats at the bottom of the page as well
% as the top. (e.g., "\begin{figure*}[!b]" is not normally possible in
% LaTeX2e). It also provides a command:
%\fnbelowfloat
% to enable the placement of footnotes below bottom floats (the standard
% LaTeX2e kernel puts them above bottom floats). This is an invasive package
% which rewrites many portions of the LaTeX2e float routines. It may not work
% with other packages that modify the LaTeX2e float routines. The latest
% version and documentation can be obtained at:
% http://www.ctan.org/pkg/stfloats
% Do not use the stfloats baselinefloat ability as the IEEE does not allow
% \baselineskip to stretch. Authors submitting work to the IEEE should note
% that the IEEE rarely uses double column equations and that authors should try
% to avoid such use. Do not be tempted to use the cuted.sty or midfloat.sty
% packages (also by Sigitas Tolusis) as the IEEE does not format its papers in
% such ways.
% Do not attempt to use stfloats with fixltx2e as they are incompatible.
% Instead, use Morten Hogholm'a dblfloatfix which combines the features
% of both fixltx2e and stfloats:
%
% \usepackage{dblfloatfix}
% The latest version can be found at:
% http://www.ctan.org/pkg/dblfloatfix




% *** PDF, URL AND HYPERLINK PACKAGES ***
%
\usepackage{url}
% url.sty was written by Donald Arseneau. It provides better support for
% handling and breaking URLs. url.sty is already installed on most LaTeX
% systems. The latest version and documentation can be obtained at:
% http://www.ctan.org/pkg/url
% Basically, \url{my_url_here}.




% *** Do not adjust lengths that control margins, column widths, etc. ***
% *** Do not use packages that alter fonts (such as pslatex).         ***
% There should be no need to do such things with IEEEtran.cls V1.6 and later.
% (Unless specifically asked to do so by the journal or conference you plan
% to submit to, of course. )


% correct bad hyphenation here
%\hyphenation{op-tical net-works semi-conduc-tor}


\begin{document}
%
% paper title
% Titles are generally capitalized except for words such as a, an, and, as,
% at, but, by, for, in, nor, of, on, or, the, to and up, which are usually
% not capitalized unless they are the first or last word of the title.
% Linebreaks \\ can be used within to get better formatting as desired.
% Do not put math or special symbols in the title.

\title{Engineering Pervasive Applications with Self-organizing Ecosystems}


% author names and affiliations
% use a multiple column layout for up to three different
% affiliations
\author{\IEEEauthorblockN{L. Baresi and D. F. Mendon\c{c}a}
\IEEEauthorblockA{Politecnico di Milano\\
Dipartimento di Elettronica, Informazione e Bioingegneria\\
Piazza L. da Vinci, 32 -- 20133 Milano, Italy\\
\{luciano.baresi|danilo.filgueira\}@polimi.it}
}

% conference papers do not typically use \thanks and this command
% is locked out in conference mode. If really needed, such as for
% the acknowledgment of grants, issue a \IEEEoverridecommandlockouts
% after \documentclass

% for over three affiliations, or if they all won't fit within the width
% of the page, use this alternative format:
% 
%\author{\IEEEauthorblockN{Michael Shell\IEEEauthorrefmark{1},
%Homer Simpson\IEEEauthorrefmark{2},
%James Kirk\IEEEauthorrefmark{3}, 
%Montgomery Scott\IEEEauthorrefmark{3} and
%Eldon Tyrell\IEEEauthorrefmark{4}}
%\IEEEauthorblockA{\IEEEauthorrefmark{1}School of Electrical and Computer Engineering\\
%Georgia Institute of Technology,
%Atlanta, Georgia 30332--0250\\ Email: see http://www.michaelshell.org/contact.html}
%\IEEEauthorblockA{\IEEEauthorrefmark{2}Twentieth Century Fox, Springfield, USA\\
%Email: homer@thesimpsons.com}
%\IEEEauthorblockA{\IEEEauthorrefmark{3}Starfleet Academy, San Francisco, California 96678-2391\\
%Telephone: (800) 555--1212, Fax: (888) 555--1212}
%\IEEEauthorblockA{\IEEEauthorrefmark{4}Tyrell Inc., 123 Replicant Street, Los Angeles, California 90210--4321}}




% use for special paper notices
%\IEEEspecialpapernotice{(Invited Paper)}

% make the title area
\maketitle

% As a general rule, do not put math, special symbols or citations
% in the abstract
\begin{abstract}
The large scale adoption of pervasive and mobile computing, ranging from smartphones and other gadgets to autonomous vehicles, constitutes a scenario in which data is been produced and consumed at an unprecedented rate by cyber-physical devices and users at the edge of the network. Targeting this scenario, the 5th generation mobile networks and wireless systems (5G) shall bring disruptive developments to telecommunication services, including context-awareness and device-to-device (D2D) communication. This combination increases the potential for new types of applications, but also poses challenges to their engineering. %, as low latency, cost, volatility and heterogeneity of devices still need to be harassed.

While cloud has been and will remain as an essential model for distributed computing, problems such as latency, cost and intermittent connectivity still need to be harassed. In this paper, we introduce the concept of \textit{Ad Hoc Edge Spaces} as part of a framework for the development of mobile applications that cannot depend solely on remote servers to satisfy their communication, processing and coordination requirements. Through edge spaces, devices can form ad hoc groups to interact locally using different communication and coordination styles and to dynamically assume application roles -- like fetching and processing data from their sensors -- and autonomously managing the ad hoc spaces itself -- like reorganizing the group-role structure and keeping group knowledge. We illustrate this idea with use cases that exploit its features and evaluate our framework with the implementation of a crowdsensing application for Android platform.

%The purpose is to avoid bottlenecks, costs and the latency of cloud servers.

%This scenario has specific characteristics, such as a high heterogeneity and volatility, intermittent connectivity, and resource limitations. To further explore its potential, pervasive and mobile applications must be able to overcome challenges such as the churn of components and fluctuations of physical and computational resources. 

%Moreover, considering the nature of personal devices, their heterogeneity, and scale, applications must make proper use of resources with an efficient and coherent distribution of its activities among participant devices.
%As more and more devices join the network or because of low latency requirements, a server-base architecture in which servers in the cloud handle most of the computation may not deliver the expected results. Conversely, 
%Many devices are now equipped with more powerful resources. This enables more computation to be performed at the edge of the network infrastructure. Some applications already explore machine-to-machine (M2M) communication to allow distributed components to interact and coordinate their behavior through local and mobile ad hoc networks (MANETs). Despite the potential benefits, an horizontal architectural must also deal with problems related to the churn of devices, consistency, robustness, availability, resource limitations, fairness, dynamic policies, security, and others.  


%In the literature, data-centric models are the state-of-art for distributed components coordination, as they enable both space and time decoupling. However, their high abstraction leaves many aspects open. In this work, we propose a framework for the development of distributed and self-adaptive applications willing to explore the potential of nowadays pervasive computing. Our model inherits well-known organizational abstractions, namely groups and roles, which allow a flexible modularization of the system and a dynamic assignment of activities among heterogeneous hosting devices in scenarios of high volatility. The abstractions and features provided by the framework target the application itself (managed system) and also the layer responsible for its adaptation (managing system) -- which should preserve the normal behavior and avoid violations of the application requirement in the advent of context changes.

%We have showed the feasibility of this approach with a real-case application for public transport monitoring and compared its features with other models and mechanisms for self-adaptation.

%In addition, we detail the self-organization mechanisms and their incorporation by a middleware.  Finally, we have showed the use of our model in different application scenarios.
\end{abstract}

% no keywords
%!TEX root = main.tex
% -*- root: main.tex -*-
\section{Introduction}
\label{sec:intro}
% target -> 1 page (with abstract)

The widespread diffusion of powerful mobile devices, such as smartphones and always-on wireless connectivity solutions as 3G and LTE, have radically changed the way we live our lives. Media sharing and consumption through social Internet have never been more engaging.

In this context, innovative scenarios are emerging. For example, pervasive and context-aware applications are aware of the end-users surroundings and behave accordingly to what is happening therein. To achieve this they rely on data that are obtained from the runtime environment itself, generally collected through appropriately deployed sensor networks. 

Since their inception, modern mobile devices have included a plethora of onboard sensors to enable applications such as navigation, gaming and health monitoring. This means that mobile devices can enable the construction of new kinds of sensor networks that can be deployed in unconventional ways to locations and contexts that were previously inconceivable. Moreover, the data that we collect through these networks are not limited by the onboard sensors; instead they could be data that are explicitly inputed into an app by the end-users. 

This novel data collection paradigm goes under the name of crowd sensing. Through crowd sensing we can gather invaluable information in disaster-recovery scenarios, we can perform geo-localized and time-limited sentiment analyses, and so on.

Crowd sensing introduces novel challenges. First of all, since the sensors are embed in mobile devices, they are always on the move and subject to high churn rates. Indeed, it is common for people to enter and leave the sensing application freely and unexpectedly. This can lead to robustness issues in which the required data cannot be collected at all times due to the absence of a certain type of sensor. Second, the precision and quality of the collected data is vital. A single source of information can sometimes be misleading; indeed, the data it provides could be imprecise, or just simply incorrect. Multiple sources, on the other hand, allow us to implement various kinds of redundancy:

\begin{itemize}

\item \emph{Temporal redundancy}, in which multiple sensors, in a limited amount of time, are providing similar information  (e.g., that the temperature is rising). 

\item \emph{Spatial redundancy}, in which mulitple sensors, in a limited amount of physical space, are providing similar information (e.g., the the temperature is rising in a certain building). 

\item \emph{Dimensional redundancy}, in which multiple sensors, of different kinds, are providing valuable and relevant information (e.g., the temperature is rising as is the level of $CO_{2}$). 

\end{itemize}

Finally and also important, applications based in crowdsensing will need to scale to support extensive sensor networks due to the high amounts of physical devices that may be available for the application at any given time.

In this paper, we advocate that there is a concrete need for a middleware for developing crowd sensing applications, one that can help developers cope with the more complex issues of crowd sensing applications without having to re-invent the wheel every time. To this end we provide A3Droid, a re-interpretation of the A-3 architectural style that specifically focuses on the needs of crowd sensing applications. 

A3Droid is based on the notion of \emph{group}. With A3Droid developers can group devices in the sensor network to reduce the dynamicity degree of the application components. Indeed, while single devices can enter and leave the application unexpectedly, entire groups of devices are less likely to do so. Groups also allow us to more easily implement temporal, spatial and dimensional sensor redundancy, as well as provide a concrete means to collect up-to-date snapshots of what devices are in the application at any given time. Finally, the groups in A3Droid can be composed and de-composed on the fly to ensure that the application scales according to the churn rate. 

The rest of the paper is structured as follows. Section~\ref{sec:a3} provides a rapid crash course on the A-3 architectural style. Therein we present the main abstractions that the architectural style provides to developers. Section~\ref{sec:a3droid} discusses in-depth details on how the A-3 architectural style was implemented for Android-based mobile devices in the form of the A3Droid middleware. Section~\ref{sec:evaluation} illustrates the experiments that were performed to assess A3Droid as an effective and efficient middleware for developing crowd sensing applications. Section~\ref{sec:related} discusses related work in literature, while Section~\ref{sec:conclusion} concludes the paper and presents insights into what we intend to do in our future work in the area.












\section{Background}\label{sec:background}

%TODO architectural fluidity
%TODO characterize the music stream example w.r.t. C[1-5] and anticipate the benefits of collaboration

%1: The network model
\subsection{Wireless Networking}

Not only computers have pervaded the space inhabit by humans, wireless networking is quickly fulfilling the gaps in connectivity through which pervasive devices can communicate, including while in transit (Figure~\ref{fig:network_model}). Whereas the connectivity between devices was mostly restricted to the areas with a private or public Wi-Fi coverage, recent developments in device-to-device (D2D) technology expand the situations in which devices can communicate and interact. For example, in addition to the already consolidated Wi-Fi direct and Bluetooth Low Energy (BLE) technologies, the fifth generation mobile networks (5G) standards include the support for D2D communication~\cite{Tehrani:2014}. Thus, a new and exciting landscape for pervasive and mobile computing is taking form.

\begin{figure*}[t!]
	\centering
	\includegraphics[width=0.8\textwidth]{figures/network_model}
	\caption{Application nodes communicating through Wi-Fi or D2D technologies}
	\label{fig:network_model}
\end{figure*}

%2: Nowadays Pervasive and mobile computing
\subsection{Pervasive Applications}~\label{sec:characterization}

The market of applications crafted for pervasive and specially mobile devices continues to increase as more people have access to this technology. In specific, applications can be hosted by pervasive devices with higher or less degree of mobility and computational power, such as smartphones, tables, gadges, and miniaturized computer platforms such as Raspberry PI (which now supports the Android platform for Internet of Things~\footnote{https://developer.android.com/things/hardware/raspberrypi.html}). Moreover, new types of specialized devices like Bluetooth beacons compose the rich ecosystem in which pervasive applications can run. Despite mobile computing and devices to be the preeminent type of pervasive computing, as well as the target platform for the majority of applications, we adopted \textit{pervasive} as a broader qualitative term (instead of mobile) to avoid the discrimination of pervasive devices that exhibit limit mobility, but are potential candidates for composing the diverse scenario target by this proposal. 


\section{Self-organizing Interaction Spaces}\label{sec:edge_spaces}

\begin{figure*}[t!]
	\centering
	\includegraphics[width=0.95\linewidth]{figures/rationale}
	\caption{Role-orientation rationale}
	\label{fig:rationale}
\end{figure*}

\subsection{Separation of Responsibilities}

%A tree with the types of role: symmetric and asymmetric branches with the corresponding types of roles in increasing levels of concreteness 

%1: functional plasticity of nowaday's devices
With the technology advancements, a class of pervasive and mobile devices acquired the ability to perform general purpose computing, as well as to communicate by different means and perceive the physical world through multiple sensors. These enhancements entails a \textit{functional plasticity}, i.e., the ability of these devices to provide variations of their functionalities not only in different applications, but in different contexts of the same application. 

%2: from functional plasticity to a separation of roles
In contrast with the rigid symmetry that characterizes the behavior performed by clients nodes in a client-server architecture, commonly used in today's applications, or compared to the narrow set of functionalities provided by specialized and extremely resource-constrained devices, the functional plasticity of smartphones and other devices can be explored by letting application nodes hosted by these devices to assume distinct responsibilities, i.e., \textit{roles}.

%3: what (functional) roles are
In an organization structure, a functional role defines a context in which an individual assume the responsibility over some functionality. 
%4: why they are important for pervasive applications
Considering the collective of application nodes as an organization, different reasons may motivate \textit{separation-of-responsibilities} among them. 

%4.1a: the case of MCS; application nodes are employers of an organization, which must efficiently employ their resources
For instance, with the purpose of increasing the \textit{efficiency} in which the organization deliver its results, namely:
	
\begin{itemize}
	\item Specific roles can be played by a subset of the available nodes (e.g., only a subset of nodes in a given region must collect data about the noise pollution);
	
	\item Roles must be played by capable nodes (e.g., only the nodes with a minimum battery level and/or computational resources)
\end{itemize}

%4.1b: why did I mention organization efficiency as a motivation
%Today, MCS is the main class of application in which mobile devices are employed not to satisfy its owners needs, but the needs of external actors (e.g., a public institution responsible for monitoring the noise pollution in certain urban areas). Despite their uniqueness, the broad range of MCS applications justifies the inclusion of efficiency as a motivation. 

%4.2: the case of collaboration among nodes
In addition, separation-or-responsibilities becomes evident whenever nodes of an application need to achieve their individual or common goals in a collaboratively way. That is:

\begin{itemize}
	
	\item When some nodes have a resource or capability and others don't (e.g., sensors, Internet connection, processing capabilities, free memory, free storage, battery level, etc);
	
	\item When a functionality that benefits multiple peers can be provided by one or a subset of them

\end{itemize}


%5: When relations among nodes can be established
Whenever two nodes of an application become visible to each other and are able to communicate, there is a potential for a collaborative relation between them to be established. 
%In this relation, each part plays a similar (symmetric) or different (asymmetric) roles. 
This same premise holds true for multiple interconnected devices. 
%6: The nature of the relation between nodes in different situations
While the nature of the relation between nodes hosted by devices of different classes is mostly predefined (e.g., between a smartphone and a gadged, a tablet and a smart-tv), the nature of the relation between devices of the same class may depend on the context each device operates (e.g., a smartphone that acts as the gateway for other smatphones without Internet access). 
%7: How the social context defines the nature of the relation between nodes when they are not predefined
Accordingly, the social context, as defined by which and how many other nodes a node can interact with at a given time, further delimits which potential relations can be established between this node and its peers. 

%Whereas a P2P file sharing can be considered as a collaboration between nodes with symmetric responsibilities (all nodes are clients and servers), in other cases a single node can address the needs of its peers, i.e., to assume an asymmetric responsibility. 

%whereas for some applications be predefined (e.g., a P2P messaging application) or 


\subsection{Role-orientation}

%behaviors and responsibilities that an individual or a group of individuals can assume. 

%8: The today's client-server model and the need for role-orientation
In today's client-server model, the functionalities to be performed by the application nodes are seen as as monolithic. For the applications whose nodes are expected to collaborate and play others than just the role of a client, the functionalities that are particular to distinct application roles form a concern of their own, i.e., they must be designed and programmed accordingly. 

%9: What do we propose in terms of role-orientation
In this work, we inherit the concept of a role, more commonly used in human organizations, as part of a framework for the engineering of pervasive applications. By making the concept of a role a first-class abstraction, we provide the means to, since the early stages of its development, reason on the organizational aspects of the application, namely:

\begin{itemize}
	
	\item The roles in the application-to-be;
	
	\item The functionalities to be provided by each role, as well as the capabilities required by these functionalities (e.g., computational resources, specific hardware components, etc); 
	
	\item The relation between roles in terms of hierarchy and how they interact (e.g., which communication or coordination model);
	
	\item The cardinality relation between roles (e.g., in the symmetric cases, all nodes play the same roles; in the asymmetric cases, k-out-of-n nodes should play a given role);	
	
\end{itemize}

%Figure~\ref{fig:asymmetry} presents some examples of asymmetric roles played by different application nodes. 

%In this sense, the \textit{organization} formed by the collective of application nodes 

%application designers and programmers must be able to identify, model and program the functionality to be provided by different application roles.

%dynamic assignment of roles among peers according to their context.

%THE WOLRD CONTEXT (PHYSICAL + SOCIAL) --> THE NODE CONTEXT (ROLE) --> THE NODE BEHAVIOR


%Similarly to well studied allocation of tasks to agents/robots with dynamic and distinct capabilities~\cite{}, mobile devices are heterogeneous and subject to hight volatility, therefore their fitness in performing different tasks varies from one device to the other and throughout time. 

%and contextualization, the dynamic allocation of tasks to different application nodes is a central feature that further distinguishes this work from previous frameworks~\cite{}. 


%their need to send/receive from each other application data or control directives (or both). 
%For instance, a coordinator is responsible for controlling the activities of its coordinates and an aggregator is responsible for aggregating data from multiple sources. The relation between roles may be further characterized by the multiplicity in each side of the relation. 



%For this, there must be a clear separation of concerns and modularization of the application logic from different roles. 


%Whereas the type of role depends on the domain of the application, in this paper we propose general abstractions and mechanisms to address the range of pervasive applications whose functionalities may be asymmetrically distribute among its nodes. 



%The relation between roles is defined based on what they are expected to provide one another. For instance, a node may be responsible for giving control directives to other nodes or to process a batch of data from other nodes. For instance,
%TODO refine this paragraph and add examples from the other motivating scenarios 
%in opportunistic MCS, application nodes may collaborate by aggregating fetched data from multiple devices into a single device~\cite{Rajagopalan:2006}, which in turn sends a preprocessed data to the server (e.g., the one with higher battery level). Thus, one-out-of-many nodes must assume the role of an aggregator, meaning that the functional role of this node changes to fit a given context. Also, if the number of nodes at a given moment is high, not every node must perform the same sensing tasks. Instead, a subset of application nodes can be assigned to a subset of sensors, reducing the overall battery consumption and networking. Once again, the functional role of these nodes varied to reflect the context they were exposed. 

%Together, the specification of what a role does (behavior specification) and how it interacts with other roles (social specification) defines the organization structure of the application. 

%FIGURE: CIRCLE WITH HALF SYMMETRIC HALF ASYMMETRIC BEHAVIOR

%TODO: add here the fitness function definition and examples

\subsection{Groups}

%From the collective of roles to the collective of grouped roles
%Why to group nodes together?
%What defines a group?
%How grouping works?

%Further refines the context of an application node: the role is not automatically set, but must be agreed 
%Strict vs relaxed group membership
%Figure of a circle representing the set of members and an inner circle representing the active members
%Relation of a group and the physical space

%Group membership criteria filters out heterogeneity

%The paradigm shift from a bipolar into a multi-polar organization discussed by this paper is strongly motivated by the co-existence of a plethora of pervasive devices from different kinds in the physical space inhabited by people. As human society evolves and blends itself with a fabric of interconnected cyber-physical devices, a question emerges: how should this ecosystem be organized? Whereas the client-server has been proven a simple and scalable answer, the potential of peer-to-peer interactions must not be ignored.

%Three dimensions characterizes the ecosystem of applications hosted by pervasive and mobile devices: \textit{space}, \textit{time}, and \textit{heterogeneity}. 


%Figure~\ref{fig:rationale} illustrates the rationale behind the assignment of roles to an application node.

%The combination of the social context with the individual context of each node 

%10: 
The application organization -- so far represented by the roles that can be played by its nodes -- may be further characterized by its partitions, here named as \textit{groups}. 
At its highest level, a group may correspond to the set of application nodes within the same network partition. Nonetheless, these nodes, or a subset of them, may exhibit common properties or states of interest to the application. For instance, a group may have a \textit{functional} meaning (e.g., a group of nodes able to fetch data from a specific type of sensor), a \textit{non-functional} meaning (e.g., all nodes within a specific geographical area), or a mix of both (e.g., all nodes able to fetch data from a specific type of sensor within a specific geographical area). Thus, multiple types of criteria may define the membership of a group.

To its members, a group defines a \textit{social context} in which they (a) may or (b) must play certain roles. Whenever the relation between nodes in a group is asymmetric, and there is no predefined criteria, nodes must agree on the specific roles they will play (a). Conversely, if the relation between nodes in a group is symmetric, there is no decision to be made: all nodes must play the same role (b). However, as (a) can be seen as a more general case of (b), we adopt a relaxed membership causality, i.e., group membership only defines the context in which its members \textit{may} play one or more roles, unless its specification makes such restriction. Accordingly, each of these cases can be specified as:

\begin{itemize}
	
	\item \textbf{Symmetric:} A m-m specification (m instances of a role in a group of m nodes) defines a symmetric role, i.e., a role played by all group members (strict membership causality);
	
	\item \textbf{Asymmetric:} A k-m specification defines an asymmetric role, i.e., a role played by a subset $K$ of the set $M$ of nodes in the group, with $|K| = k$, $|M| = m$, and $k < m$;
	
	
\end{itemize}

%TODO: look out for the difference between asymmetric group-role relation and asymmetric role-role relation

The later case encompasses any kind of asymmetric relation between roles, including the common cases in which one node in the group interacts with all other members through an hierarchical relation (e.g., a supervisor and its followers~\cite{}). Furthermore, once this specification exists, group members must agree on the actual allocation of roles. Such \textit{distributed role allocation} mechanism is presented in Section~\ref{sec:self_organization}. Figure~\ref{fig:asymmetry} illustrates examples of both symmetrical and asymmetrical relations within a group. 

\begin{figure}[t!]
	\centering
	\includegraphics[width=0.48\textwidth]{figures/asymmetry}
	\caption{Left: application nodes perform the same behavior (e.g., content sharing); middle: application nodes perform asymmetric roles without hierarchy among them (e.g., different sensing tasks); right: one node performs an hierarchical control (e.g., coordinator) or communication (e.g., data aggregator).}
	\label{fig:asymmetry}
\end{figure}


%Due to the potentially high density of application nodes within a given zone/network, there may be a surplus of nodes able to perform one or more roles. Hence, instead of adopting a strict causality (I) between group membership and the role(s) to be played, we adopt a relaxed causality (II) in which group membership defines only the context in which its members \textit{may} play certain roles (precondition). This decision has the advantage to reduce the role assignment scope, as eligible nodes are picked from the group and not from the whole set of application nodes. 

%for applications such as MCS, in which application nodes provides functionalities whose purpose is to address the needs of an external actor and not the users themselves, 




%For example, 10 out of 15 devices are capable to measure the noise level, but only two are needed. In one hand, if the membership to a noise-pollution group would imply sensing the noise (I), only two nodes would be allowed in the group. Also, the distributed election of these two members would include all 15 devices, even those not able to perform such behavior. In the other hand, if membership implies capability (II), all 10 capable nodes would join the group and the distributed election of the two nodes would happen among the 10 member.

%Hence, at a first level, groups provide a subspace in which its members may (II) play specific roles. 

%TODO: go into details about what happens once a group is formed and roles are allocated, i.e., how roles interact with each other



Within a group, member nodes are aware of each other. Not only they can agree on which roles should be performed by which nodes, the elected roles can engage in interactions following an architectural style specified for them. Thus, a group abstraction provides a container to which roles can be added or removed dynamically. Considering the functional plasticity and role asymmetric previously discussed, more than one type of role may co-exist in a group. Also, the distribution of roles to the group members should follow a \textit{fitness criteria}. This criteria must provide a metric indicating which nodes are best suit to play the group roles.

The modeling of an application group starts with the specification of its membership criteria. Any type of criteria that can be modeled as a proposition and evaluated is a potential criteria, including:

\begin{itemize}
	
	\item Static criteria
	
	\begin{itemize}
		
		\item \textbf{Hardware capabilities:} refers to the presence of a given hardware component/module. E.g.: camera, GPS, thermometer, accelerometer, gyroscope, etc.
		
	\end{itemize}
	
	\item Dynamic criteria
	
	\begin{itemize}
		
		\item \textbf{Physical world:} refers to the physical world states a node must operate in. E.g.: its current battery level, available memory, geolocation coordinates, acceleration, speed, temperature, etc.
		
		\item \textbf{Application domain:} refers to the application states a node must be to belong to a group. E.g.: currently a member of another group (or non-member), joining a chat or game session, etc.
		
	\end{itemize}
	
\end{itemize}

For example, a mobile crowdsensing campaign is designed to monitor the real-time geolocation of public buses in a city and to register unusual acceleration and deceleration events that may affect the user experience in this service. To avoid disturbing the users with the need of starting the application whenever they are within a bus, an opportunistic approach~\cite{} requires the automatic detection of such context. In this example, we assume city buses to provide wi-fi. Accordingly, a \textit{bus group} is specified with the following criteria:

\begin{itemize}
	
	\item Bus Group Membership Criteria
	
	\begin{itemize}
		
		\item Wi-Fi BSSID matches a well-known pattern; AND
		
		\item Wi-Fi signal is not weak; AND
		
		\item Device's location has changed in the last 5 minutes
		
	\end{itemize}
	
\end{itemize}

With this criteria, a background service checks for a given pattern in the BSSID of the Wi-Fi detected by the device, as the transport provider is likely to use a BSSID that identifies the company service. If that pattern is found and if the corresponding signal is not weak (meaning the user is likely to be inside the bus), the last criteria filters out any device in the proximity of parked buses. Once these criteria are met, the application node may join or create a bus group.

Each measurement target by the campaign -- geolocation and acceleration -- is provided by independent sensors -- namely GPS and accelerometer. Thus, two additional groups are specified with the following criteria:

\begin{itemize}
	
	\item Geolocation Group Membership Criteria
	
	\begin{itemize}
		
		\item Member of the Bus Group; AND
		
		\item Has GPS; AND
		
		\item Battery level above 40\%
		
	\end{itemize}
	
	\item Acceleration Group Membership Criteria
	
	\begin{itemize}
		
		\item Member of the Bus Group; AND
		
		\item Has accelerometer; AND
		
		\item Battery level above 25\%
		
	\end{itemize}
	
\end{itemize}

Both criteria are similar, except for the type of sensor the device must have and its minimum battery level. As the GPS is known for been a battery consuming sensor, a higher battery threshold has been specified, meaning only devices with 40\% of battery or more are eligible to join that group (in contrast to the minimum level of 25\% for the acceleration group). Finally, the first criteria refers to the membership to the bus group, which indicates that a node is within a bus to be monitored.


%QUESTION: is the specification of how roles should interact inherent to the role specification or to the group specification? If the interaction should happen between roles of different kind, this specification do not belong to one role class; it can belong to the group, which allows different groups to define different architectural styles for the same class of role. Thus, I believe this specification belongs to the group in which a role is to be played. 

%Additionally, groups provide a well defined scope in which these roles can interact according to a specified architectural style. Functional groups are further associated with the roles that can 

%s a functional group is associated to a set of functionalities, its specification must also include the intended type of role(s) to be played. In addition to that, its specification must define the relation between these roles, i.e., how they interact.  

%In addition to the functional precondition aspect, a group membership criteria may also be useful to group correlated nodes according to some application semantic. For instance, 

%Due to the heterogeneity and volatility of the devices hosting a mobile application, some nodes may not satisfy the requirements to perform one or more roles. Conversely, the nodes that satisfy these requirements form a group of eligible nodes. 


\begin{figure}[t!]
	\centering
	\begin{subfigure}[b]{0.4\textwidth}
		\centering
		\includegraphics[width=1\textwidth]{figures/physical_view}
		\caption{Spatial criteria: application components (white circles) grouped by their position in the physical space (blue shapes).}
		\label{fig:spatial_criteria}
	\end{subfigure}%
	
	\begin{subfigure}[b]{0.4\textwidth}
		\centering
		\includegraphics[width=1\textwidth]{figures/logical_view}
		\caption{Virtual criteria: application components (white circles) grouped by logical criteria defined by the application.}
		\label{fig:virtual_criteria}
	\end{subfigure}
	\caption{Application nodes grouped by different types of criteria}
	\label{fig:grouping_criteria}
\end{figure}

%Whereas the isolation provided by groups can be achieved using  methods that limit the audience of messages -- for instance,  publish/subscribe mechanisms in which events are only captured by registered listeners -- the group abstraction complement the organizational view of the system provided by the role abstraction and make its social aspect more explicit. Together, these abstractions form the building blocks that allow engineers to think and program highly distributed mobile applications in terms of their organizational structure. The decision of which communication and coordination methods to use within each group remains orthogonal: message queues, events, and data-centered models are likely to suit better different applications and use cases.

%abstraction brings forward the social aspect of applications whose components should behave and interact based on the role they are assigned to in the organization they belong to

%patterns of interaction and behavior that compose the structure 


%behaviors (generally asymmetric) that fulfills one or more application requirements. This concept has been used by organization centered multi-agent systems~\cite{}, for which organizations are frameworks where agents with different capabilities may interact. Such abstraction brings forward the social aspect of applications whose components should behave and interact based on the role they are assigned to in the organization they belong to, going beyond the static client-server roles of today's mainstream model for mobile applications. %Figures~\ref{fig:client-server} and~\ref{fig:crowdsensing} illustrate the difference between these models.

%FIGURE: a) CLIENTS + SERVER; b) INTERACTING ROLES + SERVER


%DISCUSSION: one could argue that the dynamicity of the scenario does not fit well with the idea of a group, as the update of the group view may become constant. In this scenario, nodes could just advertise their fitness to perform different roles/tasks for a given application. Accordingly, they would still be able to build a table and decide for the allocation. The need for a group seems superfluous, unless there are many-many application nodes/roles and all of these nodes must broadcast their fitness for all eligible roles. In this case, group membership could be seen as the state in which nodes must do something while non-members must not. For instance, only group members must advertise their fitness, and the lowest threshold (lowest FV) is made public so the non-members would know when to become a member. 
%Ex: a sonore pollution group is defined with cardinality=8. There are, simultaneasly, 20 devices in the target campaign area. When the campaign is about to start, each device broadcasts its FV for that role, which allows them to build a shared FV table. The first 8 devices in the resulting FV table are considered as active. Whenever the 8th position FV threshold is met by a non-active node or by a superior position, the corresponding node advertises its FV: the node currently in the 8th position will go, respectivelly, to the 7th position or it will become a non-active member. In both cases, all nodes become have their FV table updated. Conversely, if the FV of the 8th position node changes, it must advertise this change and let other nodes to become aware of the new threshold.
%So, the group abstraction is mostly useful for modeling purposes, but also for programming and for guiding the implementation of the middleware. 
\section{Self-organization}\label{sec:self_organization}


%While self-organization principles have been extensively used in the context of multi-agents and robots, as well as in other fields, they have mostly addressed the achievement of a specific set of application goals (e.g., in the design of rescue robots that, based on their local knowledge, actions, and interactions, will lead to the emergent behavior of finding and rescuing the victims). In contrast, in this work self-organization means the literal self-adaptation of the system organization. 

%Additionally, bio-inspired and other self-organization methods have a strong relation with the restrictions imposed by the physical world. For instance, the actions of a robot are based on its own knowledge and the knowledge it receives from its neighbors. In many cases, this is due to the limitation in the communication methods. The concept of a neighbor is, in this case, determined by spatial proximity. However, if the communication method is powerful enough to let all robots to exchange information, the concept of a neighbor cannot be determined by the spatial vicinity.

The structure of the application organization
%, formed by the two abstractions presented -- namely, roles and groups -- 
must be malleable (plastic) to accommodate changes in its ecosystem composition and environment. In contrast with the much less dynamic cases of organizations in human societies (e.g. industry and military organizations), the volatility of pervasive and mobile devices, caused by their mobility or fluctuations of their resources, may imply the formation or dissolution of relations among nodes and require the reassessment of the roles they play in the organization. 

While self-organization have been extensively used in the context of multi-agents and other fields, they have mostly addressed the achievement of system goals by means of actions and interactions to be performed by individuals based on local knowledge and without external control~\cite{}. In contrast, in this work we propose autonomous mechanisms to adapt the organization the application ecosystem, i.e., to let application nodes to agree on their role in the system according to what is required from the system and the context these nodes perceive themselves in. In specific, we propose a mechanism for each of the following dynamics:


%The long term evolution of the system has also been studied 

%As an example, rescue robots are designed to decide where to go next based on their own sensors or the knowledge received from other robots they meet~\cite{}. will lead to the emergent behavior of finding and rescuing the victims). In contrast, in this work self-organization means the literal self-adaptation of the system organization. 

%The term \textit{self-organization} have been mostly used to refer to emergent behaviors of the system based on local knowledge and actions~\cite{SELF_ORGANIZATION^4}. In this work, it denotes the distributed and autonomous adaptation of the system organization in the advent of context changes, i.e., it literally means \textit{self-organization} of the ecosystem formed by the application nodes. Nonetheless, the self-organization mechanisms proposed here share many of the characteristics of ....:

%\begin{itemize}
%	
%	\item \textbf{No external control:} no external component participates in the decision making to adapt the organization;
%	
%	\item \textbf{Distributed decision making:} ... 
%	
%\end{itemize}

%In this sense, each application node takes part in the process of perceiving its individual context (availability of resources, capabilities), as well as its social context (with who the node can interact). In this paper, we focus on two different self-organization mechanisms:

\begin{itemize}
	
	\item \textbf{Group membership:} a node may join or leave a group according to its satisfaction to the group membership criteria; to address this, a \textit{grouping} mechanism is proposed;
	
	\item \textbf{Role election:} the nodes within a group must agree on which roles they shall perform; to address this, a  \textit{role election} mechanism is proposed;
	
	\item \textbf{Role cardinality:} the relation between the number of instances of a role and attributes of system may vary with the context; to address this, a \textit{role cardinality}  mechanism is proposed;
	
	\item \textbf{Group cardinality:} finally, in some situations the membership criteria may have to be relaxed to allow more nodes to join a group and achieve its goals; to address this, \textit{group cardinality} mechanism is proposed.
	
\end{itemize}


\subsection{Self-grouping} 

%mobile crowdsensing campaigns may target specific city areas; thus, only the nodes which are currently within these areas are eligible for participating of the respective campaigns. Campaign groups can be modeled with a physical criteria specifying the geolocation coordinates of the area (Figure~\ref{fig:spatial_criteria}). Also, specific sensing activities may also correspond to groups using the hardware membership criteria (Figure~\ref{fig:virtual_criteria}).


%TODO: replace the figures bellow with a MC example

As the idea of a group in this work is not related to security, the \textit{self-grouping} mechanism is not controlled by special-purpose components external to the system; instead, each node is responsible for checking its own satisfaction to the existing membership criteria in the application organization. Whenever the satisfaction of a group membership criteria changes, this event must be  advertised to other nodes, who update their group membership registry. 
%TODO despict how the problem of time coupling by using gossip or other algorithms for advertisement/discovery
The self-grouping procedure is further depicted by the diagram in Figure~\ref{fig:self_grouping}.
 
\begin{figure}[t!]
	\centering
	\includegraphics[width=0.45\textwidth]{figures/join_or_create}
	\caption{Activities of the self-grouping procedure performed by each application node}
	\label{fig:self_grouping}
\end{figure} 
 
%TODO: update the diagram

Nodes entering the application have no registry of the existing group membership. Nonetheless, they acquire this information from the groups they eventually join for the first time. 
%TODO: describe which nodes is responsible for sending this information

\textbf{Complexity Analysis} The communication overhead of this mechanism can be divided in two parts: a) the message sent each time a node joins/leaves a group (registry update); b) the message with the copy of the membership registry to the nodes joining an application group for the first time (registry copy). 

In the worse case scenario, registry update (a) takes $m-1$ unicast messages ($O(m)$), with $m$ the group size. However, if a broadcast communication is used, a single broadcast message can advertise the registry update. This is the case, e.g., of an UDP broadcast over the 802.11 network protocol (Wi-Fi). The registry copy (b), in its turn, requires a single unicast message ($O(1)$) to be transmitted each time a node joins a group for the first time. In contrast with the previous type of message, the registry copy includes information about all $m$ nodes in the group. Thus, the larger the group, the larger the payload to be transmitted. 

 
\subsection{Distributed Role Allocation} 
 
The decision of which node(s) should be assigned to which roles(s) may depend on many aspects. Trivially, any node capable of performing a role is a potential candidate. Notwithstanding this, attributes such as the quality of the data tend to variate from one node to another and throughout time. Also, devices have different levels of battery available. Thus, a balanced role allocation must respect the trade off between what is best for the application goals and for the individual devices. 

%In specific, we adopt the definition of context as...+++..., including their physical and the social contexts; in specific, context is reified as:
%
%\begin{itemize} 
%	
%	\item which application nodes are currently present (provided by existing discovery mechanisms);
%	
%	\item the capability and fitness of these nodes to perform functionalities required by the application (addressed by this framework).
%	
%\end{itemize}

%Despite the many possible types of asymmetric tasks, in this paper we focus on those whose importance for the application and cost for the devices are significant, i.e., the computation and communication overhead for its allocating is justified. 

\textbf{Allocation Classification} We use Gerkey and Matarić’s taxonomy~\cite{} to present a categorization of the role allocation problem along three axes. In the first axis, nodes are categorized into two types: single-role versus multi-role nodes\footnote{Originally, the taxonomy refers to \textit{robots} and \textit{tasks}, while in this paper we refer to the (application) \textit{nodes} and the \textit{roles} they can perform.}. As application nodes are generally capable of performing more than one role at a time -- for instance, to fetch from multiple sensors -- nodes are considered as multi-role. In the second axis, roles are categorized into two types: single-node vs multi-node roles. While certain types of role are performed by a single node -- for instance, the sensor data aggregator -- there are cases in which a role must be performed by multiple nodes. Thus, node cardinality depends on the role type. Finally, in the third axis, the allocation is also categorized into two types: instantaneous assignment or time-extended assignment. Due to the volatility of mobile devices (churn and capability changes), the scheduling of future allocations tends to fail. Accordingly, we consider an instantaneous assignment of roles based on the context of the involved devices.

%The later is specially important in volunteer based applications such as mobile crowdsensing, as efficient and fair use of devices resources works as an incentive for participation. 

%FIGURE: META MODEL FOR NODE-ROLE

\textbf{Requirements} As the we consider a scenario in which application nodes hosted by volatile and resource constrained devices must self-organize their activities, the allocation method shall satisfy the following requirements:

\begin{enumerate}[label=R\arabic*.]
	
	\item Distributed: allocation decision should not be centralized by one device in order to avoid bottlenecks;
	
	\item Communication overhead: only essential data should be exchanged in order to minimize communication overhead; 
	
	\item Coherence: the allocation method performance/cost should be tuned according to the role constraints and criticality.
	  
\end{enumerate} 

%TODO: Move this to the related works section
In the literature, many works have tackled the problem of distributed allocation of tasks~\cite{DTA}. Table~\ref{fig:asymmetry} classifies these methods according to their type and their satisfaction of the requirements R[1-3].

\begin{table}[ht!]
	\centering
	\begin{tabularx}{\linewidth}{@{}| *1{>{\centering\arraybackslash}X}|c|c|c|@{}}
		\hline 
		 & \textbf{R1} & \textbf{R2} & \textbf{R3} \\
		\hline
		M1 & Y & N & Y\\
		\hline 
		M2 & Y & N & Y\\ 
		\hline
		M3 & Y & N & Y\\
		\hline
	\end{tabularx}
	\caption{Distributed allocation methods}
	\label{tab:role_node_cardinality}
\end{table}

Auction-based allocation methods have been extensively studied in the multi-agents field. In comparison with the dynamic scenarios tackled by these approaches, the problem of allocating roles to a large scale of mobile devices subject to high volatility tends to be much more intensive in terms of how often the allocation scheme must be reevaluated. Accordingly, a frequent message exchange between bidding and auctioneers nodes is expected. To mitigate this problem, nodes should evaluate their fitness (self-evaluation) and only advertise it to other group members if a delta in the value occurs.

\textbf{Fitness} Whereas the optimization of quality attributes may deem unfeasible due to its complexity, the allocation of roles can still be guided by the fitness (or utility) of nodes in performing distinct roles. For each role, fitness is modeled as real-value function of relevant features affecting one or more attributes of the application. For example, the fitness function of an \textit{aggregator} role in a OMCS application be modeled as: 


$$
f_{aggregator} = battery\_level * membership\_factor
$$

\noindent
in which $battery\_level$ and $membership\_age$ are both real positive numbers ranging from 0 to 1. The former parameter can be accessed from the platform hosting the node, while the last parameter refers to the time a node belongs to a group. The following function illustrates how this parameter could be evaluated:

$$
membership\_factor = MIN((60 - age)/60, 1)
$$

\noindent
in which $age$ is given in seconds and refer to the uninterrupted time a given node is in the group. Thus, if the node has been more than a minute in the group, it has a $membership\_factor$ of one; otherwise, $0 \le membership\_factor \le 1$.

If a role position is been elected for the first time, every node should advertise its fitness value and agree on the node to assume that position. Otherwise, the replacement of the current position should consider the trade off between the gain of having a new node elected and the cost of replacing the current node. 

%Figure~\ref{fig:asymmetry} presents the activity diagram for the procedure performed by each node.


%For example, one-out-of-many device may be a better candidate to host the coordinator of the activities performed by a group of application nodes or to use specific sensors to extract information from the physical environment. Whenever asymmetry characterizes a behavior, i.e., not every node must perform it, and the performance, the cost, or other attributes of such behavior are to be considered, the decision of which node to perform it should take the context of the hosting devices into account. In this work, we considered asymmetric tasks whose importance for the system justifies the overhead for their allocation. %We compare our approach with straightforward solutions to demonstrate the feasibility and benefits of a context-dependent allocation. 

%In specific, we focus on the \textit{efficiency, robustness, and fairness} resulting from an appropriate distribution of system responsibilities to local devices taking into account the dynamicity of their resources and capabilities, as well as their history of participation (fairness). To achieve a scalable solution for these problems, we propose a two-level mechanism for the partitioning of the edge space into subset of devices (groups) and the classification of these subsets by their fitness to play different system roles (assignment). 



%the partitioning of the edge space according to different criteria and

 %In special, we are concerned with system functionalities that can assigned to one or more devices within the edge space. 
%The main novelties here are:
%
%\begin{itemize}
%
%\item Pervasive and mobile devices are seen as volatile and heterogeneous computational platforms with resource limitations.
%
%\item Distinct responsibilities may be assigned to a subset of the mobile devices running the same application.
%
%\end{itemize}


%First, we show how our framework can enable ad hoc interactions without external storage and control provided by cloud infrastructure. Second, we present a mobile multi-player game scenario in which low latency communication is the main requirement. Finally, we describe a mobile crowdsensing application, which combines different requirements, including efficient use of battery and mobile data plan. 



%which here are hampered by volatility, 
%
%\begin{itemize}
%
%\item Discovery: devices must be able to recognize each other in the space
%
%\item Transparency: devices must be able to communicate without previous knowledge of their network address
%
%\item Dynamic allocation: devices must be able to agree on the role they will play in the system 
%
%\item Partitioning: devices must be able to further partition the interaction space 
%
%\end{itemize}

\section{Self-adaptation}\label{sec:self_adaptation}

\subsection{Role Cardinality} 

The relation between the number of instances of a given role and different non-functional attributes of a system may be straightforward (e.g., the number of similar sensors in a cluster and the accuracy of the information this cluster provides) or less evident (e.g., the number of replicas of a component and the availability or reliability of the functionality it provides). In addition, considering the scenario of mobile applications, the mapping between role cardinality and these attributes tend to change according to the context they operate, i.e., it may depend on which devices are assigned to the role (e.g., more/less powerful computational resources, more/less accurate sensors) and the state of each device (e.g., more/less battery, more/less signal strength). Hence, in order to avoid violations of these attributes, the cardinality of a role must reflect the circumstances. %To this end, we propose a last, but not least important mechanism for adapting the application organization. 

\subsection{Group Cardinality} 

Since the encounter of mobile application nodes is opportunistic and not deterministic, group membership criteria specifies no size threshold or limit: as long as a node satisfies the group's membership criteria, it becomes/remains a member. This choice avoids the need for a group membership control by a trustful component and simplifies the grouping process. Accordingly, the size of a group, kept stable the membership criteria satisfaction by each of its members, can only be changed by making this criteria more relaxed or strict. 

Whereas the self-organization mechanisms presented address the \textit{solution space}, i.e., which nodes provide which application functionality, here the \textit{requirement space} is the target of the adaptation, i.e., the application requirements evolve to accommodate situations in which no solution can satisfy those requirements.

%\section{Self-adaptation}\label{sec:self_adaptation}

\subsection{Role Cardinality} 

The relation between the number of instances of a given role and different non-functional attributes of a system may be straightforward (e.g., the number of similar sensors in a cluster and the accuracy of the information this cluster provides) or less evident (e.g., the number of replicas of a component and the availability or reliability of the functionality it provides). In addition, considering the scenario of mobile applications, the mapping between role cardinality and these attributes tend to change according to the context they operate, i.e., it may depend on which devices are assigned to the role (e.g., more/less powerful computational resources, more/less accurate sensors) and the state of each device (e.g., more/less battery, more/less signal strength). Hence, in order to avoid violations of these attributes, the cardinality of a role must reflect the circumstances. %To this end, we propose a last, but not least important mechanism for adapting the application organization. 

\subsection{Group Cardinality} 

Since the encounter of mobile application nodes is opportunistic and not deterministic, group membership criteria specifies no size threshold or limit: as long as a node satisfies the group's membership criteria, it becomes/remains a member. This choice avoids the need for a group membership control by a trustful component and simplifies the grouping process. Accordingly, the size of a group, kept stable the membership criteria satisfaction by each of its members, can only be changed by making this criteria more relaxed or strict. 

Whereas the self-organization mechanisms presented address the \textit{solution space}, i.e., which nodes provide which application functionality, here the \textit{requirement space} is the target of the adaptation, i.e., the application requirements evolve to accommodate situations in which no solution can satisfy those requirements~\cite{RELAX}. 



\section{Evaluation and Discussion}\label{sec:evaluation}



The simulation experiments aimed at showing the benefits of the approach and measuring the overhead imposed by the self-organization mechanisms.
%, namely self-grouping and distributed role allocation. 
As these methods add no significant overhead in terms of processing or memory, an asymptotic analysis focused on the communication overhead.
%, as the exchange of messages through wireless mediums consumes battery from devices and is subject to delays that may interfere with the application behavior. 
As for the benefits of the approach, we executed simulation experiments of a public transport monitoring application using a pure client-server and using our framework. The goal was to compare, in each case, the following two metrics:

\begin{enumerate}[label=\textbf{M}\arabic*:]
	
	\item the total number of sensing tasks performed and requisitions fired from clients to backend servers (evaluates battery consumption and Internet traffic); and
	
	\item the total number of failures in reaching backend servers due to intermittent connectivity (evaluates robustness).
	
\end{enumerate}

%In specific, this metric was evaluated 

%\begin{enumerate}[label=\Alph*]
%	
%	\item The asymptotic overhead, given by worse case number of exchanged messages required by each self-organization mechanism, as a function of the number of application nodes, groups and roles (Section~\ref{fig:asymmetry}). 
%	
%	\item The measured overhead, given by the number of exchanged messages counted during simulated executions of a MCS application for public bus monitoring, in which the number of group/roles is fixed and the number of nodes, as well as their capabilities, varies according to probabilistic distributions.
%	
%\end{enumerate}


\subsection{Asymptotic Analysis} 

\subsubsection{\textbf{Self-grouping}} the complexity analysis was divided in two parts: a) the overhead when a node joins/leaves a group (registry update); b) the additional overhead when a node joins a group (registry copy).

In the worse case scenario, registry update (a) takes $n-1$ unicast messages ($O(n)$), with $n$ the group size, and a single registry line as payload. However, if a broadcast communication is used, a single broadcast message (e.g., an UDP broadcast over Wi-Fi) can advertise the registry update ($O(1)$).
%This is the case, e.g., of an UDP broadcast over the 802.11 network protocol (Wi-Fi). 
The registry copy (b), in its turn, requires a single unicast message to be transmitted each time a node joins a group ($O(1)$). Important to mention, in contrast with the registry update message, the registry copy includes information about all $n-1$ nodes in the group. 

\subsubsection{\textbf{Discentralized Role Allocation}} 

%TODO: check the definition of FS and use it accross the paper

this mechanism involves the exchange of fitness scores ($FS$) among eligible nodes in the advent of the events depicted in the previous subsection. The actual number of $FS$ messages sent/received during a role position election depends on how many nodes in the $n$ size group satisfy the role restrictive criteria (RRC) (see Eq.~\ref{eq:rrc} in Section~\ref{sec:edge_spaces}). Once more, the type of network is a determining factor.

In the worse case, represented by an election of a vacant position (following a \textit{vacancy} or \textit{resignation} event) and without broadcast, each node $e$ in the set of eligible nodes $E$ must send each other an unicast message. If $|E| = n$, the message count is given by $O(n * (n-1)) = O(n^2)$. If the nodes in $E$ can communicate through broadcast, this number is reduced to $O(n)$. In its turn, the challenge event produces smaller overhead as the exchange of messages is restricted to a request-response between challenger and challenged and to the subsequent advertisement of the result ($O(n)$ without broadcast, otherwise $(O(1))$). 

%added to the node's classification list in polynomial time ($O(p*log(p)))$, with $1 \le p \le n$. 

Once the communication overhead of a single election round is known, the overall overhead can be estimated by the number of groups and roles in the application and the frequency of events triggering new elections. Whereas the vacancy and resignation events should be handled with a new election, the frequency in which a challenge event happens is greatly affected by the choice for $\delta$: the higher this factor is, the lower the probability of a challenge (and the need for the elected nodes to update their peers about changes in their $FS$). Therefore, the decision of which $\delta$ to be used depends on the criticality of the attributes that compose the fitness score of a node. 



\subsection{Public Transport Monitoring Simulation}

\subsubsection{Experiments Design}

the group-role specification (List.~\ref{lst:bm_criteria} in Section~\ref{sec:edge_spaces}) was used for the modeling of a dynamic bus monitoring application scenario. In it, the following variables were considered:

\begin{itemize}
	
	\item the number of nodes within a bus: \textit{from 2 to 10 nodes};
	
	\item the battery level of each node: \textit{from 10\% to 90\%};
	
	\item the Internet type of each node: \textit{cellular or Wi-Fi};
	
	\item the GPS signal in each node: \textit{from 0\% to 100\%};
	
\end{itemize}

In the client-server approach, each node tries to collect data from its accelerometer and GPS before sending it to the server through whatever type of Internet connection available (Wi-Fi or cellular data plan). In our approach, at most two simultaneous nodes are assigned to each type of sensor and one node to aggregate data before sending a preprocessed batch to the server. The number of times each sensor is fetched, as well as the number of times a connection between a client/aggregator and the server, were counted in separate (\textbf{M1}). Finally, we also counted the monitoring windows in which no measurement reached the server (\textbf{M2}).

%We modeled the cost of sampling the GPS as 5 power units (pw) and 1pw for the accelerometer, as in most cases it works continuously, but still requires a background activity running. Plus, we modeled the cost of sending data to the server through cellular plan as 10, through Wi-Fi as 5, and through 

The experiments were performed with, \textit{PeerSim}, an open source peer-to-peer simulator~\cite{p2p09-peersim}. This tool supports the creations of different network P2P topologies and provides a handful JAVA programming interface.



%To simulate a public bus monitoring crowd-sensing application, experiments were performed with an increasing number of nodes. Each role can potentially play a \textit{gps-monitor}, a \textit{acceleration-monitor}, or an \textit{aggregator} role. Each role has been implemented as a protocol extending the \textit{CDProtocol} class. At each simulation cycle, 
%
%the application roles of a \textit{gps-monitor} and \textit{acceleration-monitor} were implemented as protocols, which are called at each simulation cycle. 
%
%The simulation was executed in a computer with .... 

\subsection{Results}

Add Figures here

\subsection{Discussion}

Discuss the obtained results
\section{Related Work}
\label{sec:related_work}

Linda... LIME... TOTA...

The A-3 model also proposes the use of groups as more stable entities for the development of distributed systems that can operate in dynamic and volatile environments. While this work share many of the motivations and have similarities with respect to A-3, our model does not rely on the rigid \textit{supervisor-follower} group structured proposed by A-3, nor group compositions rely on shared members. In contrast with A-3 middleware features for topology control operations and its more abstract architectural style for self-adaptation, we give special focus to concrete mechanisms and that should guarantee the basic properties of groups, such as a high availability, efficient use of resources, and other attributes defined by the application through light-weight extension points; finally, we investigate the integration of groups with tuple spaces for both inter-group and intra-group coordination, while A-3 rely on classical group communication methods.

%A3-TAG, a programming model that facilitates the design of self-adaptive distributed systems based on group abstractions~[CT]. A3-TAG is an extension of the A-3 model, which is used as the organization model. A-3 key elements are groups and two types of roles, namely supervisor and follower. Each group has a supervisor and a variable number of followers. The main differences between this work and A3-TAG are dual: first, our organization model imposes no restriction about the number of roles a group can have, nor it specifies any hierarchy between roles. Second, our adaptation model is not based on coordination groups, but rather on the direct interaction between groups based on self-organization principles. We argue that the complexity of forming and managing coordination groups may result in excessive overhead. Instead, we propose a \textit{structureless} method to self-adapt the organization structure.

The group-role abstraction has also been used in other domains. Ferber et al.~[CT] proposed an organization centered model for multi-agents systems that contrasts with agent centered models, in which agents can communicate and interact freely. Among the problems of agent centered models, the authors cited security, modularity and the lack of support to other frameworks besides the multi-agent platform itself. In our work, we agree with those arguments as part of the justification for an organizational approach for distributed systems. The main difference are the domains: instead of agents, we consider pervasive devices as the hosts of components that play roles in groups of distributed systems. In addition, we focus on the self-management of the group-based organization in the advent of context changes.


%Kota et al.~[CT] have proposed a method for adapting the relationship between agents in a multi-agent system. In their work, agents reason about adaptation using only historical knowledge about past interactions and the cost of adaptation (meta-reasoning). Despite the similarities regarding the use of self-organization principles, our work differs in the organization model (groups, roles, and devices instead of agents and relationships) and also in the elements of the organization structure that are subject to adaptation (roles played by different devices in different groups instead of agents relationships). 



Other works based on partitions target specific domains such as mobile ad hoc networks (MANET). As an example, ...



%!TEX root = main.tex
% -*- root: main.tex -*-
\section{Conclusion and Future Work}\label{sec:conclusion}

% For peer review papers, you can put extra information on the cover
% page as needed:
% \ifCLASSOPTIONpeerreview
% \begin{center} \bfseries EDICS Category: 3-BBND \end{center}
% \fi
%
% For peerreview papers, this IEEEtran command inserts a page break and
% creates the second title. It will be ignored for other modes.
\IEEEpeerreviewmaketitle



%TODO uncomment after at least one item is present
%\begin{thebibliography}{1}
%
%\end{thebibliography}

\bibliographystyle{IEEEtran}
\bibliography{biblio}



% that's all folks
\end{document}


