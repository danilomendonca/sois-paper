\section{Ad Hoc Edge Spaces}\label{sec:edge_spaces}

We present the main concepts and features of our framework using three different cases with distinct requirements. First, we show how our framework can enable situated applications to work without external storage and control provided by cloud infrastructure. Second, we present a more complex scenario in which low latency is the main requirement. Finally, we describe a mobile crowdsensing application, which combines different requirements, including efficient use of mobile data plan.

\subsection{Temperature Consensus: local communication and simple physical-virtual mapping}

In many places, it is often the case that co-workers have disagreements on the temperature to be set in the air-conditioning of the space they share. The ideal temperature for the environment varies from person to person, leading to conflicts among colleagues. To mitigate this problem, the temperature should be set to a value that corresponds to the average of the current preferences in the office. 

In the current example, each office delimits an edge space. These spaces are populated by the personal mobile devices from co-workers, which are assumed to be connected to the same corporative Wi-Fi. The first person to enter the office triggers the creation of a \textit{temperature control group}, which is a basic organization structure to which system roles are binded to. This device assumes the \textit{data managing role}, which is responsible for keeping the group state (members and their preferred temperature). Next devices to enter the room shall join the existing group. By taking the average of the preferred temperatures each time a device enters or leaves the room, a PUT command is sent to the REST microservice exposed by the air-conditioning controller, which in turn adjusts the temperature. 

The example above illustrates how an An Hoc Edge Space is mapped to the physical space of a company to provide a meaningful solution. The \textit{ad hoc} aspect becomes clear once there is no infrastructure for hosting or supporting the application besides the mobile devices themselves and the air-conditioner controller. As all workers leave the office with their devices, the edge space disappears. Also, the mapping between an office and the temperature control group provides a crispy correlation among physical and virtual spaces. One important aspect remains open: what happens if the data managing role leaves? 

The potential improvements to wireless communications proposed by 5G include a device-2-device communication that would allow devices within a given zone to form ad hoc networks and to exchange data directly with high throughputs. In a scenario such as a real-time mobile multiplayer game (MMG), such feature would allow the formation of ad hoc edge spaces in which devices would not only be able to update the game state with low latency, but also to use a client-server architecture in which one device is dynamically elected to play the server role, avoiding the well-known problems of a P2P architecture.  play these roles. In the above example, the first device assumed the data managing role (creator criteria). Once it leaves there are multiple criteria that can be considered by the election of the following manager, such as: the available computational resources -- battery, memory, etc -- the expected availability of the device in the system -- session time, mobility pattern, statistical data -- as well as other criteria from users and application. 

In the example, the data managing role keeps transient data about the preferred temperature level. To avoid fetching this data from all members and improve its availability, the data must be replicated by other devices. The replication degree may be fixed or proportional to the group size or volatility, i.e., the rate in which devices leave the group. For instance, a replication degree of 25\% would mean that an office with 8 devices shall have two replicas of the data managing role. When the active role quits the system, one of the two replicas should be activated. The role election and the realization of different role replication schemes at runtime is a central contribution of this work. As the criteria used for electing a role and its replica may require information from participants, it is important to consider the communication overhead in the trade off between availability and performance of the distributed election algorithm. The proposed election and replication mechanisms are fully described in Section~\ref{sec:model}.

%We show how a simple application based on our framework could mitigate such problem.

\subsection{Mobile Multiplayer Game: low latency, fairness and high availability}

The next ultimate gaming experience may follow the fusion between physical and virtual worlds, where the dimensions and assets of our physical reality are augmented with non tangible and fantasy elements from the virtual reality. This combination would allow young users (but not only) to go beyond the limitations of their houses and rooms and enjoy the outside and the physical company of other people while still been able to use the power of their digital equipments to extend their selves and their interactions with rich virtual elements provided by the game industry. To achieve this, interactions must have a low latency and last as long as possible, meaning an efficient use of devices battery. As an example, a

The potential improvements to wireless communications proposed by 5G include a device-2-device communication that would allow devices within a given zone to form ad hoc networks and to exchange data directly with high throughputs. In a real-time mobile multiplayer game (MMG), such feature would allow the formation of ad hoc edge spaces in which devices would not only be able to update the game state with low latency, but also to use a client-server architecture in which one device is dynamically elected to play the server role, avoiding the well-known problems of a P2P architecture. To achieve a fair use of resources, the server role should be rotated among existing candidates even before the active role have quit the system. 

While this solution would still rely on backend servers for processing and persisting game data, a pure cloud-based solution would imply a significant amount of data to be transfered back and forward, increasing latency and the use of mobile data plans. In addition to the election and replication of roles, the physical-virtual space mapping tends to be more coarse and fuzzy than previous example. For instance, a game may choose city parks as arena for users interaction. Accordingly, arena group membership may be specified by means of geolocation coordinates. To prevent battery drain, a GPS role can be assigned to fewer members and broadcasted using low energy bluetooth. Once devices are within the same arena group and a server role is elected, the game session may start.


\subsection{Crowdsensing: efficiency through collaboration}

The last example brings more complexity to the ad hoc edge space formation and maintenance problem. To illustrate it, let's consider a city in which public transport agency provides no real time information of the whereabouts of buses. However, a collaborative crowdsensing application has bee designed to collect this information from the GPS in passengers devices. Users participation is limited by battery consumption and mobile data plan usage. Next, we describe how the ad hoc edge spaces can help to improve these metrics when compared to a self-contained, cloud-based solution for the same problem.

The physical-virtual space mapping in this example is clear: each bus is a physical space where an ad hoc edge space may exist. In a cloud-based solution, each mobile device would host a self-contained application that would collect the GPS data and send it to the server. Instead, the first device to enter the bus creates a geolocation group. The possible roles are: data managing role, GPS fetcher role, and GPS aggregator. Here, replication is used not to only to improve availability, but also to increase the accuracy of geolocation data, as its quality varies in time from one device to another. Each fetcher sends its data to the single aggregator role, who validates the data and sends the resulting coordinates to the group data. Finally, geolocation data is sent to the backend servers to be further processed and shared to other users.

The benefits of using our framework in this example are: total GPS use is reduced; the quality of the data sent to backend servers is improved, while the quantity is reduce, thus reducing the use of the mobile data plan and the load of the backend servers. Considering the plethora of sensors in the smartphones, this could significantly improve the efficiency of mobile crowdsensing campaigns.