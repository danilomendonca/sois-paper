\section{Related Work}\label{sec:related_work}

The A-3 model~\cite{Baresi:2011:2} defined an architectural style consisting of groups that can be populated by a supervisor and its followers and composed with other groups. While this work share many of the motivations and have similarities with A-3, our model does not rely on the rigid \textit{supervisor-follower} structured, nor group compositions depend on shared members.
% features for topology control operations and 
In contrast with the more abstract A-3 support for self-adaptation~\cite{Baresi:2011:2}, we also investigate  concrete adaptation mechanisms that should guarantee the basic properties of groups, such as robustness, a high availability, efficient use of resources, and other attributes defined by the application through extension points. Finally, while A-3 rely on classical group communication methods, we investigate the integration of groups with tuple spaces for both inter-group and intra-group coordination.
%A3-TAG, a programming model that facilitates the design of self-adaptive distributed systems based on group abstractions~[CT]. A3-TAG is an extension of the A-3 model, which is used as the organization model. A-3 key elements are groups and two types of roles, namely supervisor and follower. Each group has a supervisor and a variable number of followers. The main differences between this work and A3-TAG are dual: first, our organization model imposes no restriction about the number of roles a group can have, nor it specifies any hierarchy between roles. Second, our adaptation model is not based on coordination groups, but rather on the direct interaction between groups based on self-organization principles. We argue that the complexity of forming and managing coordination groups may result in excessive overhead. Instead, we propose a \textit{structureless} method to self-adapt the organization structure.

Group and role abstractions has also been used in other domains. Ferber et al.~\cite{Ferber:2004} proposed an organization centered model for multi-agents systems that contrasts with agent centered models in which agents can communicate and interact freely. Among the problems of agent centered models, the authors cited security, modularity and the lack of support to other frameworks besides the multi-agent platform itself. In our work, we agree with those arguments as part of the justification for an organizational approach for distributed systems. Despite the model similarities, the works target different domains: instead of agents, we consider pervasive and mobile devices as the hosts of components that play roles in groups of distributed applications. %In addition, our work targets the particular problems of a high volatile and heterogeneous scenario that requires autonomous adaptation of the group-role organization.

 %Furthermore, we focus on the self-management of the group-based organization in the advent of context changes, while Ferber et al. have addressed only the conceptual elements of the organizational model.


Regarding data-centered models, Linda~\cite{Gelernter:1985} is a coordination language that allows processes to communicate through objects stored in and retrieved from a tuple space. While Linda provides a globally accessible and persistent tuple space, the Lime~\cite{Murphy:2006} model tackles host mobility with transient tuples which are carried by mobile nodes and can be merged or split according to their connectivity. The Lime model have more similarities with our work, as in both cases tuples are transient and kept by mobile hosts. Nonetheless, while in Lime tuples are private and will follow their hosting devices, in our work tuples are associated to groups and must outlive exiting members. As another data-centered solution, TOTA~\cite{Mamei:2003} also provides a distributed tuple space solution. Unlike previous models, tuples in TOTA are not associated with network nodes. Instead, they can be injected and propagated in the network following patterns. %TODO
In contrast with TOTA, our work associate tuple spaces with groups and group compositions to provide means for robust coordination among distributed components subject to hight degrees of volatility.

To be added: dynamic component ensembles; coalition formation;

%Kota et al.~[CT] have proposed a method for adapting the relationship between agents in a multi-agent system. In their work, agents reason about adaptation using only historical knowledge about past interactions and the cost of adaptation (meta-reasoning). Despite the similarities regarding the use of self-organization principles, our work differs in the organization model (groups, roles, and devices instead of agents and relationships) and also in the elements of the organization structure that are subject to adaptation (roles played by different devices in different groups instead of agents relationships). 

%Other works based on partitions target specific domains such as mobile ad hoc networks (MANET). As an example, ...


