%!TEX root = main.tex
% -*- root: main.tex -*-
\section{Introduction}
\label{sec:intro}
% target -> 1 page (with abstract)

The widespread diffusion of powerful mobile devices, such as smartphones and always-on wireless connectivity solutions as 3G and LTE, have radically changed the way we live our lives. Media consumption and sharing through the social Internet, have never been more engaging.

In this context innovative scenarios are emerging, like pervasive and context-aware applications. These applications are aware of the end-users surroundings, and behave accordingly to what is happening therein. To do this they rely on data that are obtained from the runtime environment itself, collected through appropriately deployed sensor networks. 

Since their inception modern mobile devices have included a plethora of onboard sensors, to enable applications such as navigation, gaming, and health monitoring. This means that mobile devices can become vehicles for the construction of new kinds of sensor networks, networks that can be deployed in unconventional ways to locations and contexts that were previously unconceivable. Moreover, the data that we collect through these networks are not limited by the onboard sensors; instead they could be data that are explicitly inputed into an app by the end-users. 

This novel data collection paradigm goes under the name of crowd sensing. Through crowd sensing we can gather invaluable information in disaster-recovery scenarios, we can perform geo-localized and time-limited sentiment analyses, and so on.

Crowd sensing introduces novel challenges. First of all, since the sensors are on mobile devices they are always on the move, and subject to high churn rates. Indeed, it is common for people to enter and leave the sensing application freely and unexpectedly. This can lead to robustness issues in which the required data cannot be collected at all times, due to the absence of a certain type of sensor. Second, the precision and quality of the collected data is vital. A single source of information can sometimes be misleading; indeed, the data it provides could be imprecise, or just simply incorrect. Multiple sources, on the other hand, allow us to implement various kinds of redundancy. We can have \emph{temporal redundancy}, in which multiple sensors, in a limited amount of time, are providing similar information  (e.g., that the temperature is rising). We can have \emph{spatial redundancy}, in which mulitple sensors, in a limited amount of physical space, are providing similar information (e.g., the the temperature is rising in a certain building). Finally, we can have \emph{dimensional redundancy}, in which multiple sensors, of different kinds, are providing valuable and relevant information (e.g., the temperature is rising as is the level of $CO_{2}$). Third, an important issue in crowd sensing is that the application will need to scale to support extensive sensor networks, due to the high amounts of physical devices that may be present within the application at any given time.

In this paper we advocate that there is a concrete need for a middleware for developing crowd sensing applications, one that can help developers cope with the more complex issues of crowd sensing applications without having to re-invent the wheel every time. To this end we provide A3Droid, a re-interpretation of the A-3 architectural style that specifically focuses on the needs of crowd sensing applications. 

A3Droid is based on the notion of \emph{group}. With A3Droid developers group devices in the sensor network to reduce the degree of dynamicity that is in the application. Indeed, while single devices can enter and leave the application unexpectedely, entire groups of devices are less likely to do so. Groups also allow us to more easily implement temporal, spatial, and dimensional sensor redundancy, as well as provide a concrete means to collect up-to-date snapshots of what devices are in the application at any given time. Finally, the groups in A3Droid can be composed and de-composed on the fly to ensure that the application scales according to the churn rate. 

The rest of the paper is structured as follows. Section~\ref{sec:a3} provides a rapid crash course on the A-3 architectural style. Therein we present the main abstractions that the architectural style provides to developers. Section~\ref{sec:a3droid} disusses in-depth details on how the A-3 architectural style was implemented for Android-based mobile devices in the form of the A3Droid middleware. Section~\ref{sec:evaluation} illustrates the experiments that were performed to assess A3Droid as a effective and efficient middleware for developing crowd sensing applications. Section~\ref{sec:related} discusses related work in literature, while Section~\ref{sec:conclusion} concludes the paper with some insight into what we intend to do in our future work in the area.











