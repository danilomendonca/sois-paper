%!TEX root = main.tex
% -*- root: main.tex -*-
\section{Introduction}
\label{sec:intro}
% target -> x pages (with abstract)

%context and motivation: in this paragraph I present the new scenario composed of heterogeneous devices

In the last decade, the world has witnessed the popularization of pervasive and smart devices of different kinds. Some of them are equipped with powerful CPUs, memory, and storage (e.g. modern smartphones and tablets), while others are designed for very specific functions or have more limited resources (e.g., sensors and gadgets). In common, these devices are able to communicate, which allow them to interact using standard protocols. This type of technology has been available to the general public with affordable prices and adopted at very large scales. According to predictions \cite{CITATION}, many more of these devices will join the \textit{internet of things} in the years to come. Such rich scenario enables the creation of distributed systems that have strong impact in peoples' lives, including social and smart city applications. Nonetheless, their realization require tackling some relevant challenges.


%problem introduction
An important challenge of this scenario is \textit{volatility}. As participants come and go for reasons like mobility, components churn create abrupt context changes that must be handled swiftly to avoid disruptions of the normal application behavior. 
%Depending on the role played by an exiting component, it must be resumed by other(s). 
In addition to churn, the \textit{heterogeneity} of the devices that can engage and be part of distributed and pervasive applications must also be considered. This variability has many sources. First, devices have different hardware components and configurations. Second, they are subject to distinct and dynamic policies for the use of their resources. Third, the use of resources by concurrent applications in the same device varies in time. Last but not least, changes in the physical world % or in the hosting platform 
can also affect the behavior of sensors and other components. As a consequence, the capabilities of different devices to perform system tasks vary from one to another and through time.

%For example, onboard sensors can be (de)activated by users or be affected by fluctuations in the physical world. 


%gap in the current solutions
Data-centric models, like the \textit{tuple spaces} in Linda~\cite{LINDA}, are the state-of-art of distributed components coordination. These models allow for both space and time decoupling. % since created data can outlive exiting nodes and be accessed by new nodes. 
Nonetheless, these models have a high abstraction level and do not address certain problems, like how the different components of a distributed system should be organized and operate when subject to high levels of volatility and heterogeneity. As such, the potential of data-centric models can be extended with further abstractions for the development of applications based on the plethora of pervasive devices available in everyday life situations. In specific, design and programming abstractions should guide the organization of distributed components according to application-specific criteria like the role they play in the system (modularization) and to optimize the flux of information among components and modules. Moreover, the system should be empowered with the methods and mechanisms for: 1) autonomously coping with a high level of volatility and heterogeneity; 2) efficiently employ the resources of pervasive devices, be they scarce or abundant.

%TODO complete the scale(lability) aspect and consequences
%as well as the resource allocation of pervasive devices.

%More than just reacting, a preventive approach should assign replacements to standby and quickly assume orphan tasks, thus improving availability. 

%solution introduction
%Some of the relevant problems that must be handled autonomously during execution are: how to organize the components interaction; how to perform the system composition; and how to distribute the responsibilities among them. The resulting solution must be scalable to address scenarios with a high number of devices; it must also be able to adapt to context changes and maintain normal system behavior.
%
%
%the problem in this paragraph refers to the ability of interacting locally
%Nowadays, the majority of mobile~\footnote{from this point forward the terms mobile, distributed, and pervasive are used interchangeably.} applications follow a client-server architecture in which backend servers in the cloud perform significant part of the computation. However, there is an increasing interest in enabling peer-to-peer interactions among devices through local and mobile ad hoc networks. 
%This approach has been referred by some as fog-computing~\cite{FOG}. 
%In some cases, low-latency is the main requirement that justifies a local interaction. In others, connectivity to the cloud is poor or cannot be assumed. Accordingly, in these cases devices are expected to interact locally to achieve some of the distributed application goals without the centralized coordination of backend servers. Regardless of the reason for creating distributed systems at the network edge, an adequate model should ease the complexness of its development and to enable a flexible, robust, and scalable solution. 
%In this paper we propose a model for the self-organization of distributed systems. The
In this paper, we propose a model that allows the creation of distributed systems based on the key concepts of groups and roles. These abstractions should guide the organization of the system into a set of partitions (groups) and to enable the distribution of the application logic within partitions (roles). In contrast with single and volatile components, groups provide a more stable entity that can be equipped with a tuple space for the coordination of its members, as well as a series of mechanisms for achieving solutions with high level of availability and robustness. These and other attributes can be achieved through the autonomous and distributed management of groups components and also through mechanisms of self-organization between different groups.


%The proposed self-organization method is implemented by a supporting middleware, which also provides essential features such as discovery, communication, distributed state management, and group operations. %We make use of a tuple space model tailored for mobile environment to support decoupled interaction between components while preserving data availability even in scenarios of high volatility.


%\begin{figure}[t!]
%	\centering
%	\includegraphics[width=\linewidth]{figures/fog_computing}
%	\caption{Fog computing as a complementary solution to cloud computing}
%	\label{fig:fog_computing}
%\end{figure}

%paper structure

This paper is structured as follows...

%At the application level, organization abstractions allows similar or different components running at different devices to form groups in which they can interact and coordinate their actions to achieve application goals. Second, at the organization structure level, the distributed application should autonomously adapt to events such as components failures, churn, environment changes, etc.


%suits applications for which global knowledge about the system or centralized adaptation is infeasible or prohibitive. 


%The premise followed by this work is that system partitions reduce the coordination space and improve the coordination scalability between distributed components. In different words, whenever a distributed system can be reduced into a set of partitions, doing so will reduce the coordination complexity. However, as changes are expected to happen, the partitioning must evolve to changes in order to keep the system operational and avoid violations of requirements. We argue that the formation and adaptation of the organization structure can benefit from a well defined model.

%we tackle the problem of designing distributed systems at the edge of the network infrastructure, i.e., composed of pervasive devices at local or ad hoc networks with a high degree of volatility. In specific, we propose a model that enables the partition of the system into a set of self-organizing, autonomous parts. The proposed model is based on the key concepts of \textbf{g}roups, \textbf{r}oles, and \textbf{d}evices (GRD). Groups are system partitions composed of roles. Roles implement the functionalities required to achieve system goals. Each role can be played by one or more devices in one or more groups. Groups can have different degrees of centralization, ranging from a client-server to a peer-to-peer architectural style, according to the application design. Additionally, to evolve the system in the occurrence of changes in its components and environment, we propose a method based on self-organizing principles, in which adaptation happens through local group interactions without any external control.
