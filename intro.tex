%!TEX root = main.tex
% -*- root: main.tex -*-
\section{Introduction}
\label{sec:intro}
% target -> x pages (with abstract)

%context and motivation: in this paragraph I present the new scenario composed of heterogeneous devices

In the last decade, the world has witnessed the popularization of pervasive and mobile devices of different kinds. Some of them are equipped with more powerful CPUs, memory, and storage (e.g. modern smartphones and tablets), while others are designed for specific functions and have limited resources (e.g., sensors and gadgets). In common, these cyber-physical devices are able to communicate and, depending on the application, are required to interact and combine their capabilities, exchange data, or coordinate their actions and activities. Today, the mainstream model for mobile computing consists of: 1) a client application running at users devices; 2) a backend application hosted by cloud servers. Accordingly, data produced by users and devices at the edge of the network is sent to the cloud, processed there and sent back to the edge. As good as this model has proven to be, there are existing and emerging use cases that cannot afford the latency introduced by networking with distant servers; others cases cannot assume a stable and reliable connection with remote servers or users are not willing to pay for the costs of data transmission, specially if part of this data can be processed and shared locally. 

%Among the advantages of this model are the client-server architecture simplicity in which client applications communicate with servers and between themselves through servers.

%and meet requirements such as low latency, availability, reliability and efficient use of mobile data plans.

In order to enable a more distributed architecture and meet communication, coordination, and collaboration requirements, application nodes must be able to perceive each other's presence and each other's state and capabilities (discovery and contextualization). Also, interacting nodes should form an organization structure that defines the responsibilities and behavior of distinct nodes. As mobile devices are expected to come and go (churn) and are subject to changes that affect their capabilities and conditions (volatility), such organization structure must evolve to the different physical and social contexts the application nodes are exposed to (self-organization). 

In this paper, we introduce the concept of \textit{Self-Organizing Edge Spaces} as part of a framework for the engineering of mobile applications. We describe how some important application requirements can be met through local and opportunistic interactions using nowadays technologies for wireless networking and the features to be provided by 5G, such as device-to-device communication. Firstly, we introduce the concept of \textit{Edge Spaces}, which consists of the interaction space formed by the application nodes willing to exchange data, coordinate their actions, or collaborate to achieve common goals more efficiently. Then, we provide the modeling and programming abstractions for the engineering of mobile applications based on this interaction space. Finally, we present the self-organization mechanisms, which take into account the volatility, heterogeneity, and large scale that characterizes the targeted mobile applications. The framework was evaluated with a real case opportunistic crowdsensing application for Android platform. The results show the feasibility of the approach, as nodes were able to coordinate their sensing activities locally and to improve efficiency by allocating responsibilities according to the context of devices.



%Examples of data-intensive applications for which one or more of the aforementioned restrictions apply are mobile crowdsensing applications~\cite{MCSC} -- e.g., air/noise pollution measurement and traffic dynamics -- and situated social applications for which user interactions are based on proximity and locality -- e.g., autonomous vehicles and mobile gaming applications~\cite{MOBILE_USE_CASES}. 


%the assignment of these asymmetric roles is a central feature of the interaction spaces management. 


%For instance, some nodes may play key roles such as a group leader, supervisor, coordinator or local server; and some application roles may be assigned to specific nodes -- e.g., those which have the capability and the best conditions to perform tasks like sensing the environment in mobile crowdsensing. Thus, to setup a self-organizing space in which application nodes assume managing and application roles 

%However more distributed than a pure client-server architecture, local interaction spaces can still rely on some degree of centralization. For instance, some nodes may play key roles such as a group leader, supervisor, coordinator or local server. In addition, some application roles may also be distributed to specific nodes -- for instance, those which have the capability and the best conditions to perform tasks like sensing the environment in opportunistic crowdsensing. Finally, as devices are expected to come and go and are subject to changes that affect their capabilities and conditions, the assignment of these asymmetric roles is a central feature of the interaction spaces management. 

%In the literature and since before the advent and popularization of cloud computing, there has been a long lasting research effort to develop the models, methods, algorithms, middlewares and tools for pervasive and mobile computing. From grid computing and peer-to-peer communication protocol to mobile multi-agents and distinct coordination models.
%% that provides decoupling from space and time.
%. %There has also been a significant effort to address the formation of coalitions between multi-agents and multi-robots according to their dynamic capabilities. 
%Despite the wideness of these fields, they are all characterized by the multiplicity of the entities involved, or ensembles. %, usually targeting the same goals. 
%Considering today's and the near future world in which cyber-physical devices are everywhere and carried by everyone, will the cloud be able to solve every computational problem and to proxy all interactions? If not, could existing and new models help to achieve the whole potential of pervasive and mobile computing?


%it can seamlessly cope with cloud services while latency is reduced and less data has to be transmitted and processed at the cloud.

%%available to the to the general public with affordable prices and 
%This technology has been adopted at very large scales. According to predictions \cite{IHS:2016}, many more of these devices will join the \textit{internet of things} in the years to come. Such rich scenario enables the creation of systems that have strong impact in peoples' lives, such as social and smart city applications. Nonetheless, their realization require tackling the emerging problems.

%%problem introduction
%An important challenge in this scenario is \textit{volatility}. As mobile devices come and go for many reasons, components churn create abrupt context changes that must be handled swiftly to avoid disruptions of the normal application behavior or violations of non-functional requirements. 
%%Depending on the role played by an exiting component, it must be resumed by other(s). 
%In addition to churn, the \textit{heterogeneity} of the devices that can compose the application must also be considered. Such heterogeneity has many sources. First, hosting devices have different hardware components and configurations. Second, they are subject to distinct and dynamic policies for the use of their resources. Third, the use of resources by concurrent applications varies in time. Last but not least, changes in the physical world % or in the hosting platform 
%can also affect the behavior of sensors and other components. As a consequence, the capabilities of different devices to perform system tasks vary from one to another and through time.

%For example, onboard sensors can be (de)activated by users or be affected by fluctuations in the physical world. 


%%gap in the current solutions
%In the last decades, data-centric models, like the \textit{tuple spaces} proposed in Linda~\cite{Gelernter:1985}, are the state-of-art of distributed components coordination. These models allow for both space and time decoupling among components. % since created data can outlive exiting nodes and be accessed by new nodes. 
%Nonetheless, they have a high abstraction level and do not address how system components should prevail in situations with high levels of volatility and heterogeneity. While some proposals have dealt with the mobility of hosts~\cite{Murphy:2001}\cite{Mamei:2003}~\cite{Murphy:2006:2}, they do not explore further abstractions and mechanisms 

%Also, they assumed an agent-centered perspective with no \textit{supra-agent} abstractions. 

%%Q: Now it is the time to differentiate my work from what 

%In contrast with these proposals, other coordination models have adopted organizational abstractions~\cite{Baresi:2011, Baresi:2011:2}, but have not addressed the problem of the autonomous adaptation of the organization in the advent of churn and other context changes.

%but without relying on a data-centered solution for distributed coordination.



%In this paper, we propose a framework based on the key concepts of \textit{groups} and \textit{roles}. In contrast with previous works that focus on the interaction among single -- and here considered volatile -- agents/components, we give special importance to what can be achieved by collections. In accordance with an organizational perspective, we use the term role to define a component with a set of capabilities responsible for one or more system functionalities. Also, groups are seen not as implicitly sets formed and dissolved as components register and unregister their interest in a given event or data, like in event/data-centered models, 
%%enter and leave the system, 
%but explicitly and more stable entities that can outlive components churn by means of replication, redundancy, and dynamic reallocation of existing resources.

%While cloud-computing has successfully enabled the outsourcing of computation to remote providers, this model may not provide the low latency required by some applications or may not scale to a very high number of components~\cite{FOG}. Conversely, the resource limitations of devices reduce their capability to solve complex computational problems. As part of our framework, we propose a method for the adaptation of the group-role structure that respect such constraints. %For this, we adopted the principles of self-organizing systems for which a given property or behavior emerges from the sole interaction between components and their local knowledge~\cite{}. 
%In specific, we propose the autonomous assignment (membership) of resources (roles) into partitions (groups) according to the context of these devices and to high-level application requirements -- expressed in terms of grouping criteria, roles cardinality, and non-functional attributes. 
%
%should emerge from the decisions of distinct nodes to join, stay or leave the existing groups.


%dynamic assignment of system roles to groups of heterogeneous and volatile devices (nodes), according to the 



%For this, the group-role structure should reflect what is been required from each group and what can be provided by each involved device.

%, we target the dynamic and autonomous assignment of system roles to groups of heterogeneous and volatile devices (nodes), according to the context of these devices and to high-level application requirements -- expressed in terms of grouping criteria, roles cardinality (direct specification), and non-functional attributes (indirect specification). 

%As such, groups are seen not as implicitly sets formed and dissolved as components register and unregister their interest in a given event or data,
%enter and leave the system, 
%but more enduring and stable entities that can outlive components. 

%In accordance with an organizational perspective, we use the term role to define a component with a set of capabilities responsible for one or more system functionalities. We also propose the integration of these abstractions a data-centered model for decoupled intra-group and inter-group coordination. The contribution of the model is twofold, as its features and properties can be employed by the managed system (distributed application) and also by its managing system.


%Figure~\ref{fig:motivation} presents two types of criteria for grouping distributed components.

\begin{figure*}[t!]
    \centering
    \begin{subfigure}[b]{0.4\textwidth}
        \centering
        \includegraphics[width=1\textwidth]{figures/physical_view}
        \caption{Physical partitioning: application components (white circles) grouped by their position in the physical space (blue shapes).}
    \end{subfigure}%
    ~ 
    \begin{subfigure}[b]{0.4\textwidth}
        \centering
        \includegraphics[width=1\textwidth]{figures/logical_view}
        \caption{Logical partitioning: application components (white circles) grouped by logical criteria defined by the application.}
    \end{subfigure}
    \caption{Different grouping criteria.}
    \label{fig:motivation}
\end{figure*}




%Groups as more stable containers in which a collection of components (\textit{roles}) can be dynamically deployed and coordinate their activities through the \textit{group tuple space}. While the application roles should compose the distributed logic of the \textit{managed} system, \textit{managing} roles should be responsible for adapting the managed system in the advent of context changes. To accomplish this, we explore both intragroup and intergroup adaptations methods.

%hat can be equipped with a tuple space for the coordination of its members and a variety of mechanisms for robustness.

%In this paper, we propose a framework for the creation of distributed systems based on the key concepts of groups and roles. These abstractions should guide the organization of the system into partitions (groups) in which distributed components (roles) can be dynamically deployed. 
%and to enable the distribution of the application logic within these partitions (roles). 


%achieving solutions with a high level of robustness, availability, and other attributes.

%Q: What is the difference between a partition and a collection?

%that can be achieved through the autonomous and distributed management of groups components and also through mechanisms of self-organization between different groups.

%As such, the potential of data-centric models can be further extended with abstractions for the development of applications based on the plethora of pervasive devices available in everyday life situations. In specific, design and programming abstractions should guide the organization of distributed components according to application specific criteria like the role they play in the system (modularization) and to optimize the flux of information among components and modules. Moreover, the system should be empowered with the methods and mechanisms for: 1) autonomously coping with a high level of volatility and heterogeneity; 2) efficiently employ the resources of pervasive devices, be they scarce or abundant.

%TODO complete the scale(lability) aspect and consequences
%as well as the resource allocation of pervasive devices.

%More than just reacting, a preventive approach should assign replacements to standby and quickly assume orphan tasks, thus improving availability. 

%solution introduction
%Some of the relevant problems that must be handled autonomously during execution are: how to organize the components interaction; how to perform the system composition; and how to distribute the responsibilities among them. The resulting solution must be scalable to address scenarios with a high number of devices; it must also be able to adapt to context changes and maintain normal system behavior.
%
%
%the problem in this paragraph refers to the ability of interacting locally
%Nowadays, the majority of mobile~\footnote{from this point forward the terms mobile, distributed, and pervasive are used interchangeably.} applications follow a client-server architecture in which backend servers in the cloud perform significant part of the computation. However, there is an increasing interest in enabling peer-to-peer interactions among devices through local and mobile ad hoc networks. 
%This approach has been referred by some as fog-computing~\cite{FOG}. 
%In some cases, low-latency is the main requirement that justifies a local interaction. In others, connectivity to the cloud is poor or cannot be assumed. Accordingly, in these cases devices are expected to interact locally to achieve some of the distributed application goals without the centralized coordination of backend servers. Regardless of the reason for creating distributed systems at the network edge, an adequate model should ease the complexness of its development and to enable a flexible, robust, and scalable solution. 
%In this paper we propose a model for the self-organization of distributed systems. The



%The proposed self-organization method is implemented by a supporting middleware, which also provides essential features such as discovery, communication, distributed state management, and group operations. %We make use of a tuple space model tailored for mobile environment to support decoupled interaction between components while preserving data availability even in scenarios of high volatility.


%\begin{figure}[t!]
%	\centering
%	\includegraphics[width=\linewidth]{figures/fog_computing}
%	\caption{Fog computing as a complementary solution to cloud computing}
%	\label{fig:fog_computing}
%\end{figure}

%paper structure

The paper is structured as follows. Section~\ref{sec:related_work} compares this proposal with related works. Section~\ref{sec:model} presents the framework. Section~\ref{sec:evaluation} reports on the evaluation of the model and discuss its results. Finally, Section~\ref{sec:conclusion} concludes the paper with final considerations and the future work.

%At the application level, organization abstractions allows similar or different components running at different devices to form groups in which they can interact and coordinate their actions to achieve application goals. Second, at the organization structure level, the distributed application should autonomously adapt to events such as components failures, churn, environment changes, etc.


%suits applications for which global knowledge about the system or centralized adaptation is infeasible or prohibitive. 


%The premise followed by this work is that system partitions reduce the coordination space and improve the coordination scalability between distributed components. In different words, whenever a distributed system can be reduced into a set of partitions, doing so will reduce the coordination complexity. However, as changes are expected to happen, the partitioning must evolve to changes in order to keep the system operational and avoid violations of requirements. We argue that the formation and adaptation of the organization structure can benefit from a well defined model.

%we tackle the problem of designing distributed systems at the edge of the network infrastructure, i.e., composed of pervasive devices at local or ad hoc networks with a high degree of volatility. In specific, we propose a model that enables the partition of the system into a set of self-organizing, autonomous parts. The proposed model is based on the key concepts of \textbf{g}roups, \textbf{r}oles, and \textbf{d}evices (GRD). Groups are system partitions composed of roles. Roles implement the functionalities required to achieve system goals. Each role can be played by one or more devices in one or more groups. Groups can have different degrees of centralization, ranging from a client-server to a peer-to-peer architectural style, according to the application design. Additionally, to evolve the system in the occurrence of changes in its components and environment, we propose a method based on self-organizing principles, in which adaptation happens through local group interactions without any external control.
