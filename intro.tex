%!TEX root = main.tex
% -*- root: main.tex -*-
\section{Introduction}
\label{sec:intro}
%context and motivation: in this paragraph I present the new scenario composed of heterogeneous devices

%TODO Bring Internet of Things to the paragraph and make mobile devices and applications more important; 

In the last decade, the world has witnessed the popularization of pervasive and mobile devices of different kinds. Some of them are equipped with more powerful CPUs, memory, and storage (e.g. modern smartphones and tablets), while others are designed for specific functions and have limited resources (e.g., sensors and gadgets). In particular, applications hosted by mobile devices are increasing in number and complexity. Today, the mainstream architectural model for mobile applications consists of: 1) a client application running at users devices; 2) a backend application hosted by cloud servers. Accordingly, data produced by these applications at the edge of the network is sent to the cloud, processed there and sent back to the edge. As good as this model has proven to be, there are existing and emerging use cases that cannot afford the latency introduced by networking with distant servers; others cases cannot assume a stable and reliable connection with remote servers or users are not willing to pay for the costs of intensive data transmission, specially if part of this data can be processed and consumed locally. 

In order to explore the potential of a more distributed architecture and, without intermediation of remote servers, to improve the user experience of mobile applications for which interaction among nodes is a core requirement, application nodes must be able to set up and organize an interaction space in which they can recognize each other's presence and interact according to the goal they need to achieve. In its simplest form, interaction targets the exchange of application data -- like in a chat or in a mobile multiplayer game session. However, given the capabilities of mobile hosts to sense and actuate the environment, application nodes are eligible for performing autonomous tasks that require coordination -- such as in mobile crowdsensing. Finally, some applications explore the collaboration among different nodes to improve efficiency and other attributes -- like in opportunistic routing applications designed to work without Internet.

In this paper, we introduce the concept of \textit{self-organizing edge spaces} as part of a framework for the engineering of interactive mobile applications. We describe how some important application requirements can be met through local and opportunistic interactions using nowadays technologies for wireless networking and the features to be provided by 5G, such as device-to-device communication. Firstly, we introduce the concept of edge spaces, which consists of the interaction space formed by the application nodes willing to exchange data, coordinate their actions, or collaborate. Then, we present the self-organization mechanisms, which take into account the volatility, heterogeneity, and large scale that characterizes the targeted mobile applications. Finally, we provide the modeling and programming abstractions for the engineering of mobile applications based on this paradigm, including a lightweight application programming interface and middleware for Android platform devices. The framework was evaluated with an opportunistic crowdsensing application. The results show the feasibility and advantages of the approach, as nodes were able to coordinate their sensing activities locally, reduce overall battery consumption and reduce the volume of data sent to backend servers.

The paper is structured as follows. Section~\ref{sec:related_work} compares this proposal with related works. Section~\ref{sec:model} presents the framework. Section~\ref{sec:evaluation} reports on the evaluation of the model and discuss its results. Finally, Section~\ref{sec:conclusion} concludes the paper with final considerations and the future work.
