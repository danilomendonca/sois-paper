%!TEX root = main.tex
% -*- root: main.tex -*-
\section{Introduction}
\label{sec:intro}
%context and motivation: in this paragraph I present the new scenario composed of heterogeneous devices

%TODO: Bring Internet of Things to the paragraph and make mobile devices and applications more important; (deprecated)

%1: The scenario of pervasive and mobile computing today
In the last decade, the world has witnessed the massive popularization of pervasive and mobile devices. Some of them are equipped with more powerful CPUs, memory, and storage (e.g. modern smartphones and tablets), while others are designed for specific functions and have limited resources (e.g., sensors and gadgets). In particular, applications hosted by these devices are increasing in number and complexity. 

\begin{figure}[t!]
	\centering
	\includegraphics[width=0.75\linewidth]{figures/pervasive_devices}
	\caption{Different classes of pervasive devices; a) a laptop; b) a Raspberry PI; c) a smartphone; d) a tablet; e) a gadget (watch); f) a Bluetooth beacon}
	\label{fig:pervasive_devices}
\end{figure}

%2: Today's mainstream architectural model
Today, the mainstream architectural model for pervasive applications consists of: 1) a client application running at users devices; 2) a backend application hosted by cloud servers. Accordingly, data produced by these applications at the edge of the network is sent to the cloud, processed there and sent back to the edge. As good as this model has proven to be, there are existing and emerging use cases that cannot afford the latency introduced by networking with distant servers, cannot assume a stable and reliable connection with remote servers, or users are not willing to accept the costs of remote communication, specially if part of the data can be processed and consumed locally.

%3: Examples of P2P based solutions
%TODO: rethink the purpose of this paragraph and fix it
In its simplest form, a device-to-device interaction enables the exchange of application data. However, novel applications are targeting more complex and sophisticated goals (e.g., to autonomously sense the physical environment in specific areas of a city). In this paper, we argue that some of these goals can be achieved more efficiently through the direct interaction and collaboration between the application nodes. %In specific, we target the serendipitous situations in which the devices hosting these applications become visible to each other and are able to communicate. 

%3: Functional plasticity + relations between nodes
In specific, given the functional plasticity of today's pervasive and mobile devices equipped with more powerful computational resources and several types of sensors, 
%we argue that 
%sense and actuate the environment, or even to perform computation-intensive tasks,
new types of relation -- others than just the one betwenn clients and backend servers -- can be opportunistically created and dissolved among devices at the edge of the network. These relations must cope with the heterogeneity and volatility characterizing the pervasive ecosystem, including the mobility of nodes and the fluctuations of their connectivity and resources. 



%the distribution of functional responsibilities (roles) among application nodes can be asymmetric, with some nodes playing other than just the role of a client of a backend server. Among others, some nodes may become coordinators and control the activities of other nodes; some may relay messages from the server to its local peers or, inversely, some may become responsible for aggregating data from its local peers before sending a preprocessed version to the server. To this end, nodes must agree on the roles they perform based on their capabilities and state. 

%with the popularization of miniaturized platforms able to provide general purpose computing (e.g., Raspberry PI~\footnote{}), new kinds of relations between the devices composing the pervasive ecosystem can be opportunistically created and dissolved.



%this \textit{interactive} applications can be achieved by 

%In order to explore the potential of a more distributed architecture and, without intermediation of remote servers, to improve the user experience of mobile applications for which interaction among nodes is an important requirement, application nodes must be able to set up and manage an interaction space in which they can interact and behave according to their individual and social contexts. 



% more complex use cases in which application tasks are resource demanding or must be performed without user participation have emerged, 

 %-- such as in mobile crowdsensing. Finally, some applications explore the collaboration among different nodes to improve efficiency and other attributes -- like in opportunistic routing applications designed to work without Internet.

%4: Contributions
\textbf{Contribution of the work:} 
%4.1: SOIS concepts
in this paper we introduce the concept of \textit{self-organizing interaction spaces} (SOIS) as part of a framework for the engineering of pervasive applications. 
%that require or benefit from the interaction between its nodes. 
%4.1a: The SOIS rationale
Firstly, we discuss how some existing and emerging applications goals, empowered by nowadays technologies for wireless networking and the functional plasticity of its hosting devices, can be met by letting nodes to assume distinct functional roles in the dynamic pervasive ecosystem organization. 
%4.1b: The self-organization mechanisms
Secondly, we provide the organization abstractions for modeling and programming these applications. Finally, we propose the self-organization mechanisms to maintain and evolve the ecosystem organization, which take into account the social and the individual contexts of hosting devices in the formation of groups and assignment of roles to the  distributed application nodes.
%application nodes capabilities and state in the formation and adaptation of the ecosystem organization. 
%Finally, we present the requirements for a middleware to support the programming and operation of pervasive applications based on SOIS.

The feasibility of the framework has been demonstrated with case examples of pervasive applications and the complexity of the self-organization mechanisms evaluated by means of asymptotic analysis and a simulation of a mobile crowd-sensing application.
%was evaluated with a simulation of mobile crowdsensing application. %The results showed the feasibility of the approach and, 
In comparison with a pure client-server model, the SOIS based solution reduced the overall battery consumption in X\% and the volume of data sent to backend servers in Y\%, while preserving the low-latency requirements.

%5: How the paper is structured
The paper is structured as follows: Section~\ref{sec:background} brings relevant background information; Section~\ref{sec:overview} and the application example used to motivate and illustrate the framework, relating it to the characteristics of targeted applications. Section~\ref{sec:edge_spaces} presents the organization abstractions and the self-organization mechanisms composing the framework. Details about these mechanisms are provided in Section~\ref{sec:self_organization}, whereas Section~\ref{sec:evaluation} reports on the evaluation of our framework with a mobile crowdsensing application. Finally, Section~\ref{sec:related_work} compares SOIS with related works and Section~\ref{sec:conclusion} concludes this paper with final considerations and future works.
