\section{Self-organization}\label{sec:self_organization}


%While self-organization principles have been extensively used in the context of multi-agents and robots, as well as in other fields, they have mostly addressed the achievement of a specific set of application goals (e.g., in the design of rescue robots that, based on their local knowledge, actions, and interactions, will lead to the emergent behavior of finding and rescuing the victims). In contrast, in this work self-organization means the literal self-adaptation of the system organization. 

%Additionally, bio-inspired and other self-organization methods have a strong relation with the restrictions imposed by the physical world. For instance, the actions of a robot are based on its own knowledge and the knowledge it receives from its neighbors. In many cases, this is due to the limitation in the communication methods. The concept of a neighbor is, in this case, determined by spatial proximity. However, if the communication method is powerful enough to let all robots to exchange information, the concept of a neighbor cannot be determined by the spatial vicinity.

The structure of the application organization
%, formed by the two abstractions presented -- namely, roles and groups -- 
must be malleable (plastic) to accommodate changes in its ecosystem composition and environment. In contrast with the much less dynamic cases of organizations in human societies (e.g. industry and military organizations), the volatility of pervasive and mobile devices, caused by their mobility or fluctuations of their resources, may imply the formation or dissolution of relations among nodes and require the reassessment of the roles they play in the organization. 

While self-organization have been extensively used in the context of multi-agents and other fields, they have mostly addressed the achievement of system goals by means of actions and interactions to be performed by individuals based on local knowledge and without external control~\cite{}. In contrast, in this work we propose autonomous mechanisms to adapt the organization the application ecosystem, i.e., to let application nodes to agree on their role in the system according to what is required from the system and the context these nodes perceive themselves in. In specific, we propose a mechanism for each of the following dynamics:


%The long term evolution of the system has also been studied 

%As an example, rescue robots are designed to decide where to go next based on their own sensors or the knowledge received from other robots they meet~\cite{}. will lead to the emergent behavior of finding and rescuing the victims). In contrast, in this work self-organization means the literal self-adaptation of the system organization. 

%The term \textit{self-organization} have been mostly used to refer to emergent behaviors of the system based on local knowledge and actions~\cite{SELF_ORGANIZATION^4}. In this work, it denotes the distributed and autonomous adaptation of the system organization in the advent of context changes, i.e., it literally means \textit{self-organization} of the ecosystem formed by the application nodes. Nonetheless, the self-organization mechanisms proposed here share many of the characteristics of ....:

%\begin{itemize}
%	
%	\item \textbf{No external control:} no external component participates in the decision making to adapt the organization;
%	
%	\item \textbf{Distributed decision making:} ... 
%	
%\end{itemize}

%In this sense, each application node takes part in the process of perceiving its individual context (availability of resources, capabilities), as well as its social context (with who the node can interact). In this paper, we focus on two different self-organization mechanisms:

\begin{itemize}
	
	\item \textbf{Group membership:} a node may join or leave a group according to its satisfaction to the group membership criteria; to address this, a \textit{grouping} mechanism is proposed;
	
	\item \textbf{Role election:} the nodes within a group must agree on which roles they shall perform; to address this, a  \textit{role election} mechanism is proposed;
	
	\item \textbf{Role cardinality:} the relation between the number of instances of a role and attributes of system may vary with the context; to address this, a \textit{role cardinality}  mechanism is proposed;
	
	\item \textbf{Group cardinality:} finally, in some situations the membership criteria may have to be relaxed to allow more nodes to join a group and achieve its goals; to address this, \textit{group cardinality} mechanism is proposed.
	
\end{itemize}


\subsection{Self-grouping} 

%mobile crowdsensing campaigns may target specific city areas; thus, only the nodes which are currently within these areas are eligible for participating of the respective campaigns. Campaign groups can be modeled with a physical criteria specifying the geolocation coordinates of the area (Figure~\ref{fig:spatial_criteria}). Also, specific sensing activities may also correspond to groups using the hardware membership criteria (Figure~\ref{fig:virtual_criteria}).


%TODO: replace the figures bellow with a MC example

\subsubsection{\textbf{Definition}} As the idea of a group in this work is not related to security, the \textit{self-grouping} mechanism is not controlled by special-purpose components external to the system; instead, each node is responsible for checking its own satisfaction to the existing membership criteria in the application organization. Whenever the satisfaction of a group membership criteria changes, this event must be  advertised to other nodes, who update their group membership registry. 

\subsubsection{\textbf{Grouping Method}} Nodes entering the application have no registry of the existing group membership. They must acquire this information from the members of the groups they eventually join for the first time. In specific, we defined the oldest member of a group to be responsible for sending the registry replica to the newcomer. The complete self-grouping procedure is further depicted by the diagram in Figure~\ref{fig:self_grouping}.
%TODO despict how the problem of time coupling by using gossip or other algorithms for advertisement/discovery
 
%TODO: update the diagram
\begin{figure}[t!]
	\centering
	\includegraphics[width=0.45\textwidth]{figures/join_or_create}
	\caption{Activities of the self-grouping procedure performed by each application node}
	\label{fig:self_grouping}
\end{figure}

%TODO: describe which nodes is responsible for sending this information

\subsubsection{\textbf{Complexity Analysis}} We analyze the complexity of the grouping method by means of the communication overhead, i.e., the number of messages exchanged between nodes and their payload size. The analysis is divided in two parts: a) the overhead when a node joins/leaves a group (registry update); b) the additional overhead when a node joins a group (registry copy).

In the worse case scenario, registry update (a) takes $n-1$ unicast messages, ($O(n)$), with $n$ the group size, and a single registry line as payload. However, if a broadcast communication is used, a single broadcast message can advertise the registry update ($O(1)$).
%This is the case, e.g., of an UDP broadcast over the 802.11 network protocol (Wi-Fi). 
The registry copy (b), in its turn, requires a single unicast message to be transmitted each time a node joins a group. In contrast with the previous type of message, the registry copy includes information about all $n-1$ nodes in the group. Thus, the total overhead is given by $O(p)$.

%TODO: decide if the group registry must be sent to newcomers only for the first time they join the group (which implies to continue updating them while they are outside the group). R: always send the the group registry to newcomers
 
\subsection{Distributed Role Allocation} 
 
%TODO: update the diagram
\begin{figure}[t!]
	\centering
	\includegraphics[width=0.45\textwidth]{figures/role-assignment}
	\caption{Activities of the self-grouping procedure performed by each application node}
	\label{fig:self_grouping}
\end{figure}
 
\subsubsection{\textbf{Definition}} The decision of which node(s) should be assigned to which roles(s) may depend on many aspects. Trivially, any node capable of performing a role is a potential candidate. Notwithstanding this, attributes such as the quality of the data tend to variate from one node to another and throughout time. Also, devices have different levels of battery available. Thus, a balanced role allocation must respect the trade off between what is best for the application goals and for the individual devices. 

%In specific, we adopt the definition of context as...+++..., including their physical and the social contexts; in specific, context is reified as:
%
%\begin{itemize} 
%	
%	\item which application nodes are currently present (provided by existing discovery mechanisms);
%	
%	\item the capability and fitness of these nodes to perform functionalities required by the application (addressed by this framework).
%	
%\end{itemize}

%Despite the many possible types of asymmetric tasks, in this paper we focus on those whose importance for the application and cost for the devices are significant, i.e., the computation and communication overhead for its allocating is justified. 


%The later is specially important in volunteer based applications such as mobile crowdsensing, as efficient and fair use of devices resources works as an incentive for participation. 

%FIGURE: META MODEL FOR NODE-ROLE

%\textbf{Requirements} As the we consider a scenario in which application nodes hosted by volatile and resource constrained devices must self-organize their activities, the allocation method shall satisfy the following requirements:
%
%\begin{enumerate}[label=R\arabic*.]
%	
%	\item Distributed: allocation decision should not be centralized by one device in order to avoid bottlenecks;
%	
%	\item Communication overhead: only essential data should be exchanged in order to minimize communication overhead; 
%	
%	\item Coherence: the allocation method performance/cost should be tuned according to the role constraints and criticality.
%	  
%\end{enumerate} 

%TODO: Move this to the related works section;
%TODO: Add heterogeneity and volatility to the paragraph
In the literature, many works have tackled the problem of distributed task allocation~\cite{DTA}. In contrast with a task, a functional role defines a set of functionalities (possibly tasks) that a member of an organization is responsible to provide (perform). Hence, within an organization, a \textit{role} precedes a \textit{task}. Then, depending on the type of role, if multiple instances of a role have been assigned, a task allocation among these instances may still take place. Last but not least, while tasks usually have a concrete criteria for their beginning and completion and their assignment happens before task execution, roles lifespan tend to include multiple repetitions of a given functionality (or task). Therefore, in a dynamic scenario, a role assignment may have to evolve meanwhile roles are been performed.

Notwithstanding their differences, the two types of allocation problems share commonalities. For instance, in both cases an utility function may be used as a criteria for choosing an optimal or sub-optimal assignment of roles/tasks. Whereas the optimization of quality attributes may deem unfeasible due to its complexity, a sub-optimal allocation of can still be guided by the \textit{fitness} (or utility) of nodes in performing these tasks/roles. To this end, fitness/utility is modeled as real-value function of relevant features affecting one or more attributes of the application. Also, some of the existing taxonomy for classifying a task allocation problem can also be applied to the role allocation problem. 

%Thus, our method for role allocation follows the classification and algorithm from existing works.

\subsubsection{\textbf{Classification}} Gerkey and Matarić’s~\cite{} proposed a taxonomy for the classification of multi-robot task allocation problems along three axes. In the first axis, robots are categorized into single-robot versus multi-robot tasks%\footnote{Originally, the taxonomy refers to \textit{robots} and \textit{tasks}, while in this paper we refer to the (application) \textit{nodes} and the \textit{roles} they can perform.}. 
In the second axis, roles are categorized into two types: single-node vs multi-node roles. Finally, in the third axis, the allocation is also categorized into two types: instantaneous assignment or time-extended assignment.

As application nodes are generally capable of performing more than one role at a time -- for instance, to fetch from multiple sensors -- nodes are considered as multi-role. While certain types of role are performed by a single node -- for instance, the sensor data aggregator -- there are cases in which a role must be performed by multiple nodes. Thus, node cardinality depends on the role type. Due to the volatility of mobile devices (churn and capability changes), the scheduling of future allocations tends to fail. Accordingly, we consider an instantaneous assignment of roles based on the context of the involved devices.

\subsubsection{\textbf{Allocation Method}} Auction-based allocation methods have been extensively studied in multi-agents/robots domain. [describe an auction-based allocation method].

In comparison with the task allocation problem tackled by auction-based methods, an assignment of roles to pervasive and mobile devices subject to high volatility needs to evolve as nodes leave or join the system and their fitness change. However, if the assignment should continuously reflect any context change, a frequent message exchange between bidding and auctioneers would lead to excessive communication overhead. Accordingly, the replacement of existing role positions (herein referred as reelection) should take into account the trade off between the gain of having a more fit node elected and the replacement cost. 

%To mitigate this problem, nodes should evaluate their fitness (self-evaluation) and only advertise it to other group members if a delta in the value occurs.

Before presenting the methods for the election of vacant and occupied role positions, we first elicit the kind of events that require an election process:

\begin{itemize}
	
	\item \textbf{Vacancy:} a new role position is opened or the node playing a role exists the system (churn) due to a network disconnection from the remaining nodes or the abnormal termination of the application instance after a failure;
	
	\item \textbf{Resignation:} the node playing a role calls for a new election; e.g., the application node is exiting the system following a termination command issued by the users or the operational system;
	
	\item \textbf{Challenge:} one of eligible nodes calls for a new election; e.g., an eligible node has a significantly higher fitness for that role position than the actual node at the time it was elected;
	
	%\item \textbf{Consensus:} the set of nodes dependent from a functionality provided by an elected node calls for a new election; e.g., the latency experienced between a set of nodes and an elected node is high despite its higher fitness to play that role (latency is not part of the fitness criteria);
	
\end{itemize}
\medskip
 
The subtitle difference between vacancy and resignation consists of the way it is handled: the former implies in a \textit{vacant} period in which the role functionality is interrupted until the vacancy is detected and occupied (hard reelection), whereas the later allows the replacement of the position before discontinuing the functionality provided by the resigning node (soft reelection). In both cases, the eligible nodes must proceed with the election of a new candidate.

The later case presents an inverse situation: the caller of the election is another node that perceives its fitness as higher than what the actual node scored by the time it was elected. Thus, it \textit{assumes} this value has not increased, which must be confirmed by letting the actual node to participate in the election. To mitigate the overhead, the following strategy is proposed:

\begin{enumerate}

\item The challenge must only be called if the challenger fitness score ($FS_c$) is greater than the actual fitness score ($FS_e$) by a degree of $\alpha$, i.e., $FS_c = \alpha * FS_e$, with $\alpha > 1$;

\item Assuming all eligible nodes adopt the same $\alpha$, they are exempt of participating and the election happens between the challenger and the actual nodes;

\end{enumerate}

If the challenger node has indeed a higher $FS$ than its challenger, it assumes the role position and this result is advertised; otherwise, the actual fitness score of the elected node ($FS_a$) is advertised so that new challenges are based on the most updated score.



%The first and foremost overhead is due to the exchange of messages among eligible nodes ()

%like the vacancy of a role position due to churn or  resignation of the node in that position, 

%as the new node must be elected and advertised. Second, the context of the role, kept by the leaving node, may have to be transfered to the elected node (if the role is stateful). Depending on the domain of the application, other costs may apply.

%Table~\ref{fig:asymmetry} classifies these methods according to their type and their satisfaction of requirements R[1-3].
%
%\begin{table}[ht!]
%	\centering
%	\begin{tabularx}{\linewidth}{@{}| *1{>{\centering\arraybackslash}X}|c|c|c|@{}}
%		\hline 
%		 & \textbf{R1} & \textbf{R2} & \textbf{R3} \\
%		\hline
%		M1 & Y & N & Y\\
%		\hline 
%		M2 & Y & N & Y\\ 
%		\hline
%		M3 & Y & N & Y\\
%		\hline
%	\end{tabularx}
%	\caption{Distributed allocation methods}
%	\label{tab:role_node_cardinality}
%\end{table}


\subsubsection{\textbf{Example}}

The fitness function of an \textit{aggregator} role in a OMCS application could be modeled as: 

$$
f_{aggregator} = battery\_level * membership\_age
$$

\noindent
in which $battery\_level$ and $membership\_age$ are both real positive numbers ranging from 0 to 1. The former parameter can be accessed from the platform hosting the node, while the last parameter refers to the time a node belongs to a group. The following function illustrates how this parameter could be evaluated:

$$
membership\_age = MIN((60 - age)/60, 1)
$$

\noindent
in which $age$ is given in seconds and refer to the uninterrupted time a given node is in the group. Thus, if the node has been more than a minute in the group, it has a $membership\_age$ of one; otherwise, $0 \le membership\_factor \le 1$.

Thus, given a set of eligible nodes, the one with the highest product of battery level and membership age becomes elected as an aggregator. 

\subsubsection{\textbf{Complexity Analysis}}

%TODO: check the definition of FS and use it accross the paper

The distributed role election involves the exchange of fitness scores among eligible nodes in the advent of the events depicted in the previous subsection. As each node receives the FS from the other nodes, this value is added to the classification list in polynomial time. The number of exchanged messages depend on the network topology and the event that triggers the election. 

In the worse case, represented by an election of a vacant position and no broadcast, each node $e$ in the set of eligible nodes $E$ must send each other an unicast message. If $|E| = n$, the asymptotic message count is given by $O(n * (n-1)) = O(n^2)$. If the nodes in $E$ can communicate through broadcast, this number is reduced to $O(n)$. In its turn, the challenge event produces smaller overhead as the exchange of messages is restricted to a request-response between challenger and challenged and to the subsequent advertisement of the result ($O(n)$ without broadcast, otherwise $(O(1))$).

Once the communication overhead of a single election has been showed, the overall overhead can be estimated by the number of roles in the application and the frequency of events triggering new elections. Whereas the vacancy and resignation event cannot be avoided, the frequency of  challenges depend on the choice for $\alpha$: the higher this factor is, the lower the probability of a challenge to happen. Therefore, unless the quality of the functionality provided by the role is strongly affected by variations in the fitness score to the point of justifying the election overhead, $\alpha$ should be set with higher values. 

%For example, one-out-of-many device may be a better candidate to host the coordinator of the activities performed by a group of application nodes or to use specific sensors to extract information from the physical environment. Whenever asymmetry characterizes a behavior, i.e., not every node must perform it, and the performance, the cost, or other attributes of such behavior are to be considered, the decision of which node to perform it should take the context of the hosting devices into account. In this work, we considered asymmetric tasks whose importance for the system justifies the overhead for their allocation. %We compare our approach with straightforward solutions to demonstrate the feasibility and benefits of a context-dependent allocation. 

%In specific, we focus on the \textit{efficiency, robustness, and fairness} resulting from an appropriate distribution of system responsibilities to local devices taking into account the dynamicity of their resources and capabilities, as well as their history of participation (fairness). To achieve a scalable solution for these problems, we propose a two-level mechanism for the partitioning of the edge space into subset of devices (groups) and the classification of these subsets by their fitness to play different system roles (assignment). 



%the partitioning of the edge space according to different criteria and

 %In special, we are concerned with system functionalities that can assigned to one or more devices within the edge space. 
%The main novelties here are:
%
%\begin{itemize}
%
%\item Pervasive and mobile devices are seen as volatile and heterogeneous computational platforms with resource limitations.
%
%\item Distinct responsibilities may be assigned to a subset of the mobile devices running the same application.
%
%\end{itemize}


%First, we show how our framework can enable ad hoc interactions without external storage and control provided by cloud infrastructure. Second, we present a mobile multi-player game scenario in which low latency communication is the main requirement. Finally, we describe a mobile crowdsensing application, which combines different requirements, including efficient use of battery and mobile data plan. 



%which here are hampered by volatility, 
%
%\begin{itemize}
%
%\item Discovery: devices must be able to recognize each other in the space
%
%\item Transparency: devices must be able to communicate without previous knowledge of their network address
%
%\item Dynamic allocation: devices must be able to agree on the role they will play in the system 
%
%\item Partitioning: devices must be able to further partition the interaction space 
%
%\end{itemize}

\section{Self-adaptation}\label{sec:self_adaptation}

\subsection{Role Cardinality} 

The relation between the number of instances of a given role and different non-functional attributes of a system may be straightforward (e.g., the number of similar sensors in a cluster and the accuracy of the information this cluster provides) or less evident (e.g., the number of replicas of a component and the availability or reliability of the functionality it provides). In addition, considering the scenario of mobile applications, the mapping between role cardinality and these attributes tend to change according to the context they operate, i.e., it may depend on which devices are assigned to the role (e.g., more/less powerful computational resources, more/less accurate sensors) and the state of each device (e.g., more/less battery, more/less signal strength). Hence, in order to avoid violations of these attributes, the cardinality of a role must reflect the circumstances. %To this end, we propose a last, but not least important mechanism for adapting the application organization. 

\subsection{Group Cardinality} 

Since the encounter of mobile application nodes is opportunistic and not deterministic, group membership criteria specifies no size threshold or limit: as long as a node satisfies the group's membership criteria, it becomes/remains a member. This choice avoids the need for a group membership control by a trustful component and simplifies the grouping process. Accordingly, the size of a group, kept stable the membership criteria satisfaction by each of its members, can only be changed by making this criteria more relaxed or strict. 

Whereas the self-organization mechanisms presented address the \textit{solution space}, i.e., which nodes provide which application functionality, here the \textit{requirement space} is the target of the adaptation, i.e., the application requirements evolve to accommodate situations in which no solution can satisfy those requirements~\cite{RELAX}. 
